\documentclass[../main]{subfiles}

\begin{document}

\thispagestyle{fancy}
\setlength{\parindent}{18pt}

\begin{onehalfspacing}

    \vspace*{75pt}

    \section*{\huge Abstract}

    The first part of this thesis is an analysis of the \textit{virtual large cardinals}, being critical points of a set-sized generic elementary embedding with target model being a subset of the ground model. We show that virtually measurables are equiconsistent with virtually strongs, and that virtually Woodins are virtually Vop\v enka. We separate most of these large cardinals, but show that such separations do not hold within core models. We define \textit{prestrong cardinals}, being an equivalent characterisation of strongs, but which in a virtual setting are strictly weaker than virtually strongs. We show that the existence of this separation is equivalent to the existence of virtually rank-into-rank cardinals in the universe, and that virtually Berkeley cardinals can be characterised in the same fashion with $\on$ being virtually \textit{pre-Woodin} but not virtually Woodin, answering a question by Gitman and Hamkins. Building on work of Wilson, we show that the virtual version of the \textit{Weak Vop\v enka Principle} is equivalent to a weakening of virtually pre-Woodins. We end the first part with several indestructibility results, including that a slight strengthening of the virtually supercompacts is always indestructible by ${<}\kappa$-directed closed forcings.

    The second part is concerned with connections between the virtual large cardinals and other set-theoretic objects. We analyse cardinals arising from a certain \textit{filter game}, for various lengths of the game. When the games are finite then we show that this results in a characterisation of the completely ineffable cardinals, and at length $\omega$ we arrive at another characterisation of the virtually measurable cardinals. At length $\omega+1$ the cardinals become equiconsistent with a measurable cardinal, and at uncountable cofinalities the cardinals are downward absolute to $K$ below $0^\pistol$. The results in this section answers most of the open questions raised in \cite{HolySchlicht}. We also introduce the notion of \textit{ideal-absolute} properties of forcings, being properties such that generic elementary embeddings can be characterised by ideals in the ground model. We show that several properties are ideal-absolute, which includes an improvement of an unpublished theorem of Foreman. This also results in another characterisation of completely ineffables.

\end{onehalfspacing}

\end{document}
