\documentclass[../main]{subfiles}
\pagestyle{fancy}

\begin{document}

\chapter{Notation}
\thispagestyle{fancy}

\setlength{\parindent}{18pt}
\begin{onehalfspacing}

We will denote the class of ordinals by $\on$. For $X,Y$ sets we denote by ${^X}Y$ the set of all functions from $X$ to $Y$. For an infinite cardinal $\kappa$, we let $H_\kappa$ be the set of sets $X$ such that the cardinality of the transitive closure of $X$ is ${<}\kappa$. $\zf^-$ will denote $\zf$ with the Collection scheme but without the Power Set axiom, following the results of \cite{ZFwithoutPowerSet}. We write \gbc\ for G\"odel-Bernays class theory with the Axiom of Choice, and \gb\ for \gbc\ without the Axiom of Choice. The symbol $\contr$ will denote a contradiction and $\p(X)$ denotes the power set of $X$. We will sometimes denote elementary embeddings $\pi\colon(\M,\in)\to(\N,\in)$ by simply $\pi\colon\M\to\N$. Generally, $\alpha,\beta,\gamma,\zeta$ will denote ordinals and $\kappa,\lambda,\theta,\delta$ cardinals. We will always assume elementary embeddings to be non-trivial unless otherwise stated, meaning that the elementary embedding in question is not the identity.


\end{onehalfspacing}
\setlength{\parindent}{0pt}

\end{document}
