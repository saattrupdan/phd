\documentclass[../main]{subfiles}
\pagestyle{fancy}

\begin{document}

\chapter{Introduction}
\thispagestyle{fancy}

\setlength{\parindent}{18pt}
\begin{onehalfspacing}

G\"odel proved his Incompleteness Theorems in \cite{godel-incompleteness}, one of which showed that to every consistent sufficiently strong\footnote{Being able to prove \textsf{PA} counts as ``sufficiently strong''.} theory there would be statements which the system can neither prove nor disprove; we say that such a theory is \textit{incomplete} and say that the statement in question is \textit{independent} of the theory. Of notable importance is \zfc, the established foundational theory of Mathematics. Mathematicians at the time were generally disinterested in his result, as they considered the statement he constructed in his proof to be ``unnatural'' and therefore have no real consequence to mathematical practice.

\quad Proceeding G\"odel's proof of the consistency of the \textit{Continuum Hypothesis} in \cite{godel-continuum}, which was also the first problem that appeared on Hilbert's famous list of 23 problems in Mathematics published in year 1900, G\"odel proposed a program in \cite{godel-continuum-problem}, the goal of which was to ``decide interesting mathematical propositions independent of \zfc\ in well-justified extensions of \zfc.'' His result had shown the ``first half'' for the Continuum Hypothesis, namely that \zfc\ cannot disprove it.

\quad The second half of the proof that the Continuum Hypothesis is indeed independent of \zfc\ came about thirty years later, when Cohen used his newly developed notion of \textit{forcing} in \cite{cohen} to prove the consistency of the \textit{negation} of the Continuum Hypothesis, showing that there \textit{are} natural statements which are independent of \zfc.

\quad Today, many others have followed in G\"odel's footsteps and have made great efforts to analyse the nature of these natural independent statements. This organically led to the development of \textit{large cardinal axioms}, being axioms that extend \zfc\ in terms of consistency strength and seem to be the \textit{canonical} such axioms, in that all natural theories found ``in the wild'' have been shown to be equiconsistent with a known large cardinal axiom.

\quad A notable phenomenon is that for ``natural'' theories $T$ and $U$, if $T$ has smaller consistency strength than $U$ then the $\Sigma^0_\omega$ consequences of $T$ are also $\Sigma^0_\omega$ consequences of $U$ -- so by climbing this large cardinal hierarchy we in fact uncover more truths about the natural numbers. The reals \textit{also} attain this monotone behaviour as long as one has moved sufficiently far up the hierarchy, namely past the existence of infinitely many so-called \textit{Woodin cardinals}. This phenomenon also occurs for sets of reals.

\quad Now, it has been found that most large cardinals having the strength of at least a measurable cardinal can be characterised in terms of \textit{elementary embeddings}, enabling a uniform analysis of these cardinals. The large cardinals below the measurables have historically not had such uniform characterisations, but recently the notion of a \textit{virtual large cardinal} was introduced in \cite{Schindler} and \cite{GitmanSchindler} that essentially \textit{reflects} a lot of the behaviour of the larger large cardinals down to the lower realms. Here Cohen's method of forcing is in full force, as the definition of a virtual version of a large cardinal characterised by elementary embeddings is essentially stating that we can \textit{force} such an embedding to exist, rather than postulating their existence in the universe.

\quad This thesis is an extensive analysis of this virtual phenomenon. The thesis naturally splits into two parts, with the first part being an analysis of the virtuals in isolation and the second part being how these virtuals relate to commonly used set-theoretic objects. Chapter~\ref{chapter.virtual-large-cardinals} covers the first part, and Chapters~\ref{chapter.filters-and-games} and \ref{chapter.ideal-absoluteness} the second. 

\quad In the first part we examine how the virtual large cardinals relate to each other, and highlights how they differ from their non-virtual counterparts. A crucial difference between the virtuals and the standard large cardinals is that we do not get a Kunen inconsistency for the virtuals. One consequence of this is that the property that $j(\kappa)>\theta$ always holds when $\kappa$ is a $\theta$-strong cardinal with $j\colon V\to\M$ being the associated elementary embedding, do not always hold in the virtual world. This leads us to define \textit{prestrong} cardinals as the cardinals not having this property, and in Theorem~\ref{theo.virtchar} we characterise the virtual $\theta$-prestrong cardinals into either virtually $\theta$-strong cardinals or virtually $(\theta,\omega)$-superstrong cardinals. One consequence of this is that virtually measurable cardinals are \textit{equiconsistent} with virtually strong cardinals, without being equivalent. Another consequence is Corollary~\ref{coro.rank-into-rank-prestrong-strong} that ``virtual Kunen inconsistencies'', being the existence of virtual rank-into-rank cardinals, happen exactly when we can separate the virtually prestrong cardinals from the virtually strongs.

\quad The virtual large cardinals also differ from the standard large cardinals by how they interact with structures closed under sequences. We first see this difference in Theorem~\ref{theo.rem}, due to Ralf Schindler and Victoria Gitman, who showed that the virtually strongs are equivalent to the virtually supercompacts. This is expanded to virtually Woodin cardinals in Proposition~\ref{prop.woodin}, yielding a plethora of characterisations of these cardinals, as well as in Theorem~\ref{theo.vopwood}, where we show that the Woodin cardinals and the Vop\v enka cardinals are also equivalent in the virtual world. These two results are joint with Stamatis Dimopoulos and Victoria Gitman.

\quad We next delve into a weak version of the Vop\v enka principle, denoted \wvp, which originates from category theory. Trevor Wilson has shown that \wvp\ is equivalent to \on\ being a Woodin cardinal, and we show that this equivalence \textit{only} holds in the virtual world if we work with \textit{pre}-Woodin cardinals, in analogy with the prestrongs mentioned above, as well as not explicitly requiring the target model to be well-founded. This result is joint with Victoria Gitman.

\quad Since there are no Kunen inconsistencies in the virtual world, this allows us to study the virtual versions of the \textit{Berkeley cardinals} in \zfc. We introduce these and show that in Theorem~\ref{theo.berkeleyequiv} that the virtual Vop\v enka principle implies that \on\ is Mahlo \textit{exactly} when there are no virtually Berkeley cardinals, improving on a result by Victoria Gitman and Joel Hamkins. We furthermore show that the virtually Berkeley cardinals exist exactly when \on\ is virtually pre-Woodin without being virtually Woodin, which parallels the result for the rank-into-rank cardinals mentioned above. This also hints at Berkeley cardinals being a natural large cardinal notion.

\quad The virtual large cardinals all require the target model of the generic elementary embedding to be a subset of the ground model, and if we remove this condition then we get the \textit{faint} large cardinals. We show in Corollary~\ref{coro.tall-separation} that the virtuals and faints are consistently distinct notions. We provide further separations in Theorem~\ref{theo.virtualsep}, this theorem being joint with Victoria Gitman. However, in Theorem~\ref{theo.core-model-behaviour} we show that in $L$, $L[\mu]$ as well as in the core model $K$ below a Woodin cardinal, the two notions are equivalent.

\quad The first part ends with an analysis of indestructibility properties of the faints, which is joint with Philipp Schlicht. We work with a strengthening of the faintly supercompact cardinals in which the target model is closed under sequences in the generic extension and not just in the ground model. We show that these cardinals enjoy many indestructibility properties, including ${<}\kappa$-directed closed forcings and $\add(\omega,\kappa)$. In an attempt to understand how strong these cardinals are, we show that no such cardinals can exist in neither $L$ nor $L[\mu]$. Using the stationary tower we can show that a proper class of Woodin cardinals is an upper bound, but a recent unpublished result by Toshimichi Usuba shows that, surprisingly, virtually extendibles also provide an upper consistency bound for these cardinals.

\quad The second part is split into two chapters, the first one exploring filters and games, with ideals being the focus of the second. We perform a thorough analysis of certain Ramsey-like cardinals introduced by Peter Holy and Philipp Schlicht, defined using \textit{filter games}, in which player I plays set-sized structures $\M_\alpha$ and player II has to follow up with $\M_\alpha$-filters on the cardinal $\kappa$ in question. They focused on the case in which player I doesn't have a winning strategy, where they showed that this results in a large cardinal notion characterised by elementary embeddings between set-sized structures. 

\quad Our main focus is when player II \textit{does} have a winning strategy. When the games are of finite length we show in Theorems~\ref{theo.ind} and \ref{theo.normind} that the resulting large cardinal notions form a strict hierarchy via the use of indescribability properties, and characterise in Theorem~\ref{theo.ineff} the completely ineffable cardinals with these games.

\quad As we move to infinite games this is when we reach the connection to the virtual large cardinals. Indeed, Theorem~\ref{theo.gengame} shows that the faintly $\theta$-measurable cardinals can be characterised in terms of a slight weaknening of the $\omega$-length version of these games. Theorem~\ref{theo.virtstrat} shows that this weakened game is equivalent to the original game in $L$, and is related to the above-mentioned separation- and core model results. These two theorems are joint with Ralf Schindler. 

\quad Taking one more step, to games of length $\omega+1$, our consistency strength suddenly dramatically increases to measurable cardinals, as shown in Theorem~\ref{theo.omegaplusone}. For these countable length games we show that our resulting large cardinals can be characterised in terms of the \textit{indiscernible games} introduced in \cite{SharpeWelch}. We also include proofs due to Philip Welch and Ralf Schindler that the cardinals corresponding to the $\omega_1$-length games are measurable in $K$ below $0^\pistol$ and a Woodin cardinal, respectively.

\quad The last case is when the games have length of uncountable cofinality, where we show that the resulting large cardinals are downward absolute to $K$ below $0^\pistol$, the sharp of a strong cardinal. We also show how the cardinals relate to the \textit{strongly-} and \textit{super Ramseys} introduced by Victoria Gitman.

\quad The following chapter asks the question of when these cardinals characterised by generic elementary embeddings can equivalently be characterised by the existence of ideals in the ground model. To organise our results we define a poset property to be \textit{ideal-absolute} if this holds for forcings having that property. We show in Theorem~\ref{theo.distributive-ideal-abs} that distributivity properties are ideal-absolute and Theorem~\ref{theo.hopelessideal} and the subsequent Corollary~\ref{coro.powerclosed} show that $(\kappa,\kappa)$-distributive ${<}\lambda$-closure is also ideal-absolute, for $\lambda\in[\omega_1,\kappa^+]$. This main result is an improvement of the proof of an unpublished result due to Matthew Foreman, Theorem~\ref{theo.foreman}. 

\quad Building on these results, we give in Corollary~\ref{coro.ineff} another characterisation of the completely ineffables, in terms of ideals, and in Theorem~\ref{theo.hopelessideal2} and Corollary~\ref{coro.cohen} we show that ${<}\lambda$-closure is also ideal-absolute. This ties in with the above-mentioned weakening of the games, also showing that these games characterise the ${<}\lambda$-closed faintly- and ideally measurables.

\quad We end with a final chapter containing a range of open questions, continuing on from our results in this thesis.

\end{onehalfspacing}
\setlength{\parindent}{0pt}

\end{document}
