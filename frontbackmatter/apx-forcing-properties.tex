\documentclass[../main]{subfiles}
\pagestyle{fancy}

\begin{document}

\chapter{Forcing properties}
\label{apx.forcing}
\thispagestyle{fancy}

\section{Lifting Criterion}
When we are working with an elementary embedding $\pi\colon\M\to\N$ between sets in the universe, we would sometimes like to lift such an embedding to a generic extension, meaning that given a forcing notion $\mathbb P\in\M$ and an $\M$-generic $g\subset\mathbb P$, we're interested in when we can lift $\pi$ to an embedding
\eq{
  \pi^+\colon\M[g]\to\N[h],
}

where $h\subset\pi(\mathbb P)$ is $\N$-generic. The \textbf{lifting criterion} shows exactly when this is possible. The following is Proposition 9.1 in \cite{handbook-cummings}.

\prop[The Lifting Criterion][prop.lifting-criterion]{
  Let $\pi\colon\M\to\N$ be an elementary embedding between weak $\kappa$-models. Fix a forcing notion $\mathbb P\in\M$, an $\M$-generic $g\subset\mathbb P$ and an $\N$-generic $h\subset\pi(\mathbb P)$. Then the following are equivalent:
  \begin{itemize}
    \item $\pi[g]\subset h$;
    \item There exists an elementary $\pi^+\colon\M[g]\to\N[h]$ such that $\pi^+(g)=h$ and $\pi^+\restr\M = \pi$.
  \end{itemize}
}
\proof{
  $(ii)\Rightarrow(i)$ is clear, so assume $(i)$. Define $\pi^+\colon\M[g]\to\N[h]$ as $\pi^+(\dot\tau^g) := \pi(\tau)^h$. To see that $\pi^+$ is well-defined fix $\dot\sigma,\dot\tau\in\M^{\mathbb P}$ such that $\dot\sigma^g=\dot\tau^g$, and fix $p\in g$ such that $p\forces\dot\sigma=\dot\tau$. By elementarity $\pi(p)\forces\pi(\dot\sigma)=\pi(\dot\tau)$, so since $\pi(p)\in h$ by $(i)$ we get that $\pi(\dot\sigma)^h=\pi(\dot\tau)^h$.

  \qquad To show elementarity, note that for $x\in\M$ it holds that $\pi(\check x) = \check{\pi(x)}$, implying $\pi^+(x)=\pi^+(\check x^g) = \pi(\check x)^h = \pi(x)$. Further, letting $\dot g\in\M^{\mathbb P}$ be the standard $\mathbb P$-name for $g$, then $\pi(\dot g)$ is the standard $\pi(\mathbb P)$-name for $h$ and therefore $\pi^+(g)=h$.
}

\section{Countable Embedding Absoluteness}
A key folklore lemma which we will frequently need when dealing with elementary embeddings existing in generic extensions is the following.

\lemm[Countable Embedding Absoluteness][lemm.ctblabs]{
  Let $\M,\N$ be sets, $\P$ a transitive class with $\M,\N\in\P$, and let $\pi\colon\M\to\N$ be an elementary embedding. Assume that $\P\models\zf^-+\dc+\godel{\text{$\M$ is countable}}$ and fix any finite $X\subset\M$.
  
  \qquad Then $\P$ contains an elementary embedding $\pi^*\colon\M\to\N$ which agrees with $\pi$ on $X$. If $\pi$ has a critical point and if $\M$ is transitive then we can also assume that $\crit\pi=\crit\pi^*$.\footnote{We are using transitivity of $\M$ to ensure that the \textit{ordinal} $\crit\pi$ exists.}
}
\proof{
  Let $\{a_i\mid i<\omega\}\in\P$ be an enumeration of $\M$ and set $\M\restr n:=\{a_i\mid i<n\}$. Then, in $\P$, build the tree $\T$ of all partial isomorphisms between $\M\restr n$ and $\N$ for $n<\omega$, ordered by extension. Then $\T$ is illfounded in $V$ by assumption, so it's also illfounded in $\P$ since $\P$ is transitive and $\P\models\zf^-+\dc$. The branch then gives us the embedding $\pi^*$, and if $\crit\pi$ exists then we can ensure that it agrees with $\pi$ on the critical point and finitely many values by adding these conditions to $\T$.
}

\section{Preservation of sequence closure}
The following lemma is from \cite{lucke-schlicht-notes} and gives a useful condition on when sequence closure is preserved when moving to generic extensions.

\lemm[][lemm.lucke-schlicht]{
  Let $\lambda$ be an infinite cardinal, $\M\models\zf^-$ a transitive model, $\mathbb P\in\M$ a $\lambda^+$-cc forcing notion and $g\subset\mathbb P$ an $\M$-generic filter. Then $V\models{^\lambda}\M\subset\M$ implies that $V[g]\models{^\lambda}\M\subset\M$.
}
\proof{
  Work in $V[g]$. Let $c:=\bra{c_\alpha\mid\alpha<\lambda}$ be a $\lambda$-sequence such that $c_\alpha\in\M[g]$ for every $\alpha<\lambda$. Fix for every $\alpha<\lambda$ a $\mathbb P$-name $\dot c_\alpha$ such that $\dot c_\alpha^g=c_\alpha$. Also let $\dot a$ be a $\mathbb P$-name with $\dot a^g=\bra{\dot c_\alpha\mid\alpha<\lambda}$ and choose $p\in g$ such that $V\models\godel{p\forces\forall\alpha<\check\lambda\colon\dot a(\alpha)\in\M^{\mathbb P}}$.

  \qquad Now, working in $V$, there is for each $\alpha<\lambda$ a maximal antichain $A_\alpha$ below $p$ such that every $q\in A_\alpha$ decides $\dot a(\alpha)$; i.e., $q\forces\godel{\dot a(\alpha)=\check x}$ for some $x\in\M$. Define now
  \eq{
    \sigma := \{((\alpha,x),q)\mid\alpha\in\lambda\land q\in A_\alpha\land q\forces\godel{\dot a(\alpha)=\check x}\}.
  }

  Then $p\forces\godel{\sigma=\dot a}$. Note that $\abs{\sigma}\leq\lambda$, since $\abs{A_\alpha}\leq\lambda$ for each $\alpha<\lambda$. Thus $\sigma\in\M$. Now it holds that
  \eq{
    V[g]\models\godel{\bra{\dot c_\alpha\mid\alpha<\lambda}=\dot a^g=\sigma^g\in\M[g]},
  }
  
  and we can compute $c=\bra{c_\alpha\mid\alpha<\lambda}=\bra{\dot c_\alpha^g\mid\alpha<\lambda}$ from $\bra{\dot c_\alpha\mid\alpha<\lambda}$ and $g$, so $c\in\M[g]$ by Replacement.
}

\end{document}
