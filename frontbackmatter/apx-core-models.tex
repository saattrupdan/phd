\documentclass[../main]{subfiles}
\pagestyle{fancy}

\begin{document}

\chapter{Core model theory}
\label{apx.core-model-theory}
\thispagestyle{fancy}

As we will be utilising the \textit{core model} at various points throughout this thesis, we give here an idea of what we mean by the \textit{core model}. A convenient feature of core model theory is that most of the technical details regarding the construction is not needed for applications; it suffices to know only its abstract properties. That being said, we \textit{will} provide a glimpse of the construction at the end of this section. To see the full construction we refer the interested reader to \cite{MSc}, \cite{Zeman} and \cite{JensenSteel}. 


\section{The core model $K$}

The \textbf{core model}\footnote{$K$ is short for \textit{Kern}, meaning \textit{core} in German.} $K$ of a universe is the roughly speaking the subuniverse that strikes a balance between retaining the complexity of the universe while being as simple as possible. The problem is then making all of this precise. Some aspects of the definition is agreed upon by most researchers:
\begin{enumerate}
  \item We choose to define the \textit{complexity} of a universe by its large cardinal structure. This is based on the empirical fact that large cardinals seem to capture the strength of every ``naturally defined'' hypothesis, and gives us a convenient yard stick. For instance, a universe containing a measurable cardinal is more complex than $L$, as Scott's Theorem \ref{theo.scott} shows that $L$ cannot contain any measurable cardinals (or any large cardinals stronger than measurables);
  \item We further postulate that $L$ is the simplest universe there is, and the simplicity of a universe should therefore be measured in terms of how much it resembles $L$. We will be more precise about what it means to ``resemble $L$'' below, but with this intuitive notion is should at least be clear that, say, $L$ is simpler than $L[\mu]$.\\
\end{enumerate}

Even though $(i)$ captures what we mean by complexity, it leaves much to be desired. For instance, as the structure of the large cardinal hierarchy can only be verified empirically, we might end up in an unfortunate situation where we simply do not know whether a given universe is more complex than another one\footnote{It might also be the case that the large cardinal hierarchy is not linear at all.}. The famous example of this is the current situation with the superstrong- and strongly compact cardinals, that we simply do not know which one is stronger\footnote{Although the general consensus is that the strongly compact cardinals should be equiconsistent with the supercompacts, making them stronger than the superstrongs.}. Thus, given a universe whose strength corresponds to that of a strongly compact and another one at the level of superstrongs, we would not be able to say which one is more complex. 

\qquad To remedy this unfortunate situation, we choose instead to define the complexity of a universe in terms of an intermediate property. A universe satisfying this property should then entail that it inherits the large cardinal structure of its surrounding universe. All the intermediate properties currently being used are all instances of a general phenomenon called \textit{covering}. The intuitive idea is that every set in the universe can be ``approximated'' by a set in the subuniverse, and arose from a seminal theorem of Jensen, see \cite[Theorem 11.56]{SchindlerBook}, stating that $0^\sharp$ exists if and only if \textit{strong covering} fails for $L$, defined as follows.

\defi[Jensen]{
  We say that \textbf{strong covering} holds for universes $\U\subset\V$ if to every $\alpha<o(\V)$ and $X\in\p^{\V}(\alpha)$ there exists $A\in\U$ such that $X\subset A$ and $\Card^{\V}(X) = \Card^{\V}(A)$.
}

We can then interpret Jensen's result as saying that, if the complexity of the surrounding universe $\V$ is below the strength of $0^\sharp$ then $L$ is a good candidate for $K$. In a complex universe we would therefore be looking for the core model among subuniverses more complex than $L$, and it turns out that also requiring strong covering to hold in such models is too much to ask; the current definition of covering has thus been weakened to the following.

\defi[][defi.weak-covering]{
  We say that \textbf{(weak) covering} holds for universes $\U\subset\V$ if $\cof^{\V}(\alpha^{+\U}) = \Card^{\V}(\alpha^{+\U})$ holds for any ordinal $\alpha$ with $\alpha^{+\U}\geq\aleph_2^{\V}$.
}

This statement might seem very distant from the strong version, but one can think of weak covering as saying that $\U$ ``knows'' the true cofinality of its successor cardinals $\kappa\geq\aleph_2^{\V}$ within the error margin $\varepsilon:=\kappa^{+\U}-\Card^{\V}(\kappa)$. More concretely, we could equivalently define weak covering as $\U$ containing all cofinal maps $f\colon\gamma\to\kappa$ in $\V$ for every $\gamma\in\Card^{\V}(\kappa)$, making it closer in spirit to the strong covering property.

\figx{weak-covering.jpg}{Weak covering property}

\qquad When it comes to $(ii)$ we have to define what we mean by ``resembling $L$''. Ultimately this boils down to the current working definition of a \textit{mouse} and is still work in progress. If our universe is no more complex than the strength of a Woodin cardinal however, then we know what the correct definition of a mouse is, and hence also what ``resembling $L$'' would mean in this context. The definition of mice along with the assumption of covering then turns out to imply that the core model will indeed inherit the large cardinal strength of the universe\footnote{To show this one first uses covering to show that $K$ is \textit{universal}, i.e. that it wins every coiteration. With universality at hand, a comparison argument with any $L[\vec E]$-model containing a large cardinal will then show that $K$ will have an inner model with the large cardinal in question.}.

\qquad To construct the core model one could then take a bottom-up approach, starting with $L$ and then carefully include the complexity of the universe while remaining similar to $L$\footnote{This is the strategy undertaken by Steel and Sargsyan.}. Alternatively, a top-down approach would be to define a structure which has \textit{all} the complexity of the universe, and then showing that this structure indeed exhibits these $L$-like properties\footnote{Woodin is pursuing this path.}.

\section{Constructing $K$}
The standard construction of $K$ takes the bottom-up approach. The first step towards this is the construction of $K^c$\footnote{The ``c'' stands for \textit{certified}, as the extenders we put on the sequence was historically called \textit{certified extenders}.}, which we build by recursion on the ordinals. We start with $K^c_0 := \emptyset$ and at every successor ordinal $\alpha$ we do one of two things:
\begin{enumerate}
  \item If there exists a ``nice'' extender indexed at $\alpha$ then we put it onto the extender sequence of $\core(K^c_\alpha$), where $\core(X)$ is the transitive collapse of a certain hull of $X$\footnote{Think of $\core(X)$ as ``removing the noise of $X$''.};
  \item Otherwise we let $K^c_\alpha := \J(\core(K^c_{\alpha-1}))$, with $\J(x):=\text{rud}(\trcl(x\cup\{x\}))$ being the usual operator we use to build $L$ with Jensen's hierarchy.\\
\end{enumerate}

In other words, we are essentially building $L$ with extenders attached onto it in a canonical fashion. Taking cores at every step will ensure that the initial segments will be \textit{sound}, which ultimately is what guarantees iterability of $K^c$. The fact that we put on all the relevant extenders from $V$ is what will ensure the covering property of the model. It turns out that $K^c$ isn't exactly what we want however, as it relies \textit{too much} on the surrounding universe, in contrast with $L$ whose construction procedure builds the exact same model in every universe. To attain this \textit{canonicity} we are again taking certain ``thick'' hulls of $K^c$ (again, think of it as removing the noise). The resulting construction \textit{almost} gives us what we want and is dubbed \textit{pseudo-K}. The problem with this is that the technicalities of the construction uses certain properties of a fixed cardinal $\Omega$, so to build the true core model we ``glue'' these pseudo-K's together.

\qquad The takeaway here is that whenever we're working with an initial segment of $K$ then that segment will be build using the recursive steps $(i)$ and $(ii)$ above, carefully including extenders from $V$.

\section{Properties of $K$}

In terms of applications of core model theory, the properties of $K$ is usually what matters. We touched on the weak covering property above, but for completeness we state most of the properties usually employed when working with $K$, here. In \cite{JensenSteel} they isolate a set of properties of $K$ which leads them to \textit{define} $K$ as the structure satisfying the conjunction of these properties. These are as follows.\footnote{See REF for definitions of premice, iteration trees and iteration strategies.}

\begin{enumerate}
  \item $K$ is a transitive proper class premouse satisfying \zfc;
  \item $K$ is $\Sigma_2$-definable;
  \item $K$ has a $\Sigma_2$-definable iteration strategy $\Sigma$;
  \item $K$ is \textbf{generically absolute}, meaning that $K^V=K^{V[g]}$ and $\Sigma^{V[g]}\restr V=\Sigma^V$ for any $V$-generic filter $g\subset\mathbb P$ for a set-sized forcing notion $\mathbb P$;
  \item $K$ is \textbf{inductively defined}, meaning that $K\l\omega_1^V$ is $\Sigma_1$-definable over $J_{\omega_1}(\mathbb R)$;
  \item $K$ satisfies \textbf{weak covering} as in Definition \ref{defi.weak-covering}.\\
\end{enumerate}

On top of these properties, we will also employ the following property, which is proven in Lemmata 7.3.7--7.3.9 and 8.3.4 in \cite{Zeman}.\todo{Check if Zeman is the author}

\qtheo[Zeman][theo.beaver]{
  Assume $0^\pistol$ does not exist. If $\mu$ is a countably complete weakly amenable $K$-measure then $\mu\in K$.
}


\end{document}
