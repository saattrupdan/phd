\documentclass[../main]{subfiles}
\pagestyle{fancy}

\begin{document}

\chapter{The Internal Core Model Induction}
\label{ch1}
\thispagestyle{fancy}

\todo[inline]{Introduction}

\section{Operators and hybrid mice}

\todo[inline]{Define model operator, (hybrid) mouse operator, mouse reflection, condenses finely/well, determines itself on generic extensions, relativises well...}

\section{The hybrid core model dichotomy}

\lemm[][lemm.opit]{
  Let $\theta$ be a regular uncountable cardinal or
  $\theta=\infty$ and let $\N$ be a tame hybrid mouse
  operator on $H_\theta$ which relativises well. Then $\N$ is
  countably iterable iff it's $(\theta,\theta)$-iterable, guided by
  $\N$. Furthermore, for every $x\in H_\theta$, if $M_1^{\N}(x)$ exists and is countably iterable, then it's also
  $(\theta,\theta)$-iterable, guided by $\N$.\todo{Change this to model operators; perhaps change parts of the proof and/or assumptions needed.}
  }
  \proof{
    We first show
  that $\N(x)$ is $(\theta,\theta)$-iterable. Let $\T\in H_\theta$ be
  a normal tree of limit length on $\N(x)$. Let $\eta\gg\rk(\T)$ and
  let
    \eq{
      \h:= \chull^{H_\eta}(\{x,\N(x),\T\})
    }
    
  with uncollapse $\pi\colon\h\to H_\eta$. Set
  $\overline a:=\pi^{-1}(a)$ for every $a\in\ran\pi$. Note that
  $\overline{\N(x)}=\N(\overline x)$ since $\N$ relativises well. Now
  $\overline\T$ is a normal, countable iteration tree on
  $\N(\overline x)$ and hence our iteration strategy yields a
  wellfounded cofinal branch $\overline b\in V$ for
  $\overline\T$. Note that $\overline\Q:=\Q(\overline b,\overline\T)$
  exists, since if $\overline b$ drops then there's nothing to do, and
  otherwise we have that \eq{ \rho_{1}(\M^{\overline\T}_{\overline
      b})=\rho_{1}(\N(\overline x))=\rk\overline
    x<\delta(\overline\T),
  }

  so $\delta(\overline\T)$ is not definably Woodin over
  $\M^{\overline\T}_{\overline b}$.

  \clai{
    $\overline\Q\init\N(\M(\overline\T))$
  }

  \cproof{
    If $\overline\Q=\M(\overline\T)$ then the claim is trivial,
    so assume that $\M(\overline\T)\pinit\overline\Q$. Note that
    $\overline\Q\init M_{\overline b}^{\overline\T}$ by definition of
    $\Q$-structures, and that $M_{\overline b}^{\overline\T}$
    satisfies $(2)$ of the definition of relativises well, meaning
    that
    
    \eq{
      M_{\overline b}^{\overline\T}\models\godel{\text{$\forall\eta\forall\zeta>\eta:$ if $\eta$ is a cutpoint then $M_{\overline b}^{\overline\T}|\zeta\not\models\varphi_{\N}[\bar x,p_{\N}]$}}.\tag*{(1)} 
    }

    This statement is $\Pi^1_2$ and $\overline\Q$ is $\Pi^1_2$-correct
    since it contains a Woodin cardinal, so that $\Q$ satisfies the
    statement as well. Since $\N$ is tame we get that
    $\delta(\overline\T)$ is a cutpoint of $\overline\Q$, so that
    $\N(\M(\overline\T))=\N(\overline\Q|\delta(\overline\T))$ is
    \textit{not} a proper initial segment of $\overline\Q$. Further,
    as we're assuming that both $\N(\M(\overline\T))$ and
    $\M^{\overline\T}_{\overline b}$ are $(\omega_{1}{+}1)$-iterable
    above $\delta(\overline\T)$ the same thing holds for
    $\overline\Q\init\M_{\overline b}^{\overline\T}$, so that we can
    compare $\N(\M(\overline\T))$ with $\overline\Q$ (in $V$). Let
    \eq{ (\N(\M(\overline\T)),\overline\Q) \leadsto (\P,\R) }

    be the result of the coiteration. We claim that
    $\R\init\P$. Suppose $\P\pinit\R$. Then there is no drop in
    $\N(\M(\overline\T))\leadsto\P$ and in fact
    $\N(\M(\overline\T))=\P$ since $\N(\M(\overline\T))$ projects to
    $\delta(\overline\T)$. Furthermore, as we established that
    $\N(\M(\overline\T))=\N(\overline\Q|\delta(\overline\T))$ isn't a
    proper initial segment of $\overline\Q$ it can't be a proper
    initial segment of $\R$ either, as the coiteration is above
    $\delta(\overline\T)$. But we're assuming that
    $\N(\M(\overline\T))=\P\pinit\R$, a contradiction. So $\R\init\P$.
        
    \qquad Since $\N(\M(\overline\T))$ and $\overline\Q$ agree up to
    $\delta(\overline\T)$ and there is no drop $\overline\Q\leadsto\R$
    we have that $\overline\Q=\R$. If $\N(\M(\overline\T))\leadsto\P$
    doesn't move either we're done, so assume not. Let $F$ be the
    first exit extender of $\N(\M(\overline\T))$ in the
    coiteration. We have $\lh(F) \le o(\overline\Q)$,
    $\overline\Q\init\P$ and $\lh(F)$ is a cardinal in $\P$.
        
    \qquad As $\overline\Q$ is $\delta(\overline\T)$-sound and
    projects to $\delta(\overline\T)$ it follows that
    $J(\overline\Q|\lh(F))$ collapses $\lh(F)$, so it has to be the
    case that $\overline\Q|\lh(F)=\P$ and thus $o(\P)=\lh(F)$. But
    this means that $\P=\N(\M(\overline\T))$ even though we assumed
    that $\N(\M(\T))\leadsto\P$ moved, a contradiction.
    }

  \qquad Now, in a sufficiently large collapsing extension extension
  of $\h$, $\overline b$ is the unique cofinal, wellfounded branch of
  $\overline\T$ such that
  $\Q(\overline b,\overline\T) \init\N(\M(\overline\T))$
  exists. Hence, by the homogeneity of $\col(\omega,\theta)$,
  $\overline b \in H$. By elementarity there is a unique cofinal,
  wellfounded branch $b$ of $\T$ such that $\Q(b,\T)
  \init\N(\M(\T))$. This proves that $M$ is (uniquely) $\on$-iterable
  and virtually the same argument yields the iterability of $M$ via
  successor-many stacks of normal trees.
  
  \qquad To show that $M$ is fully iterable, it remains to be seen
  that the unique iteration strategy (guided by $\N$) of $M$ outlined
  above leads to wellfounded direct limits for stacks of normal trees
  on $M$ of limit length. Let $\lambda$ be a limit ordinal and
  $\vec\T = (\T_i \mid i<\lambda)$ a stack according to our iteration
  strategy. Suppose $\lim_{i<\lambda}\M^{\T_i}_\infty$ is illfounded.
  
  \qquad Redefine $\eta\gg\rk(\vec\T)$,
  $\h:=\chull^{H_\eta}(\{x,M,\vec\T\})$ and $\pi:\h\to H_\eta$ the
  uncollapse, again with $\overline a:=\pi^{-1}(a)$ for every
  $a\in\ran\pi$. By elementarity we get that
  $\h\models\godel{\lim_{i<\overline\lambda}\M^{\overline\T_i}_\infty\text{ is illfounded}}$. But $\overline{\vec\T}$ is countable and
  according to the iteration strategy guided by $\N$, so that
    \eq{
    V\models\godel{\lim_{i<\overline\lambda}\M^{\overline\T_i}_\infty\text{ is wellfounded.}}
    }
  
  Now note that
  $(\lim_{i<\overline\lambda}\M^{\overline\T_i}_\infty)^{\h}=(\lim_{i<\overline\lambda}\M^{\overline\T_i}_\infty)^V$
  and wellfoundedness is absolute between $\h$ and $V$, a
  contradiction.
    
  \qquad Now assume that $M_1^{\N}(x)$ exists for some
  $x\in H_\theta$, and that it's countably iterable. We then do
  exactly the same thing as with $\N(x)$ \textit{except} that in the
  claim we replace $(1)$ with
  \eq{
    \overline\Q\models\forall\eta(\overline\Q|\eta\not\models\godel{\text{$\delta(\overline\T)$ is not Woodin}}),
  }

  so that if $\P\pinit\R$ then $\delta(\overline\T)$ is still Woodin
  in $\P=\N(\M(\overline\T))$, contradicting the defining property of
  $M_1^{\N}(x)$ (and thus also of $\R$). The rest of the proof is a
  copy of the above.
  }

\theo[Hybrid core model dichotomy][theo.kdichotomy]{
  \index{Hybrid core model dichotomy}
  Let $\theta$ be a $\beth$-fixed point or $\theta=\infty$, and let $F$ be a tame \todo{I don't think tame is needed here, as we're only indexing extenders at $F$-initial segments. -Dan} model operator on $H_\theta$ that condenses well. Let $x\in H_\theta$. Then either:
  \begin{enumerate}
    \item The core model $K^F(x)|\theta$ exists and is $(\theta,\theta)$-iterable; or
    \item $M_1^F(x)$ exists and is $(\theta,\theta)$-iterable.
  \end{enumerate}
}
\proof{
  Assume first that $K^{c,F}(x)|\theta$ reaches a premouse which isn't $F$-small; let $\N_\xi$ be the first part of the construction witnessing this. Then $\core(\N_\xi)=M_1^F(x)$\todo{Insert argument?}, and by Lemma \ref{lemm.opit} it suffices to show that $M_1^F(x)$ is countably iterable.\todo[inline]{Show that $M_1^F(x)$ is countably iterable.}

  \qquad We can thus assume that $K^{c,F}(x)|\theta$ is $F$-small. Note that if $K^{c,F}(x)|\theta$ has a Woodin cardinal then because the model is $F$-closed we contradict $F$-smallness, so the model has no Woodin cardinals either, making it $(\theta,\theta)$-iterable.
    
  \qquad Let $\kappa<\theta$ be any uncountable cardinal and let
  $\Omega:=\beth_\kappa(\kappa)^+$. Note that $\Omega<\theta$ since we
  assumed that $\theta$ is a $\beth$-fixed point and
  $\kappa<\theta$. If $\Omega$ is a limit cardinal in
  $K^{c,F}(x)|\theta$ then let $\S:=\lp(K^{c,F}(x)|\Omega)$ and
  otherwise let $\S:=K^{c,F}(x)|\Omega$. Then by Lemma 3.3 of
  \cite{mousestack} we get that $\S$ is countably iterable, with
  largest cardinal $\Omega$ in the ``limit cardinal case''.
    
  \qquad This also means that $\Omega$ isn't Woodin in $L[\S]$, as
  it's trivial in the case where $\Omega$ is a successor cardinal of
  $K^{c,F}(x)|\theta$ by our case assumption, and in the ``limit
  cardinal case'' it also holds since \eq{
    K^{c,F}(x)|\Omega^{+K^{c,F}(x)|\theta}\subseteq\S.  }
    
  By \cite{fernandes} and \cite{JensenSteel} this means that we can
  build $K^{F}(x)|\kappa$, as the only places they use that there's
  no inner model with a Woodin are to guarantee that
  $K^{c,F}(x)|\Omega$ exists and has no Woodin cardinals, and in
  Lemma 4.27 of \cite{JensenSteel} in which they only require that
  $\Omega$ isn't Woodin in $L[\S]$.
    
  \qquad As $\kappa<\theta$ was arbitrary we then get that
  $K^{F}(x)|\theta$ exists. Note that $K^{F}(x)|\theta$ has no
  Woodin cardinals either and is $F$-small, so that $\Q$-structures trivially exist,
  making it $(\theta,\theta)$-iterable.
}



\section{The hybrid witness equivalence}

\defi{
  asd\todo[inline]{Define coarse $(k,U,x)$-Woodin pairs}
}

\defi{
  \index{Coarse mouse witness condition}\index{W$^*_\alpha(F)$}
  Let $F$ be a total condensing operator and let $\alpha$ be an ordinal. Then the \emph{coarse mouse witness condition at $\alpha$ with $F$}, written $W^*_\alpha(F)$, states that given any scaled-co-scaled $U\subset\mathbb R$ whose associated sequences of prewellorderings are elements of $\lp^F_\alpha(\mathbb R)$, we have for every $k<\omega$ and $x\in\mathbb R$ a coarse $(k,U,x)$-Woodin pair $(N,\Sigma)$ with $\Sigma\restr\hc\in\lp^F_\alpha(\mathbb R)$.\todo{Check if this is a reasonable definition.}
}

\theo[Hybrid witness equivalence][theo.witness]{
  \index{Hybrid witness equivalence}
  Let $\theta>0$ be a cardinal, $g\subset\col(\omega,{<}\theta)$ $V$-generic,  $\mathbb R^g:=\bigcup_{\alpha<\theta}\mathbb R^{V[g\restr\alpha]}$, $F$ a total radiant operator and $\alpha$ a critical ordinal of $\lp^F(\mathbb R^g)$. Assume that $\lp^F(\mathbb R^g)\models\dc+\godel{W^*_\beta(F)\text{ holds for all }\beta\leq\alpha}$. Then there is a hybrid mouse operator $\N\in V$ on $H_{\aleph_1^{V[g]}}$ such that
\eq{
  \lp^F(\mathbb R^g)\models W^*_{\alpha+1}(F)\quad\text{iff}\quad V\models\godel{\text{$M_n^{\N}$ is total on $H_{\aleph_1^{V[g]}}$ for all $n<\omega$}}
}

Furthermore, if $\theta<\aleph_1^V$ then we only need to assume that $F$ is total and condensing.
}

\todo[inline]{Be more explicit about what the given operator $\N$ looks like.}



\section{Determinacy in mice from $\di$}

\prop[][prop.mousereflection]{
  If $\omega_1$ carries a saturated ideal
  then mouse reflection holds at
  $\omega_1$.
}
\proof{
  Let $\N$ be a mouse operator defined on $\hc$
  and fix some $x\in H_{\omega_2}$; we want to show that $\N(x)$ is
  defined. Let $j:V\to M$ be the generic ultrapower with
  $\crit j=\omega_1^V$ and note that
  $j(\omega_1^V)=\omega_1^M=\omega_1^{V[g]}=\omega_2^V$ by saturation
  of the ideal. This means in particular that
  $\hc\prec H_{\omega_2}^M$. Since
  \eq{
    \hc\models\godel{\text{$\N(y)$ exists for all sets $y$}}
  }

  we get that $H_{\omega_2}^M$
  believes the same is true. But
  $H_{\omega_2}^V\subset H_{\omega_2}^M$ since $\crit j=\omega_1^V$,
  so that in particular $H_{\omega_2}^M$ believes that $x^\sharp$
  exists. Since $M$ is closed under $\omega$-sequences in
  $V[g]$ by Proposition \ref{prop.ideal}, we get that $x^\sharp$
  exists in $V[g]$ and hence also in $V$ as set forcing can't add
  sharps.\todo{Prove this or give a reference.}
}

\prop[][prop.sharps]{
  If $\omega_1$ carries a precipitous ideal then
  $\hc$ is closed under sharps. If the ideal is furthermore saturated
  then $H_{\omega_2}$ is closed under sharps.
}
\proof{
  Proposition
  \ref{prop.mousereflection} gives the latter statement if we show the
  former, so fix an $x\in\hc$ and let $j:V\to M$ be the generic
  ultrapower from a precipitous ideal on $\omega_1^V$. Since $j(x)=x$
  we get that $j:L[x]\to L[x]$ with $\crit j>\rk x$, implying that
  $x^\sharp$ exists in the generic extension. But set forcing can't
  add sharps \todo{Add argument or reference.} so $x^\sharp$ exists in
  $V$ as well.
}

\defi{
  Let $j:V\to M$ be an elementary embedding in some $V[g]$ and let $F$ be a model operator. Then $F$ is \index{Operator!Radiant operator} \emph{$j$-radiant} if it condenses well, determines itself on generic extensions and satisfies the \index{Extension property}\index{Operator!Extension property of an operator} \emph{extension property}, which says that $F\subset j(F)$ and $j(F)\restr\hc^{V[g]}$ is definable in $V[g]$.
}

\lemm[$\di$][lemm.m1f]{
  $M_1^F$ is total on $H_{\omega_2}$ for any $j$-radiant model operator $F$ on $H_{\omega_2}$.
}
\proof{
  We want to use the hybrid core model dichotomy \ref{theo.kdichotomy}, but the problem is that $F$ is not total. We solve this by going to a smaller model; the model $W:=L^F_{\omega_2^V}(\mathbb R)$ will be a first attempt (note that $\mathbb R\in\dom F$ as we're assuming $\ch$). To be able to apply the dichotomy in a model we need it to satisfy $\zfc$. The following claim is the first step towards this.
 
  \clai{
    Given any real $x$, $L^F_{\omega_2}(x)\models\godel{\text{$\omega_1^V$ is inaccessible}}$.
  }

  \cproof{
    Letting $j:V\to M$ be the generic elementary embedding, note that $j$ doesn't move $x$, so that
    \eq{
      j\restr L^F_{\omega_2^V}(x):L^F_{\omega_2^V}(x)\to L^{j(F)}_{\omega_2^M}(x).
    }
    
    Since $F$ has the extension property, $L^{j(F)}_{\omega_2^M}(x)$ is just an end-extension of $L^F_{\omega_2^V}(x)$. In particular $\omega_1^V$ is still a cardinal in there, meaning that, for every $\alpha<\omega_1^V$,
    \eq{
      L_{\omega_1^M}^{j(F)}(x)\models\godel{\text{there's a cardinal $>\alpha$}}.
    }

    By elementarity this makes $\omega_1^V$ a limit cardinal in $L^F_{\omega_2^V}(x)$ and by $\gch$ in $L^F_{\omega_2^V}(x)$ it's inaccessible.
  }

  This claim is now transferred to $M$, and as $\mathbb R^V$ is a real from the point of view of $M$, we get that
  \eq{
    L^{j(F)}_{\omega_2^M}(\mathbb R^V)\models\godel{\text{$\omega_1^M$ is inaccessible}}.
  }

  Noting that $\omega_1^M=\omega_2^V$ and again using the extension property of $F$, we get that $W\models\zf$. We don't get choice in $W$ as it doesn't contain a wellorder of the reals, so we we'll work with $W[h]$ instead, where $h\subset\col(\omega_1,\mathbb R)^W$ is $W$-generic. Since we're assuming $\ch$ we get that $g\in V$, making $W[h]\in V$ as well, $W[h]$ is still closed under $F$ since $F$ determines itself on generic extensions, and $W[h]\models\zfc$.

  \qquad We can now apply the hybrid core model dichotomy \ref{theo.kdichotomy} inside $W[h]$ to conclude that, for every real $x$, either $K^F(x)^{W[h]}$ exists or $M_1^F(x)$ exists (note that $(\omega_1,\omega_1)$-iterability is absolute between $W[h]$ and $V$ since $W[h]$ contains all the reals). Since mouse reflection holds at $\omega_1$ by Proposition \ref{prop.mousereflection} if the latter conclusion held at all reals $x$ then we would also get that $M_1^F$ is total on $H_{\omega_2}$ and we'd be done. So assume $K:=K^F(x)^{W[h]}$ exists.

  \clai{
    $j(K)\in V$.
  }

  \cproof{
    This is where we'll be using homogeneity of our ideal. Firstly $K$ is definable in $W[h]$ and thus also in $W$ by homogeneity of $\col(\omega_1,\mathbb R)$, so that $j(K)$ is definable in $j(W)$. But $j(W)$ is definable in $V[g]$ as the unique $j(F)$-premouse over $\mathbb R$ of height $\omega_1$, making $j(K)$ definable in $V[g]$ with $j(F)\restr\hc$ as a parameter. But $j(F)\restr\hc$ is definable in $V[g]$ since $F$ satisfies the extension property, so homogeneity of our ideal implies that $j(F)\in V$ and hence $j(K)\in V$ as well.
}

  This claim also implies that $\omega_1^V$ is inaccessible in $K$, as if it wasn't, say $\omega_1^V=\lambda^{+K}$, then $\omega_2^V=j(\omega_1^V)=j(\lambda)^{+j(K)}=\lambda^{+j(K)}$, so that $\omega_2^V$ isn't a cardinal in $V$, $\contr$.

  \qquad We then also get that $(\omega_1^V)^{+j(K)}<\omega_2^V$, since if they were equal then elementarity would imply that $\omega_1^V$ was a successor in $K$, $\contr$.

  \qquad Since $K|\omega_1^V=j(K)|\omega_1^V$, elementarity and the above implies that
  \eq{
    j^2(K)|(\omega_1^V)^{+j^2(K)}=j(K)|(\omega_1^V)^{+j(K)},
  }

  which makes sense as $j(K)\in V$.

  \qquad Let now $E$ be the $(\omega_1^V,\omega_2^V)$-extender derived from $j\restr j(K)$, and note that $E\restr\alpha\in M$ for every $\alpha<\omega_2^V=\omega_1^M$ as $M$ is closed under countable sequences in $V[g]$.

  \clai{
    $E\restr\alpha$ is on the $j(K)$-sequence for every $\alpha<\omega_2^V$.
  }

  \cproof{
    We need to show \todo{Why is this sufficient?} that
    \eq{
      j(W)\models\godel{\text{$\bra{\bra{j(K),\ult(j(K),E\restr\alpha)}, \alpha}$ is $\on$-iterable}}.
    }

    Assume not. Then by reflection \todo{What kind of reflection?} we get, in $j(W)$, a countable $\overline K$ and an elementary $\sigma:\overline K\to\ult(j(K),E\restr\alpha)$ with $\sigma\restr\alpha=\id$ and $\bra{\bra{j(K),\overline K},\alpha}$ isn't $\omega_1$-iterable.

    \qquad Let $k:\ult(j(K),E\restr\alpha)\to j^2(K)$ be the factor map with $k\restr\alpha=\id$ and define $\psi:=k\circ\sigma:\overline K\to j^2(K)$, so that $(k\circ\sigma)\restr\alpha=\id$. We have both $\psi$ and $\overline K$ in $M$, which is the generic ultrapower $\ult(V,g)$, so we also get that $\psi=[\vec\psi_\xi]_g$, $\overline K=[\vec K_\xi]_g$ and $\alpha=[\vec\alpha_\xi]_g$. We need to show that
    \eq{
      \text{For $g$-almost every $\xi<\omega_1^V$ it holds that }W\models\godel{\text{$\bra{\bra{K,K_\xi},\alpha_\xi}$ is $\omega_1$-iterable}}
    }

    By \L o\' s' Lemma we have that, in $V$ and hence also in $V[g]$, there are embeddings $\psi_\xi:K_\xi\to j(K)$ with $\psi_\xi\restr\alpha_\xi=\id$ for $g$-almost every $\xi<\omega_1^V$. As $j(W)$ is closed under countable sequences in $V[g]$ it sees that the $K_\xi$'s are countable, so that an application of absoluteness of wellfoundedness \todo{Include this argument perhaps.} shows that $j(W)$ also has elementary embeddings $\psi^*_\xi:K_\xi\to j(K)$ with $\psi^*_\xi\restr\alpha_\xi$.

    \qquad But $j(K)=K^{j(F)}(x)^{j(W[h])}$, so $j(W[h])$ sees that $\bra{\bra{K,K_\xi},\alpha_\xi}$ is $\omega_1$-iterable, which is therefore also true in $W$ since $W\cap\mathbb R\subset\mathbb R^{V[g]}=j(W[h])\cap\mathbb R$.
  }

  Our desired contradiction is then showing that $K$ has a Shelah cardinal, which is impossible\todo{Insert argument?}. Let $f:\omega_1^V\to\omega_1^V$ be a function in $j(K)$ and pick some $\alpha\in(j(f)(\kappa),\omega_2^V)$. Letting
  \eq{
    k:\ult(j(K),E\restr\alpha)\to j^2(K)
  }

  be the factor map, we get that $\crit k\geq\alpha$ by coherence of extenders on the $K$-sequence and hence that $i_{E\restr\alpha}(f)(\omega_1^V)<\alpha$ as well. This shows that $\omega_1^V$ is Shelah in $j(K)$ and hence $K$ has a Shelah cardinal by elementarity, $\contr$.
}

\theo[$\di$][theo.internalinduction]{
  $\lp^{\Gamma,\Sigma}(\mathbb R)\models\ad$ for all ``nice'' $\Gamma$ and $\Sigma$.\todo{Specify niceness.}
}
\proof{
  \todo[inline]{Show that all the operators occuring in the $\lp^{\Gamma,\Sigma}(\mathbb R)$ induction are $j$-radiant.}
}

\end{document}
