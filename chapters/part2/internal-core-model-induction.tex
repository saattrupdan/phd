\documentclass[../../main]{subfiles}
\pagestyle{fancy}

\begin{document}

\chapter{Internal core model induction}
\thispagestyle{fancy}


\section{Operators}
\lipsum[1]


\section{Mouse witness equivalence}

\defi{
  asd\todo[inline]{Define coarse $(k,U,x)$-Woodin pairs}
}

\defi{
  \index{Coarse mouse witness condition}\index{W$^*_\alpha(F)$}
  Let $F$ be a total condensing operator and let $\alpha$ be an ordinal. Then the \textbf{coarse mouse witness condition at $\alpha$ with $F$}, written $W^*_\alpha(F)$, states that given any scaled-co-scaled $U\subset\mathbb R$ whose associated sequences of prewellorderings are elements of $\lp^F_\alpha(\mathbb R)$, we have for every $k<\omega$ and $x\in\mathbb R$ a coarse $(k,U,x)$-Woodin pair $(N,\Sigma)$ with $\Sigma\restr\hc\in\lp^F_\alpha(\mathbb R)$.\todo{Check if this is a reasonable definition.}
}

\theo[Hybrid witness equivalence][theo.witness]{
  \index{Hybrid witness equivalence}
  Let $\theta>0$ be a cardinal, $g\subset\col(\omega,{<}\theta)$ $V$-generic,  $\mathbb R^g:=\bigcup_{\alpha<\theta}\mathbb R^{V[g\restr\alpha]}$, $F$ a total radiant operator and $\alpha$ a critical ordinal of $\lp^F(\mathbb R^g)$. Assume that $\lp^F(\mathbb R^g)\models\dc+\godel{W^*_\beta(F)\text{ holds for all }\beta\leq\alpha}$. Then there is a hybrid mouse operator $\N\in V$ on $H_{\aleph_1^{V[g]}}$ such that
  \eq{
    \lp^F(\mathbb R^g)\models W^*_{\alpha+1}(F)\quad\text{iff}\quad V\models\godel{\text{$M_n^{\N}$ is total on $H_{\aleph_1^{V[g]}}$ for all $n<\omega$}}
  }

  Furthermore, if $\theta<\aleph_1^V$ then we only need to assume that $F$ is total and condensing.
}

\todo[inline]{Be more explicit about what the given operator $\N$ looks like.}


\section{Core models}
\lipsum[1]


\section{Core model dichotomy}

\lemm[][lemm.opit]{
  Let $\theta$ be a regular uncountable cardinal or $\theta=\infty$ and let $\N$ be a tame \todo{Define this} hybrid mouse operator on $H_\theta$ which relativises well. Then $\N$ is countably iterable iff it's $(\theta,\theta)$-iterable, guided by $\N$. Furthermore, for every $x\in H_\theta$, if $M_1^{\N}(x)$ exists and is countably iterable, then it's also $(\theta,\theta)$-iterable, guided by $\N$.\todo{Change this to model operators; perhaps change parts of the proof and/or assumptions needed.}
  }
\proof{
  Fix $x\in H_\theta$. We first show that $\N(x)$ is $(\theta,\theta)$-iterable. Let $\T\in H_\theta$ be a normal tree of limit length on $\N(x)$. Let $\eta\gg\rk(\T)$ and let
  \eq{
    \h:= \chull^{H_\eta}(\{x,\N(x),\T\})
  }
    
  with uncollapse $\pi\colon\h\to H_\eta$. Set $\overline a:=\pi^{-1}(a)$ for every $a\in\ran\pi$. Note that $\overline{\N(x)}=\N(\overline x)$ since $\N$ relativises well. Now $\overline\T$ is a normal, countable iteration tree on $\N(\overline x)$ and hence our iteration strategy yields a wellfounded cofinal branch $\overline b\in V$ for $\overline\T$. Note that $\overline\Q:=\Q(\overline b,\overline\T)$ exists, since if $\overline b$ drops then there's nothing to do, and otherwise we have that 
  \eq{ 
    \rho_{1}(\M^{\overline\T}_{\overline b})=\rho_{1}(\N(\overline x))=\rk\overline x<\delta(\overline\T),
  }

  so $\delta(\overline\T)$ is not definably Woodin over $\M^{\overline\T}_{\overline b}$.\todo{Why is that?}
  \clai{
    $\overline\Q\init\N(\M(\overline\T))$
  }

  \cproof{
    If $\overline\Q=\M(\overline\T)$ then the claim is trivial, so assume that $\M(\overline\T)\pinit\overline\Q$. Note that $\overline\Q\init M_{\overline b}^{\overline\T}$ by definition of $\Q$-structures, and that $M_{\overline b}^{\overline\T}$ satisfies $(2)$ of the definition of relativises well\todo{Define this, cutpoint and $\M_b^{\T}$}, meaning that
    \eq{
      M_{\overline b}^{\overline\T}\models\godel{\text{$\forall\eta\forall\zeta>\eta:$ if $\eta$ is a cutpoint then $M_{\overline b}^{\overline\T}|\zeta\not\models\varphi_{\N}[\bar x,p_{\N}]$}}.\tag*{(1)} 
    }

    This statement is $\Pi^1_2$ and $\overline\Q$ is $\Pi^1_2$-correct since it contains a Woodin cardinal, so that $\Q$ satisfies the statement as well. Since $\N$ is tame we get that $\delta(\overline\T)$ is a cutpoint of $\overline\Q$, so that $\N(\M(\overline\T))=\N(\overline\Q|\delta(\overline\T))$ is \textit{not} a proper initial segment of $\overline\Q$. Further, as we're assuming that both $\N(\M(\overline\T))$ and $\M^{\overline\T}_{\overline b}$ are $(\omega_{1}{+}1)$-iterable above $\delta(\overline\T)$ the same thing holds for $\overline\Q\init\M_{\overline b}^{\overline\T}$, so that we can compare $\N(\M(\overline\T))$ with $\overline\Q$ (in $V$). Let
    \eq{ 
      (\N(\M(\overline\T)),\overline\Q) \leadsto (\P,\R) 
    }

    be the result of the coiteration. We claim that $\R\init\P$. Suppose $\P\pinit\R$. Then there is no drop in $\N(\M(\overline\T))\leadsto\P$ and in fact $\N(\M(\overline\T))=\P$ since $\N(\M(\overline\T))$ projects to $\delta(\overline\T)$. Furthermore, as we established that $\N(\M(\overline\T))=\N(\overline\Q|\delta(\overline\T))$ isn't a proper initial segment of $\overline\Q$ it can't be a proper initial segment of $\R$ either, as the coiteration is above $\delta(\overline\T)$. But we're assuming that $\N(\M(\overline\T))=\P\pinit\R$, a contradiction. So $\R\init\P$.
        
    \qquad Since $\N(\M(\overline\T))$ and $\overline\Q$ agree up to $\delta(\overline\T)$ and there is no drop $\overline\Q\leadsto\R$ we have that $\overline\Q=\R$. If $\N(\M(\overline\T))\leadsto\P$ doesn't move either we're done, so assume not. Let $F$ be the first exit extender \todo{Define this} of $\N(\M(\overline\T))$ in the coiteration. We have $\lh(F) \le o(\overline\Q)$, $\overline\Q\init\P$ and $\lh(F)$ is a cardinal in $\P$.
        
    \qquad As $\overline\Q$ is $\delta(\overline\T)$-sound and projects to $\delta(\overline\T)$ it follows that $J(\overline\Q|\lh(F))$ collapses $\lh(F)$, so it has to be the case that $\overline\Q|\lh(F)=\P$ and thus $o(\P)=\lh(F)$. But this means that $\P=\N(\M(\overline\T))$ even though we assumed that $\N(\M(\T))\leadsto\P$ moved, a contradiction.
  }

  Now, in a sufficiently large collapsing extension extension of $\h$, $\overline b$ is the unique cofinal, wellfounded branch of $\overline\T$ such that $\Q(\overline b,\overline\T) \init\N(\M(\overline\T))$ exists. Hence, by the homogeneity of $\col(\omega,\theta)$, $\overline b \in H$. By elementarity there is a unique cofinal, wellfounded branch $b$ of $\T$ such that $\Q(b,\T)\init\N(\M(\T))$. This proves that $M$ is (uniquely) $\on$-iterable and virtually the same argument yields the iterability of $M$ via successor-many stacks of normal trees.
  
  \qquad To show that $M$ is fully iterable, it remains to be seen that the unique iteration strategy (guided by $\N$) of $M$ outlined above leads to wellfounded direct limits for stacks of normal trees on $M$ of limit length. Let $\lambda$ be a limit ordinal and $\vec\T = (\T_i \mid i<\lambda)$ a stack according to our iteration strategy. Suppose $\lim_{i<\lambda}\M^{\T_i}_\infty$ is illfounded.
  
  \qquad Redefine $\eta\gg\rk(\vec\T)$, $\h:=\chull^{H_\eta}(\{x,M,\vec\T\})$ and $\pi:\h\to H_\eta$ the uncollapse, again with $\overline a:=\pi^{-1}(a)$ for every $a\in\ran\pi$. By elementarity we get that $\h\models\godel{\lim_{i<\overline\lambda}\M^{\overline\T_i}_\infty\text{ is illfounded}}$. But $\overline{\vec\T}$ is countable and according to the iteration strategy guided by $\N$, so that
  \eq{
  V\models\godel{\lim_{i<\overline\lambda}\M^{\overline\T_i}_\infty\text{ is wellfounded.}}
  }

  Now note that $(\lim_{i<\overline\lambda}\M^{\overline\T_i}_\infty)^{\h}=(\lim_{i<\overline\lambda}\M^{\overline\T_i}_\infty)^V$ and wellfoundedness is absolute between $\h$ and $V$, a contradiction.
    
  \qquad Now assume that $M_1^{\N}(x)$ exists for some $x\in H_\theta$, and that it's countably iterable. We then do exactly the same thing as with $\N(x)$ \textit{except} that in the claim we replace $(1)$ with
  \eq{
    \overline\Q\models\forall\eta(\overline\Q|\eta\not\models\godel{\text{$\delta(\overline\T)$ is not Woodin}}),
  }

  so that if $\P\pinit\R$ then $\delta(\overline\T)$ is still Woodin in $\P=\N(\M(\overline\T))$, contradicting the defining property of $M_1^{\N}(x)$ (and thus also of $\R$). The rest of the proof is a copy of the above.
}

\theo[Hybrid core model dichotomy][theo.kdichotomy]{
  \index{Hybrid core model dichotomy}
  Let $\theta$ be a $\beth$-fixed point or $\theta=\infty$, and let $F$ be a tame \todo{I don't think tame is needed here, as we're only indexing extenders at $F$-initial segments} model operator on $H_\theta$ that condenses well. Let $x\in H_\theta$. Then either:
  \begin{enumerate}
    \item The core model $K^F(x)|\theta$ exists and is $(\theta,\theta)$-iterable; or
    \item $M_1^F(x)$ exists and is $(\theta,\theta)$-iterable.
  \end{enumerate}
}
\proof{
  Assume first that $K^{c,F}(x)|\theta$ reaches a premouse which isn't $F$-small; let $\N_\xi$ be the first part of the construction witnessing this. Then $\core(\N_\xi)=M_1^F(x)$\todo{Insert argument?}, and by Lemma \ref{lemm.opit} it suffices to show that $M_1^F(x)$ is countably iterable.\todo[inline]{Show that $M_1^F(x)$ is countably iterable.}

  \qquad We can thus assume that $K^{c,F}(x)|\theta$ is $F$-small. Note that if $K^{c,F}(x)|\theta$ has a Woodin cardinal then because the model is $F$-closed we contradict $F$-smallness, so the model has no Woodin cardinals either, making it $(\theta,\theta)$-iterable.
    
  \qquad Let $\kappa<\theta$ be any uncountable cardinal and let $\Omega:=\beth_\kappa(\kappa)^+$. Note that $\Omega<\theta$ since we assumed that $\theta$ is a $\beth$-fixed point and $\kappa<\theta$. If $\Omega$ is a limit cardinal in $K^{c,F}(x)|\theta$ then let $\S:=\lp(K^{c,F}(x)|\Omega)$ and otherwise let $\S:=K^{c,F}(x)|\Omega$. Then by Lemma 3.3 of
  \cite{mousestack} we get that $\S$ is countably iterable, with largest cardinal $\Omega$ in the ``limit cardinal case''.
    
  \qquad This also means that $\Omega$ isn't Woodin in $L[\S]$, as it's trivial in the case where $\Omega$ is a successor cardinal of $K^{c,F}(x)|\theta$ by our case assumption, and in the ``limit cardinal case'' it also holds since 
  \eq{
    K^{c,F}(x)|\Omega^{+K^{c,F}(x)|\theta}\subseteq\S.  
  }

  By \cite{fernandes} and \cite{JensenSteel} this means that we can build $K^{F}(x)|\kappa$, as the only places they use that there's no inner model with a Woodin are to guarantee that $K^{c,F}(x)|\Omega$ exists and has no Woodin cardinals, and in Lemma 4.27 of \cite{JensenSteel} in which they only require that $\Omega$ isn't Woodin in $L[\S]$.
    
  \qquad As $\kappa<\theta$ was arbitrary we then get that $K^{F}(x)|\theta$ exists. Note that $K^{F}(x)|\theta$ has no Woodin cardinals either and is $F$-small, so that $\Q$-structures trivially exist, making it $(\theta,\theta)$-iterable.
}


\end{document}
