\documentclass[../../main]{subfiles}
\pagestyle{fancy}

\begin{document}

\chapter{Part I Introduction}
\thispagestyle{fancy}

This first part of the thesis is dedicated to the smaller realms of the large cardinal hierarchy. This section of the hierarchy, ranging all the way from inaccessible cardinals up to the measurable cardinals, has not received the same uniform treatment as the large cardinals above the measurables. 

\qquad As a prime example of this we can consider the weakly compact cardinals which has no less than 11 distinct equivalent characterisations, from graph colouring properties and existence of branches through trees, to syntactic indescribability, compactness properties of infinitary languages and elementary embeddings between set-sized structures. This is definitely a positive feature of these cardinals, but the disconnect between the way we are working with them and the way we are working with the rest of them makes generalisation of methods in one realm to the other difficult.

\qquad The \textit{virtual large cardinals}, introduced in \cite{remarkable} and further analysed in \cite{GitmanSchindler}, can be viewed as a way to remedy this situation. These cardinals have definitions similar to the larger cardinals above the measurables, but with much weaker consistency strengths. This means that we can essentially get a ``copy'' of the upper half of the large hierarchy much further down\footnote{We don't get an exact copy, as the virtual hierarchy behaves in different ways, see Chapter \ref{chapter.virtual-large-cardinals}.}, and as these virtual cardinals are also defined in terms of elementary embeddings, like their stronger counterparts, methods using such embeddings can now be employed at the lower section of the large cardinal hierarchy.

\qquad In Chapter \ref{chapter.virtual-large-cardinals} we will introduce the virtual large cardinals, prove many results regarding their internal properties and how they relate to each other. This chapter contains results from collaborations with Victoria Gitman, Stamatis Dimopoulos and Philipp Schlicht.

\qquad The following Chapter \ref{chapter.set-theoretic-connections} then steps outside the virtual world and investigates how these virtual large cardinals relate to well-established set-theoretic concepts, such as infinite games, embeddings between set-sized structures and precipitous ideals. This latter chapter is based on work with Philip Welch and Ralf Schindler, with a large part of it coming from the paper \cite{NielsenWelch}.


\end{document}
