\documentclass[../main]{subfiles}
\pagestyle{fancy}

\begin{document}

\chapter{Introduction}
\thispagestyle{fancy}

\section{Getting started}

A key folklore lemma which we will frequently need when dealing with elementary embeddings existing in generic extensions is the following.

\lemm[Countable Embedding Absoluteness][lemm.ctblabs]{
  Let $\M,\N$ be sets, $\P$ a transitive class with $\M,\N\in\P$, and let $\pi\colon\M\to\N$ be an elementary embedding. Assume that
  \eq{
    \P\models\zf^-+\dc+\godel{\text{$\M$ is countable}}
  }  
  
  and fix any finite $X\subset\M$. Then $\P$ contains an elementary embedding $\pi^*\colon\M\to\N$ which agrees with $\pi$ on $X$. If $\pi$ has a critical point and if $\M$ and $\N$ are both transitive then we can also assume that $\crit\pi=\crit\pi^*$.\footnote{We are using transitivity of $\M$ and $\N$ to ensure that the \textit{ordinal} $\crit\pi$ exists.}
}
\proof{
  Let $\{a_i\mid i<\omega\}\in\P$ be an enumeration of $\M$ and set $\M\restr n:=\{a_i\mid i<n\}$. Then, in $\P$, build the tree $\T$ of all partial isomorphisms between $\M\restr n$ and $\N$ for $n<\omega$, ordered by extension. Then $\T$ is illfounded in $V$ by assumption, so it's also illfounded in $\P$ since $\P$ is transitive and $\P\models\zf^-+\dc$. The branch then gives us the embedding $\pi^*$, and if $\crit\pi$ exists then we can ensure that it agrees with $\pi$ on the critical point and finitely many values by adding these conditions to $\T$.
}

\section{Setting the scene 2}
\label{sect.settingthescene}

In this section we will recall a handful of definitions concerning Ramsey-like cardinals, as well as define the $\alpha$-Ramsey cardinals for arbitrary ordinals $\alpha$. We start out with the models and measures that we are going to consider.

\defi{
	For a cardinal $\kappa$, a \textbf{weak $\kappa$-model} is a set $\M$ of size $\kappa$ satisfying that $\kappa+1\subset\M$ and $(\M,\in)\models\zfc^-$. If furthermore $\M^{<\kappa}\subset \M$, $\M$ is a \textbf{$\kappa$-model}.\footnote{Note that our (weak) $\kappa$-models do not have to be transitive, in contrast to the models considered in \cite{Ramsey1} and \cite{Ramsey2}. Not requiring the models to be transitive was introduced in \cite{HolySchlicht}.}\todo{Define this at some earlier point?}
}

Recall that $\mu$ is an \textbf{$\M$-measure} if $(\M,\in,\mu)\models\godel{\mu\text{ is a $\kappa$-complete ultrafilter on $\kappa$}}$.

\defi{
	Let $\M$ be a weak $\kappa$-model and $\mu$ an $\M$-measure. Then $\mu$ is
\begin{itemize}
	\item \textbf{weakly amenable} if $x\cap\mu\in\M$ for every $x\in\M$ with $\M$-cardinality $\kappa$;
	\item \textbf{countably complete} if $\bigcap\vec X\neq\emptyset$ for every $\omega$-sequence $\vec X\in{^\omega\mu}$;
	\item \textbf{$\M$-normal} if $(\M,\in,\mu)\models\forall\vec X\in{^\kappa}\mu:\triangle\vec X\in\mu$;
	\item \textbf{genuine} if $|\triangle\vec X|=\kappa$ for every $\kappa$-sequence $\vec X\in{^\kappa\mu}$;
	\item \textbf{normal} if $\triangle\vec X$ is stationary in $\kappa$ for every $\kappa$-sequence $\vec X\in{^\kappa\mu}$;
	\item \textbf{$0$-good}, or simply \textbf{good}, if it has a well-founded ultrapower;
	\item \textbf{$\alpha$-good} for $\alpha>0$ if it is weakly amenable and has $\alpha$-many well-founded iterates.
\end{itemize}
}

Note that a genuine $\M$-measure is $\M$-normal and countably complete, and a countably complete weakly amenable $\M$-measure is $\alpha$-good for all ordinals $\alpha$. We'll use the fact shown in \cite{HolySchlicht} that an $\M$-measure $\mu$ is normal iff $\triangle\vec X$ is stationary for some enumeration $\vec X=\bra{X_\alpha\mid\alpha<\kappa}$ of $\mu$. We are also going to use the following alternative characterisation of weak amenability.

\qprop[Folklore]{
	Let $\M$ be a weak $\kappa$-model, $\mu$ an $\M$-measure and $j:\M\to\N$ the associated ultrapower embedding. Then $\mu$ is weakly amenable if and only if $j$ is \textbf{$\kappa$-powerset preserving}, meaning that $\M\cap\p(\kappa)=\N\cap\p(\kappa)$.
}

The $\alpha$-Ramsey cardinals in \cite{HolySchlicht} are based upon the following game.\footnote{Unless otherwise stated, every game considered will be a game with perfect information between two players I and II. For a formal framework modelling these games, see e.g. \cite{Kanamori}.}

\defi[Holy-Schlicht]{
	For an uncountable cardinal $\kappa=\kappa^{<\kappa}$, a limit ordinal $\gamma\leq\kappa$ and a regular cardinal $\theta>\kappa$ define the game $wfG_\gamma^\theta(\kappa)$ of length $\gamma$ as follows.
	\game{\M_0}{\mu_0}{\M_1}{\mu_1}{\M_2}{\mu_2}{\cdots}{\cdots}

 Here $\M_\alpha\prec H_\theta$ is a $\kappa$-model and $\mu_\alpha$ is a filter for all $\alpha<\gamma$, such that $\mu_\alpha$ is an $\M_\alpha$-measure, the $\M_\alpha$'s and $\mu_\alpha$'s are $\subset$-increasing and $\bra{\M_\xi\mid\xi<\alpha},\bra{\mu_\xi\mid\xi<\alpha}\in\M_\alpha$ for every $\alpha<\gamma$. Letting $\mu:=\bigcup_{\alpha<\gamma}\mu_\alpha$ and $\M:=\bigcup_{\alpha<\gamma}\M_\alpha$, player II wins iff $\mu$ is an $\M$-normal good $\M$-measure.
}

Recall that two games $G_1$ and $G_2$ are \textbf{equivalent} if player I has a winning strategy in $G_1$ iff they have one in $G_2$, and player II has a winning strategy in $G_1$ iff they have one in $G_2$. \cite{HolySchlicht} showed that the games $wfG_\gamma^{\theta_0}(\kappa)$ and $wfG_\gamma^{\theta_1}(\kappa)$ are equivalent for any $\gamma$ with $\cof\gamma\neq\omega$ and any regular $\theta_0,\theta_1>\kappa$. We will be working with a variant of the $wfG_\gamma(\kappa)$ games in which we require less of player I but more of player II. It will turn out that this change of game is innocuous, as Proposition \ref{prop.tildegame} will show that they are equivalent.

\defi[Holy-N.-Schlicht]{
	Let $\kappa=\kappa^{<\kappa}$ be an uncountable cardinal, $\gamma\leq\kappa$ and $\zeta$ ordinals and $\theta>\kappa$ a regular cardinal. Then define the following game $\G_\gamma^\theta(\kappa,\zeta)$ with $(\gamma{+}1)$-many rounds:
	\game{\M_0}{\mu_0}{\M_1}{\mu_1}{\cdots}{\cdots}{\M_\gamma}{\mu_\gamma}

	Here $\M_\alpha\prec H_\theta$ is a weak $\kappa$-model for every $\alpha\leq\gamma$, $\mu_\alpha$ is a normal $\M_\alpha$-measure for $\alpha<\gamma$, $\mu_\gamma$ is an $\M_\gamma$-normal good $\M_\gamma$-measure and the $\M_\alpha$'s and $\mu_\alpha$'s are $\subset$-increasing. For limit ordinals $\alpha\leq\gamma$ we furthermore require that $\M_\alpha=\bigcup_{\xi<\alpha}\M_\xi$, $\mu_\alpha=\bigcup_{\xi<\alpha}\mu_\xi$ and that $\mu_\alpha$ is $\zeta$-good. Player II wins iff they could continue to play throughout all $(\gamma{+}1)$-many rounds.
}

For convenience we will write $\G_\gamma^\theta(\kappa)$ for the game $\G_\gamma^\theta(\kappa,0)$, and $\G_\gamma(\kappa)$ for $\G_\gamma^\theta(\kappa)$ whenever $\cof\gamma\neq\omega$, as again the existence of winning strategies in these games doesn't depend upon a specific $\theta$. Note that we assume that $\kappa=\kappa^{<\kappa}$ is uncountable in the definition of the games that we're considering, so this is a standing assumption throughout the paper, whenever any one of the above two games are considered.

\defi{
	Define the \textbf{Cohen game} $\C_\gamma^\theta(\kappa)$ as $\G_\gamma^\theta(\kappa)$ but where we require that $\abs{\M_\alpha-H_\kappa}<\gamma$ for every $\alpha<\gamma$, i.e. that we only allow player I to add ${<}\gamma$ new elements to the models in each round, and where we only require $\M_\alpha\models\zfc^-$ and $\M_\alpha\prec H_\theta$ for $\alpha\leq\gamma$ limit.\footnote{$\C_\omega^\theta(\kappa)$ is similar to the $H(F,\lambda)$-games in \cite{DonderLevinski}.}

	\qquad Also define the \textbf{weak Cohen game} $\C_\gamma^-(\kappa)$ in analogy with $\G_\gamma^-(\kappa)$.
}

\prop[N.]{
	Assume $\gamma^{\aleph_0}=\gamma$ and let $\kappa$ be regular. Then $\C_\gamma^-(\kappa)$ is equivalent to $\C_\gamma^\theta(\kappa)$ for all regular $\theta>\kappa$. In particular, if $\ch$ holds then $\C_{\omega_1}^-(\kappa)$ is equivalent to $\C_{\omega_1}^\theta(\kappa)$ for all regular $\theta>\kappa$.
}
\proof{
	The assumption that $\gamma^{\aleph_0}=\gamma$ allows us to ensure that ${^\omega}\M_\alpha\subset\M_\gamma$ for all $\alpha<\gamma$.	If player I has a winning strategy in $\C_\gamma^\theta(\kappa)$ for some regular $\theta>\kappa$ then they still win if we require that ${^\omega}\M_\alpha\subset\M_\gamma$ (since they're only enlargening their models, making it even harder for player II to win), in which case the final measure $\mu_\gamma$ is countably complete and hence automatically has a wellfounded ultrapower.
	
	\qquad If player II has a winning strategy in $\C_\gamma^-(\kappa)$ then they still win if player I plays $\M_\alpha$ such that ${^\omega}\M_\alpha\subset\M_\gamma$, again ensuring that $\mu_\gamma$ has a wellfounded ultrapower.
}

\prop[Holy-N.-Schlicht]{
\label{prop.tildegame}
$\G_\gamma^\theta(\kappa)$, $\G_\gamma^\theta(\kappa,1)$ and $wfG_\gamma^\theta(\kappa)$ are all equivalent for all limit ordinals $\gamma\leq\kappa$, and $\G_\gamma^\theta(\kappa,\zeta)$ is equivalent to $\G_\gamma^\theta(\kappa)$ whenever $\cof\gamma>\omega$ and $\zeta\in\on$.
}
\proof{
	We start by showing the latter statement, so assume that $\cof\gamma>\omega$. Consider now the auxilliary game, call it $\mathcal G$, which is exactly like $\G_\gamma^\theta(\kappa,0)$, but where we also require that $^\omega{\M_\alpha}\subset\M_{\alpha+1}$ and $\bra{\M_\xi\mid\xi\leq\alpha},\bra{\mu_\xi\mid\xi\leq\alpha}\in\M_{\alpha+1}$ for every $\alpha<\gamma$.

	\clai{
		$\mathcal G$ is equivalent to $\G_\gamma^\theta(\kappa)$.
	}

	\cproof{
		If player I has a winning strategy in $\mathcal G$ then they also have one in $\G_\gamma^\theta(\kappa)$, by doing exactly the same. Analogously, if player II has a winning strategy in $\G_\gamma^\theta(\kappa)$ then they also have one in $\mathcal G$. If player I has a winning strategy $\sigma$ in $\G_\gamma^\theta(\kappa)$ then we can construct a winning strategy $\sigma'$ in $\mathcal G$, which is defined as follows. Fix some $\alpha\leq\gamma$ and, writing $\vec\M_\xi:=\bra{\M_\xi\mid\xi\leq\alpha}$ and $\vec\mu_\xi:=\bra{\mu_\xi\mid\xi\leq\alpha}$, we set
		\eq{
			\sigma'(\bra{\M_\xi,\mu_\xi\mid\xi\leq\alpha}):=\hull^{H_\theta}(\sigma(\bra{\M_\xi,\mu_\xi\mid\xi\leq\alpha})\cup{^\omega{\M_\alpha}}\cup\{\vec\M_\xi,\vec\mu_\xi\}),
		}

		i.e. that we're simply throwing in the sequences into our models and making sure that we're still an elementary substructure of $H_\theta$. This new strategy $\sigma'$ is clearly winning. Assuming now that $\tau$ is a winning strategy for player II in $\mathcal G$, we define a winning strategy $\tau'$ for player II in $\G_\gamma^\theta(\kappa)$ by letting $\tau'(\bra{\M_\xi,\mu_\xi\mid\xi\leq\alpha})$ be the result of throwing in the appropriate sequences into the models $\M_\xi$, applying $\tau$ to get a measure, and intersecting that measure with $\M_\alpha$ to get an $\M_\alpha$-measure.
	}

	Now, letting $\M_\gamma$ be the final model of a play of $\mathcal G$, $\cof\gamma>\omega$ implies that any $\omega$-sequence $\vec X\in\M_\gamma$ really is a sequence of elements from some $\M_\xi$ for $\xi<\gamma$, so that $\vec X\in\M_{\xi+1}$ by definition of $\mathcal G$, making $\M_\gamma$ closed under $\omega$-sequences and thus also $\mu_\gamma$ countably complete. Since $\gamma$ is a limit ordinal and the models contain the previous measures and models as elements, the proof of e.g. Theorem 5.6 in \cite{HolySchlicht} shows that $\mu_\gamma$ is also weakly amenable, making it $\zeta$-good for all ordinals $\zeta$.

	\qquad Now we deal with the first statement, so fix a limit ordinal $\gamma$. Firstly $\G_\gamma^\theta(\kappa)$ is equivalent to $\G_\gamma^\theta(\kappa,1)$ as above, since both are equivalent to the auxilliary game $\G$ when $\gamma$ is a limit ordinal. So it remains to show that $\G_\gamma^\theta(\kappa)$ is equivalent to $wfG_\gamma^\theta(\kappa)$. If player I has a winning strategy $\sigma$ in $wfG_\gamma^\theta(\kappa)$ then define a winning strategy $\sigma'$ for player I in $\G_\gamma^\theta(\kappa)$ as
	\eq{
		\sigma'(\bra{\M_\xi,\mu_\xi\mid\xi\leq\alpha}):=\sigma(\bra{\M_0,\mu_0}^\smallfrown\bra{\M_{\xi+1},\mu_{\xi+1}\mid\xi+1\leq\alpha})
	}

	and for limit ordinals $\alpha\leq\gamma$ set $\sigma'(\bra{\M_\xi,\mu_\xi\mid\xi<\alpha}):=\bigcup_{\xi<\alpha}\M_\xi$; i.e. they simply follow the same strategy as in $wfG_\gamma^\theta(\kappa)$ but plugs in unions at limit stages. Likewise, if player II had a winning strategy in $\G_\gamma^\theta(\kappa)$ then they also have a winning strategy in $wfG_\gamma^\theta(\kappa)$, this time just by skipping the limit steps in $\G_\gamma^\theta(\kappa)$.

	\qquad Now assume that player I has a winning strategy $\sigma$ in $\G_\gamma^\theta(\kappa)$ and that player I \textit{doesn't} have a winning strategy in $wfG_\gamma^\theta(\kappa)$. Then define a strategy $\sigma'$ for player I in $wfG_\gamma^\theta(\kappa)$ as follows. Let $s=\bra{\M_\alpha,\mu_\alpha\mid\alpha\leq\eta}$ be a partial play of $wfG_\gamma^\theta(\kappa)$ and let $s'$ be the modified version of $s$ in which we have 'inserted' unions at limit steps, just as in the above paragraph. We can assume that every $\mu_\alpha$ in $s'$ is good and $\M_\alpha$-normal as otherwise player II has already lost and player I can play anything. Now, we want to show that $s'$ is a valid partial play of $\G_\gamma^\theta(\kappa)$. All the models in $s$ are $\kappa$-models, so in particular weak $\kappa$-models.
	
	\clai{
		Every $\mu_\alpha$ in $s'$ is normal.
	}	
	
	\cproof{
		Assume without loss of generality that $\alpha=\eta$. Let player I play any legal response $\M$ to $s$ in $wfG_\gamma^\theta(\kappa)$ (such a response always exists). If player II can't respond then player I has a winning strategy by simply following $s^\cap\bra{\M}$, $\contr$, so player II \textit{does} have a response $\mu$ to $s^\cap\M$. But now the rules of $wfG_\gamma^\theta(\kappa)$ ensures that $\mu_\eta\in\M$, so since
		\eq{
		(\M,\in,\mu)\models\forall\vec X\in{^\kappa}\mu:\godel{\triangle\vec X\text{ is stationary in }\kappa},
		}

		we then also get that $\M\models\godel{\triangle\mu_\eta\text{ is stationary in $\kappa$}}$ since $\mu_\eta\subset\mu$, so elementarity of $\M$ in $H_\theta$ implies that $\triangle\mu_\eta$ really \textit{is} stationary in $\kappa$, making $\mu_\eta$ normal.
	}
	
	This makes $s'$ a valid partial play of $\G_\gamma^\theta(\kappa)$, so we may form the weak $\kappa$-model $\tilde\M_\eta:=\sigma(s')$. Now let $\M_\eta\prec H_\theta$ be a $\kappa$-model with $\tilde\M_\eta\subset\M_\eta$ and $s\in\M_\eta$ and set $\sigma'(s):=\M_\eta$. This defines the strategy $\sigma'$ for player I in $wfG_\gamma^\theta(\kappa)$, which is winning since the winning condition for the two games is the same for $\gamma$ a limit.\footnote{More precisely, that $\sigma$ is winning in $\G_\gamma^\theta(\kappa)$ means that there's a sequence $\bra{f_n:\kappa\to\kappa\mid n<\omega}$ with the $f_n$'s all being elements of the last model $\tilde\M_\gamma$, witnessing the illfoundedness of the ultrapower. But then all these functions will also be elements of the union of the $\M_\alpha$'s, since we ensured that $\M_\alpha\supset\tilde\M_\alpha$ in the construction above, making the ultrapower of $\bigcup_{\alpha<\gamma}\M_\alpha$ by $\bigcup_{\alpha<\gamma}\mu_\alpha$ illfounded as well.}

	\qquad Next, assume that player II has a winning strategy $\tau$ in $wfG_\gamma^\theta(\kappa)$. We recursively define a strategy $\tilde\tau$ for player II in $\G_\gamma^\theta(\kappa)$ as follows. If $\tilde\M_0$ is the first move by player I in $\G_\gamma^\theta(\kappa)$, let $\M_0\prec H_\theta$ be a $\kappa$-model with $\tilde\M_0\subset\M_0$, making $\M_0$ a valid move for player I in $wfG_\gamma^\theta(\kappa)$. Write $\mu_0:=\tau(\bra{\M_0})$ and then set $\tilde\tau(\bra{\tilde\M_0})$ to be $\tilde\mu_0:=\mu_0\cap\tilde\M_0$, which again is normal by the same trick as above, making $\tilde\mu_0$ a legal move for player II in $\G_\gamma^\theta(\kappa)$. Successor stages $\alpha+1$ in the construction are analogous, but we also make sure that $\bra{\M_\xi\mid\xi<\alpha+1},\bra{\mu_\xi\mid\xi<\alpha+1}\in\M_{\alpha+1}$. At limit stages $\tau$ outputs unions, as is required by the rules of $\G_\gamma^\theta(\kappa)$. Since the union of all the $\mu_\alpha$'s is good as $\tau$ is winning, $\tilde\mu_\gamma:=\bigcup_{\alpha<\gamma}\tilde\mu_\alpha$ is good as well, making $\tilde\tau$ winning and we are done.
}

We now arrive at the definitions of the cardinals we will be considering. They were in \cite{HolySchlicht} only defined for $\gamma$ being a cardinal, but given the above result we generalise it to all ordinals $\gamma$.

\defi{
	Let $\kappa$ be a cardinal and $\gamma\leq\kappa$ an ordinal. Then $\kappa$ is \textbf{$\gamma$-Ramsey} if player I does not have a winning strategy in $\G_\gamma^\theta(\kappa)$ for all regular $\theta>\kappa$. We furthermore say that $\kappa$ is \textbf{strategic $\gamma$-Ramsey} if player II \textit{does} have a winning strategy in $\G_\gamma^\theta(\kappa)$ for all regular $\theta>\kappa$. Define \textbf{(strategic) genuine $\gamma$-Ramseys} and \textbf{(strategic) normal $\gamma$-Ramseys} analogously, but where we require the last measure $\mu_\gamma$ to be genuine and normal, respectively.
}

\defi[N.]{
	\label{defi.cohramsey}
	A cardinal $\kappa$ is \textbf{${<}\gamma$-Ramsey} if it is $\alpha$-Ramsey for every $\alpha<\gamma$, \textbf{almost fully Ramsey} if it is ${<}\kappa$-Ramsey and \textbf{fully Ramsey} if it is $\kappa$-Ramsey. Further, say that $\kappa$ is \textbf{coherent ${<}\gamma$-Ramsey} if it's strategic $\alpha$-Ramsey for every $\alpha<\gamma$ and that there exists a choice of winning strategies $\tau_\alpha$ in $\G_\alpha(\kappa)$ for player II satisfying that $\tau_\alpha\subset\tau_\beta$ whenever $\alpha<\beta$. In other words, there is a single strategy $\tau$ for player II in $\G_\gamma(\kappa)$ such that $\tau$ is a winning strategy for player II in $\G_\alpha(\kappa)$ for every $\alpha<\gamma$.\footnote{Note that, with this terminology, ``coherent'' is a stronger notion than ``strategic''. We could've called the cardinals \textit{coherent strategic ${<}\gamma$-Ramseys}, but we opted for brevity instead.}
}

This is not the original definition of (strategic) $\gamma$-Ramsey cardinals however, as this involved elementary embeddings between weak $\kappa$-models -- but as the following theorem of \cite{HolySchlicht} shows, the two definitions coincide whenever $\gamma$ is a regular cardinal.

\qtheo[Holy-Schlicht]{
	\label{theo.Ramseydef}
	For regular cardinals $\lambda$, a cardinal $\kappa$ is $\lambda$-Ramsey iff for arbitrarily large $\theta>\kappa$ and every $A\subset\kappa$ there is a weak $\kappa$-model $\M\prec H_\theta$ with $\M^{<\lambda}\subset\M$ and $A\in\M$ with an $\M$-normal 1-good $\M$-measure $\mu$ on $\kappa$.
}



\end{document}
