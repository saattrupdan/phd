\documentclass[../main]{subfiles}
\pagestyle{fancy}

\begin{document}

\chapter{Introduction}
\thispagestyle{fancy}


\section{Filters on small structures}

\defi{
	For a cardinal $\kappa$, a \textbf{weak $\kappa$-model} is a set $\M$ of size $\kappa$ satisfying that $\kappa+1\subset\M$ and $(\M,\in)\models\zfc^-$. If furthermore $\M^{<\kappa}\subset \M$, $\M$ is a \textbf{$\kappa$-model}.\footnote{Note that our (weak) $\kappa$-models do not have to be transitive, in contrast to the models considered in \cite{Ramsey1} and \cite{Ramsey2}. Not requiring the models to be transitive was introduced in \cite{HolySchlicht}.}
}

Recall that $\mu$ is an \textbf{$\M$-measure} if $(\M,\in,\mu)\models\godel{\mu\text{ is a $\kappa$-complete ultrafilter on $\kappa$}}$.

\defi{
	Let $\M$ be a weak $\kappa$-model and $\mu$ an $\M$-measure. Then $\mu$ is
\begin{itemize}
	\item \textbf{weakly amenable} if $x\cap\mu\in\M$ for every $x\in\M$ with $\M$-cardinality $\kappa$;
	\item \textbf{countably complete} if $\bigcap\vec X\neq\emptyset$ for every $\omega$-sequence $\vec X\in{^\omega\mu}$.
\end{itemize}
}

\qprop[Folklore]{
	Let $\M$ be a weak $\kappa$-model, $\mu$ an $\M$-measure and $j:\M\to\N$ the associated ultrapower embedding. Then $\mu$ is weakly amenable if and only if $j$ is \textbf{$\kappa$-powerset preserving}, meaning that $\M\cap\p(\kappa)=\N\cap\p(\kappa)$.
}


\section{Embeddings between small structures}

A key folklore lemma which we will frequently need when dealing with elementary embeddings existing in generic extensions is the following.

\lemm[Countable Embedding Absoluteness][lemm.ctblabs]{
  Let $\M,\N$ be sets, $\P$ a transitive class with $\M,\N\in\P$, and let $\pi\colon\M\to\N$ be an elementary embedding. Assume that
  \eq{
    \P\models\zf^-+\dc+\godel{\text{$\M$ is countable}}
  }  
  
  and fix any finite $X\subset\M$. Then $\P$ contains an elementary embedding $\pi^*\colon\M\to\N$ which agrees with $\pi$ on $X$. If $\pi$ has a critical point and if $\M$ and $\N$ are both transitive then we can also assume that $\crit\pi=\crit\pi^*$.\footnote{We are using transitivity of $\M$ and $\N$ to ensure that the \textit{ordinal} $\crit\pi$ exists.}
}
\proof{
  Let $\{a_i\mid i<\omega\}\in\P$ be an enumeration of $\M$ and set $\M\restr n:=\{a_i\mid i<n\}$. Then, in $\P$, build the tree $\T$ of all partial isomorphisms between $\M\restr n$ and $\N$ for $n<\omega$, ordered by extension. Then $\T$ is illfounded in $V$ by assumption, so it's also illfounded in $\P$ since $\P$ is transitive and $\P\models\zf^-+\dc$. The branch then gives us the embedding $\pi^*$, and if $\crit\pi$ exists then we can ensure that it agrees with $\pi$ on the critical point and finitely many values by adding these conditions to $\T$.
}

We'll need the following well-known lemmata; see Lemma 7.1 in \cite{HolySchlicht} and Lemma 3.1 in \cite{GitmanSchindler} for their proofs.

\qlemm[Ancient Kunen Lemma]{
  \label{lemm.Kunen}
  Let $M\models\zfc^-$ and $j:M\to N$ an elementary embedding with critical point $\kappa$ such that $\kappa+1\subset M\subset N$. Assume that $X\in M$ has $M$-cardinality $\kappa$. Then $j\restr X\in N$.
}


\section{Ideals}
\lipsum[1]



\end{document}
