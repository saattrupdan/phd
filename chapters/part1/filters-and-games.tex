\documentclass[../../main]{subfiles}
\pagestyle{fancy}

\begin{document}

\chapter{Filters \& Games}
\label{chapter.filters-and-games}
\thispagestyle{fancy}

Moving away from the pure theory of the virtual large cardinals from Chapter \ref{chapter.virtual-large-cardinals}, we now move to connections between these large cardinals and common set-theoretic objects of study. In this chapter those objects are filters and games, with the next chapter dealing with connections to ideals. This chapter covers the content of the paper \cite{NielsenWelch}, which started out as a further analysis of the results in \cite{HolySchlicht} and somewhat surprisingly we ended up in the realm of virtual large cardinals.

\qquad We will in this section be dealing with many properties of $\M$-measures\footnote{See Section \ref{prelims.filters} for the definitions of weak $\kappa$-models $\M$ and their associated $\M$-measures.}, so we start with a couple of definitions.

\xdefi{
	Let $\kappa$ be a cardinal, $\M$ a weak $\kappa$-model and $\mu$ an $\M$-measure. Then $\mu$ is...
  \begin{itemize}
    \item \textbf{$\M$-normal} if $(\M,\in,\mu)\models\forall\vec X\in{^\kappa}\mu:\triangle\vec X\in\mu$;
    \item \textbf{genuine} if $|\triangle\vec X|=\kappa$ for every $\kappa$-sequence $\vec X\in{^\kappa\mu}$;
    \item \textbf{normal} if $\triangle\vec X$ is stationary in $\kappa$ for every $\kappa$-sequence $\vec X\in{^\kappa\mu}$;
    \item \textbf{$0$-good}, or simply \textbf{good}, if it has a well-founded ultrapower when applied to $\M$;
    \item \textbf{$\alpha$-good} for $\alpha>0$ if it is weakly amenable and has $\alpha$-many well-founded iterated ultrapowers when applied to $\M$.$\hfill\circ$
  \end{itemize}
}

We emphasise that the main difference between $\M$-normality and normality (and genuineness) is that the former is \textit{local} and the latter are \textit{global}.\todo{Question: Are normal measures always $\M$-normal?}

\qquad We note a few basic relations between these properties.

\prop{
  Let $\kappa$ be a cardinal, $\M$ a weak $\kappa$-model. Then
  \begin{enumerate}
    \item Every genuine $\M$-measure on $\kappa$ is countably complete;
    \item Every countably complete weakly amenable $\M$-measure on $\kappa$ is $\alpha$-good for all ordinals $\alpha$.
  \end{enumerate}
}
\proof{
  $(i)$: Let $\mu$ be a genuine $\M$-measure on $\kappa$. To show countable completeness, let $\vec X\in{^\omega}\mu$ be an $\omega$-sequence of measure one sets and define a $\kappa$-sequence $\vec Y\in{^\kappa}\mu$ as $Y_n:=X_n$ for $n<\omega$ and $Y_\alpha:=\kappa$ for $\alpha\in[\omega,\kappa)$. Then $\abs{\triangle\vec Y}=\kappa$ as $\mu$ is genuine, so letting $\alpha\in\triangle\vec Y-\omega$ we get that $\alpha\in\bigcap\vec X$, making $\mu$ countably complete.
  
  \qquad $(ii)$: Now let $\mu$ be a countably complete weakly amenable $\M$-measure on $\kappa$. Firstly note that countable completeness implies that the ultrapower $\ult(\M,\mu)$ is well-founded. Next, weak amenability implies that $X:=\{\alpha<\kappa\mid X_\alpha\in\mu\}\in\M$ for every $\vec X\in{^\kappa\mu}\cap\M$ since we can rewrite the set as
  \eq{
    X=\{\alpha<\kappa\mid X_\alpha\in\{X_\alpha\mid\alpha<\kappa\}\cap\mu\} 
  }
  
  and weak amenability ensures that $\{X_\alpha\mid\alpha<\kappa\}\cap\mu\in\M$. From this we can form iterated ultrapowers as in Chapter 19 of \cite{Kanamori}, which will all be well-founded by countable completeness of the measure.
}

In \cite{HolySchlicht} they provide the following characterisation of the normal measures.

\lemm[Holy-Schlicht]{
  Let $\M$ be a weak $\kappa$-model and $\mu$ an $\M$-measure. Then $\mu$ is normal iff $\triangle\vec X$ is stationary for some enumeration $\vec X$ of $\mu$.
}
\proof{
  $(\Rightarrow)$ is trivial since $\abs{\vec X}=\abs{\mu}\leq\abs{\M}=\kappa$, so assume that $\vec X$ is an enumeration of $\mu$ such that $\triangle\vec X$ is stationary. Let $\vec Y\in{^\kappa\mu}$ be a $\kappa$-sequence and define $g\colon\kappa\to\kappa$ such that $Y_\alpha = X_{g(\alpha)}$ for $\alpha<\kappa$. Letting $C_g\subset\kappa$ be the club of closure points of $g$ we get that $\triangle\vec X\cap C_g\subset \triangle\vec Y\cap C_g$, making $\triangle\vec Y$ stationary.
}

We next move on to the games. All of our games will be two-player games with perfect information; see e.g. \cite[Chapter 27]{Kanamori} for an introduction to set-theoretic game theory. We will also, mostly for convenience, use the following \textit{game equivalence} notion.

\defi{
  Two games $\G_0$ and $\G_1$ are said to be \textbf{game equivalent}, or simply \textbf{equivalent}, if player I has a winning strategy in $\G_0$ iff they have one in $\G_1$, and player II has a winning strategy in $\G_0$ iff they have one in $\G_1$. We will also denote such an equivalence as $\G_0\sim\G_1$.
}

The following is a game which was introduced in \cite{HolySchlicht} and led to their notion of \textit{$\alpha$-Ramsey cardinals}.

\defi[Holy-Schlicht]{
	For an uncountable cardinal $\kappa=\kappa^{<\kappa}$, a regular cardinal $\gamma\leq\kappa$ and a regular cardinal $\theta>\kappa$ define the game $wfG_\gamma^\theta(\kappa)$ of length $\gamma$ as follows.
	\game{\M_0}{\mu_0}{\M_1}{\mu_1}{\M_2}{\mu_2}{\cdots}{\cdots}

 Here $\M_\alpha\prec H_\theta$ is a $\kappa$-model and $\mu_\alpha$ is an $\M_\alpha$-measure, the $\M_\alpha$'s and $\mu_\alpha$'s are $\subset$-increasing and $\bra{\M_\xi\mid\xi<\alpha},\bra{\mu_\xi\mid\xi<\alpha}\in\M_\alpha$ for every $\alpha<\gamma$. Letting $\mu:=\bigcup_{\alpha<\gamma}\mu_\alpha$ and $\M:=\bigcup_{\alpha<\gamma}\M_\alpha$, player II wins iff $\mu$ is an $\M$-normal good $\M$-measure.
}

We will also be using the following fact from \cite[Lemma 3.3]{HolySchlicht}, that the games $wfG_\gamma^{\theta}(\kappa)$ do not depend upon the values of $\theta$.

\qlemm[Holy-Schlicht][lemm.holy-schlicht-theta-does-not-matter]{
  For a fixed $\kappa$ and $\gamma$, $wfG_\gamma^{\theta_0}(\kappa)$ and $wfG_\gamma^{\theta_1}(\kappa)$ are equivalent for any regular $\theta_0,\theta_1>\kappa$.
}

See the proof of Proposition \ref{prop.theta-does-not-matter} below for an idea of the proof strategy of this lemma.

\qquad We will be working with the following variant of the $wfG_\gamma(\kappa)$ games in which we require less of player I and more of player II. It will turn out that this change of game is innocuous, as Proposition \ref{prop.tildegame} will show that they are (game) equivalent.

\defi[Holy-N.-Schlicht]{
  Let $\kappa=\kappa^{<\kappa}$ be an uncountable cardinal, $\gamma\leq\kappa$ and $\zeta$ ordinals and $\theta>\kappa$ a regular cardinal. Then define the following \textbf{filter game} $\G_\gamma^\theta(\kappa,\zeta)$ with $(\gamma{+}1)$-many rounds.
	\game{\M_0}{\mu_0}{\M_1}{\mu_1}{\cdots}{\cdots}{\M_\gamma}{\mu_\gamma}

	Here $\M_\alpha\prec H_\theta$ is a weak $\kappa$-model for every $\alpha\leq\gamma$, $\mu_\alpha$ is a normal $\M_\alpha$-measure for $\alpha<\gamma$, $\mu_\gamma$ is an $\M_\gamma$-normal good $\M_\gamma$-measure and the $\M_\alpha$'s and $\mu_\alpha$'s are $\subset$-increasing. For limit ordinals $\alpha\leq\gamma$ we furthermore require that $\M_\alpha=\bigcup_{\xi<\alpha}\M_\xi$, $\mu_\alpha=\bigcup_{\xi<\alpha}\mu_\xi$ and that $\mu_\alpha$ is $\zeta$-good. Player II wins iff they could continue to play throughout all $(\gamma{+}1)$-many rounds.
}

Note that we assume that $\kappa=\kappa^{<\kappa}$ is uncountable in the definition of the games that we are considering, so this is a standing assumption throughout this chapter, whenever any one of the above two games are considered.

\qquad As in Lemma \ref{lemm.holy-schlicht-theta-does-not-matter}, we do not have to worry about the $\theta$ parameter. The proof is almost identical to the proof of Lemma \ref{lemm.holy-schlicht-theta-does-not-matter}, but we supply it here for completeness.

\prop[Holy-Schlicht][prop.theta-does-not-matter]{
$\G_\gamma^{\theta_0}(\kappa)\sim\G_\gamma^{\theta_1}(\kappa)$ for all regular $\theta_0, \theta_1>\kappa$.
}
\proof{
  Fix regular cardinals $\theta_0,\theta_1>\kappa$. First assume that player I has a winning strategy $\sigma_0$ in $\G_\gamma^{\theta_0}(\kappa)$; we will informally describe a winning strategy $\sigma_1$ for player I in $\G_\gamma^{\theta_1}(\kappa)$. Whenever $\sigma_0$ plays a weak $\kappa$-model $\M_\alpha\prec H_{\theta_0}$ we simply let $\sigma_1$ play
  \eq{
    \widetilde\M_\alpha:=\hull^{H_{\theta_1}}(\p^{\M}(\kappa)\cup\bigcup_{\xi<\alpha}\widetilde\M_\xi), 
  }
  
  which is by definition an elementary substructure of $H_{\theta_1}$. Then any response from player II in $\G_\gamma^{\theta_1}(\kappa)$ is also a valid response in $\G_\gamma^{\theta_0}(\kappa)$, as $\p^{\M_\alpha}(\kappa)\subset\p^{\widetilde\M_\alpha}(\kappa)$, so $\sigma_1$ is well-defined. Further, as $\sigma_0$ is winning for player I, we get that $\mu_\gamma$, the last measure played by player II in $\G_\gamma^{\theta_0}(\kappa)$, does not have a well-founded ultrapower when applied to $\M_\gamma$. This is witnessed by an $\omega$-sequence $\bra{f_n\mid n<\omega}$ of functions $f_n\colon\kappa\to\kappa$ in $\M_\gamma$ such that
  \eq{
    \{\alpha<\kappa\mid f_{n+1}(\alpha)\in f_n(\alpha\}\in\mu_\gamma.
  }

  These $f_n$'s can be encoded as subsets of $\kappa$, so that $f_n\in\widetilde\M_\gamma$ as well, meaning that $\mu_\gamma$ does not have a well-founded ultrapower when applied to $\widetilde\M_\gamma$ either.

  \qquad Next, assume that player II has a winning strategy $\tau_0$ in $\G_\gamma^{\theta_0}(\kappa)$; again we describe a winning strategy $\tau_1$ for player II in $\G_\gamma^{\theta_1}(\kappa)$. Here, every time player I plays some $\M$ in $\G_\gamma^{\theta_1}(\kappa)$ we form $\widetilde\M\prec H_{\theta_0}$ as above and let $\tau_1$'s response to $\M$ be $\tau_0$'s response to $\widetilde\M$. Since $\tau_0$'s measures have well-founded ultrapowers when applied to the $\widetilde\M_\gamma$'s, they will also have well-founded ultrapowers when applied to the $\M_\alpha$'s, for the same reason as above.
}

We will for convenience write $\G_\gamma^\theta(\kappa)$ for the game $\G_\gamma^\theta(\kappa,0)$, and with the above Proposition \ref{prop.theta-does-not-matter} in mind we will also write $\G_\gamma(\kappa)$ for $\G_\gamma^\theta(\kappa)$.

\qquad We will at times work with the following weakening of the filter game.

\defi{
  Define the \textbf{weak filter game} $\G_\gamma^-(\kappa, \zeta)$ like $\G_\gamma^{\kappa^+}(\kappa, \zeta)$, but where we do not require that $\mu_\gamma$ is $\zeta$-good.
}

We will also be working with the following variant of these games.



\defi{
	Define the \textbf{Cohen game} $\C_\gamma^\theta(\kappa)$ as $\G_\gamma^\theta(\kappa)$ but where we require that $\abs{\M_\alpha-H_\kappa}<\gamma$ for every $\alpha<\gamma$, i.e. that we only allow player I to add ${<}\gamma$ new elements to the models in each round, and where we only require $\M_\alpha\models\zfc^-$ and $\M_\alpha\prec H_\theta$ for $\alpha\leq\gamma$ limit.\footnote{$\C_\omega^\theta(\kappa)$ is similar to the $H(F,\lambda)$-games in \cite{DonderLevinski}.}

	\qquad Also define the \textbf{weak Cohen game} $\C_\gamma^-(\kappa)$ in analogy with $\G_\gamma^-(\kappa)$.
}

\prop[N.]{
	Assume $\gamma^{\aleph_0}=\gamma$ and let $\kappa$ be regular. Then $\C_\gamma^-(\kappa)$ is equivalent to $\C_\gamma^\theta(\kappa)$ for all regular $\theta>\kappa$. In particular, if $\ch$ holds then $\C_{\omega_1}^-(\kappa)$ is equivalent to $\C_{\omega_1}^\theta(\kappa)$ for all regular $\theta>\kappa$.
}
\proof{
	The assumption that $\gamma^{\aleph_0}=\gamma$ allows us to ensure without loss of generality that ${^\omega}\M_\alpha\subset\M_\gamma$ holds for all $\alpha<\gamma$: If player I has a winning strategy in $\C_\gamma^\theta(\kappa)$ for some regular $\theta>\kappa$ then they still win if we require that ${^\omega}\M_\alpha\subset\M_\gamma$ (since they are only enlargening their models, making it even harder for player II to win), in which case the final measure $\mu_\gamma$ is countably complete and hence automatically has a well-founded ultrapower.
	
	\qquad If player II has a winning strategy in $\C_\gamma^-(\kappa)$ then they still win if player I plays $\M_\alpha$ such that ${^\omega}\M_\alpha\subset\M_\gamma$, ensuring that $\mu_\gamma$ has a well-founded ultrapower.
}

\prop[Holy-N.-Schlicht][prop.tildegame]{
  $\G_\gamma^\theta(\kappa)$, $\G_\gamma^\theta(\kappa,1)$ and $wfG_\gamma^\theta(\kappa)$ are all game equivalent for all limit ordinals $\gamma\leq\kappa$, and $\G_\gamma^\theta(\kappa,\zeta)$ is equivalent to $\G_\gamma^\theta(\kappa)$ whenever $\cof\gamma>\omega$ and $\zeta\in\on$.
}
\proof{
	We start by showing the latter statement, so assume that $\cof\gamma>\omega$. Consider now the auxilliary game, call it $\mathcal G$, which is exactly like $\G_\gamma^\theta(\kappa,0)$, but where we also require that $^\omega{\M_\alpha}\subset\M_{\alpha+1}$ and $\bra{\M_\xi\mid\xi\leq\alpha},\bra{\mu_\xi\mid\xi\leq\alpha}\in\M_{\alpha+1}$ for every $\alpha<\gamma$.

	\clai{
		$\G$ is equivalent to $\G_\gamma^\theta(\kappa)$.
	}

	\cproof{
		If player I has a winning strategy in $\mathcal G$ then they also have one in $\G_\gamma^\theta(\kappa)$, by doing exactly the same. Analogously, if player II has a winning strategy in $\G_\gamma^\theta(\kappa)$ then they also have one in $\mathcal G$. If player I has a winning strategy $\sigma$ in $\G_\gamma^\theta(\kappa)$ then we can construct a winning strategy $\sigma'$ in $\mathcal G$, which is defined as follows. Fix $\alpha\leq\gamma$ and, writing $\vec\M_\xi:=\bra{\M_\xi\mid\xi\leq\alpha}$ and $\vec\mu_\xi:=\bra{\mu_\xi\mid\xi\leq\alpha}$, we set
		\eq{
			\sigma'(\bra{\M_\xi,\mu_\xi\mid\xi\leq\alpha}):=\hull^{H_\theta}(\sigma(\bra{\M_\xi,\mu_\xi\mid\xi\leq\alpha})\cup{^\omega{\M_\alpha}}\cup\{\vec\M_\xi,\vec\mu_\xi\}),
		}

		i.e. that we are simply throwing in the sequences into our models and making sure that we are still an elementary substructure of $H_\theta$. This new strategy $\sigma'$ is clearly winning. Assuming now that $\tau$ is a winning strategy for player II in $\mathcal G$, we define a winning strategy $\tau'$ for player II in $\G_\gamma^\theta(\kappa)$ by letting $\tau'(\bra{\M_\xi,\mu_\xi\mid\xi\leq\alpha})$ be the result of throwing in the appropriate sequences into the models $\M_\xi$, applying $\tau$ to get a measure, and intersecting that measure with $\M_\alpha$ to get an $\M_\alpha$-measure.
	}

  Now, letting $\M_\gamma$ be the final model of a play of $\mathcal G$, $\cof\gamma>\omega$ implies that any $\omega$-sequence $\vec X\in\M_\gamma$ really is a sequence of elements from some $\M_\xi$ for $\xi<\gamma$, so that $\vec X\in\M_{\xi+1}$ by definition of $\mathcal G$, making $\M_\gamma$ closed under $\omega$-sequences and thus also $\mu_\gamma$ countably complete. Since $\gamma$ is a limit ordinal and the models contain the previous measures and models as elements, the proof of e.g. Theorem 5.6 in \cite{HolySchlicht} \todo{Maybe prove this here?} shows that $\mu_\gamma$ is also weakly amenable, making it $\zeta$-good for all ordinals $\zeta$.

    \qquad Now we deal with the first statement, stating that $\G_\gamma^\theta(\kappa)$, $\G_\gamma^\theta(\kappa,1)$ and $wfG_\gamma^\theta(\kappa)$ are all game equivalent for all limit ordinals $\gamma\leq\kappa$. Fix a limit ordinal $\gamma$. Firstly $\G_\gamma^\theta(\kappa)$ is equivalent to $\G_\gamma^\theta(\kappa,1)$ as above, since both are equivalent to the auxilliary game $\G$ when $\gamma$ is a limit ordinal. So it remains to show that $\G_\gamma^\theta(\kappa)$ is equivalent to $wfG_\gamma^\theta(\kappa)$. If player I has a winning strategy $\sigma$ in $wfG_\gamma^\theta(\kappa)$ then define a winning strategy $\sigma'$ for player I in $\G_\gamma^\theta(\kappa)$ as
	\eq{
		\sigma'(\bra{\M_\xi,\mu_\xi\mid\xi\leq\alpha}):=\sigma(\bra{\M_0,\mu_0}^\smallfrown\bra{\M_{\xi+1},\mu_{\xi+1}\mid\xi+1\leq\alpha})
	}

	and for limit ordinals $\alpha\leq\gamma$ set $\sigma'(\bra{\M_\xi,\mu_\xi\mid\xi<\alpha}):=\bigcup_{\xi<\alpha}\M_\xi$; i.e. they simply follow the same strategy as in $wfG_\gamma^\theta(\kappa)$ but plugs in unions at limit stages. Likewise, if player II had a winning strategy in $\G_\gamma^\theta(\kappa)$ then they also have a winning strategy in $wfG_\gamma^\theta(\kappa)$, this time just by skipping the limit steps in $\G_\gamma^\theta(\kappa)$.

	\qquad Now assume that player I has a winning strategy $\sigma$ in $\G_\gamma^\theta(\kappa)$ and that player I does \textit{not} have a winning strategy in $wfG_\gamma^\theta(\kappa)$. Then define a strategy $\sigma'$ for player I in $wfG_\gamma^\theta(\kappa)$ as follows. Let $s=\bra{\M_\alpha,\mu_\alpha\mid\alpha\leq\eta}$ be a partial play of $wfG_\gamma^\theta(\kappa)$ and let $s'$ be the modified version of $s$ in which we have ``inserted'' unions at limit steps, just as in the above paragraph. We can assume that every $\mu_\alpha$ in $s'$ is good and $\M_\alpha$-normal as otherwise player II has already lost and player I can play anything. Now, we want to show that $s'$ is a valid partial play of $\G_\gamma^\theta(\kappa)$. All the models in $s$ are $\kappa$-models, so in particular weak $\kappa$-models.
	
	\clai{
		Every $\mu_\alpha$ in $s'$ is normal.
	}	
	
	\cproof{
		Assume without loss of generality that $\alpha=\eta$. Let player I play any legal response $\M$ to $s$ in $wfG_\gamma^\theta(\kappa)$ (such a response always exists). If player II cannot respond then player I has a winning strategy by simply following $s^\cap\bra{\M}$, $\contr$, so player II \textit{does} have a response $\mu$ to $s^\cap\M$. But now the rules of $wfG_\gamma^\theta(\kappa)$ ensures that $\mu_\eta\in\M$, so since
		\eq{
		(\M,\in,\mu)\models\forall\vec X\in{^\kappa}\mu:\godel{\triangle\vec X\text{ is stationary in }\kappa},
		}

		we then also get that $\M\models\godel{\triangle\mu_\eta\text{ is stationary in $\kappa$}}$ since $\mu_\eta\subset\mu$, so elementarity of $\M$ in $H_\theta$ implies that $\triangle\mu_\eta$ really \textit{is} stationary in $\kappa$, making $\mu_\eta$ normal.
	}
	
	This makes $s'$ a valid partial play of $\G_\gamma^\theta(\kappa)$, so we may form the weak $\kappa$-model $\tilde\M_\eta:=\sigma(s')$. Now let $\M_\eta\prec H_\theta$ be a $\kappa$-model with $\tilde\M_\eta\subset\M_\eta$ and $s\in\M_\eta$ and set $\sigma'(s):=\M_\eta$. This defines the strategy $\sigma'$ for player I in $wfG_\gamma^\theta(\kappa)$, which is winning since the winning condition for the two games is the same for $\gamma$ a limit. More precisely, that $\sigma$ is winning in $\G_\gamma^\theta(\kappa)$ means that there is a sequence
  \eq{
    \bra{f_n:\kappa\to\kappa\mid n<\omega} 
  }
  
  with the $f_n$'s all being elements of the last model $\tilde\M_\gamma$, witnessing the illfoundedness of the ultrapower. But then all these functions will also be elements of the union of the $\M_\alpha$'s, since we ensured that $\M_\alpha\supset\tilde\M_\alpha$ in the construction above, making the ultrapower of $\bigcup_{\alpha<\gamma}\M_\alpha$ by $\bigcup_{\alpha<\gamma}\mu_\alpha$ illfounded as well.

	\qquad Next, assume that player II has a winning strategy $\tau$ in $wfG_\gamma^\theta(\kappa)$. We recursively define a strategy $\tilde\tau$ for player II in $\G_\gamma^\theta(\kappa)$ as follows. If $\tilde\M_0$ is the first move by player I in $\G_\gamma^\theta(\kappa)$, let $\M_0\prec H_\theta$ be a $\kappa$-model with $\tilde\M_0\subset\M_0$, making $\M_0$ a valid move for player I in $wfG_\gamma^\theta(\kappa)$. Write $\mu_0:=\tau(\bra{\M_0})$ and then set $\tilde\tau(\bra{\tilde\M_0})$ to be $\tilde\mu_0:=\mu_0\cap\tilde\M_0$, which again is normal by the same trick as above, making $\tilde\mu_0$ a legal move for player II in $\G_\gamma^\theta(\kappa)$. Successor stages $\alpha+1$ in the construction are analogous, but we also make sure that $\bra{\M_\xi\mid\xi<\alpha+1},\bra{\mu_\xi\mid\xi<\alpha+1}\in\M_{\alpha+1}$. At limit stages $\tau$ outputs unions, as is required by the rules of $\G_\gamma^\theta(\kappa)$. Since the union of all the $\mu_\alpha$'s is good as $\tau$ is winning, $\tilde\mu_\gamma:=\bigcup_{\alpha<\gamma}\tilde\mu_\alpha$ is good as well, making $\tilde\tau$ winning and we are done.
}

We now arrive at the definitions of the cardinals we will be considering. They were in \cite{HolySchlicht} only defined for $\gamma$ being a cardinal, but given the above result we generalise it to all ordinals $\gamma$.

\defi{
	Let $\kappa$ be a cardinal and $\gamma\leq\kappa$ an ordinal. Then $\kappa$ is \textbf{$\gamma$-Ramsey} if player I does not have a winning strategy in $\G_\gamma^\theta(\kappa)$ for all regular $\theta>\kappa$. We furthermore say that $\kappa$ is \textbf{strategic $\gamma$-Ramsey} if player II \textit{does} have a winning strategy in $\G_\gamma^\theta(\kappa)$ for all regular $\theta>\kappa$. 
  
  \qquad Define \textbf{(strategic) genuine $\gamma$-Ramseys} and \textbf{(strategic) normal $\gamma$-Ramseys} analogously, but where we require the last measure $\mu_\gamma$ to be genuine and normal, respectively.
}

\defi[N.]{
	\label{defi.cohramsey}
	A cardinal $\kappa$ is \textbf{${<}\gamma$-Ramsey} if it is $\alpha$-Ramsey for every $\alpha<\gamma$, \textbf{almost fully Ramsey} if it is ${<}\kappa$-Ramsey and \textbf{fully Ramsey} if it is $\kappa$-Ramsey.
  
    \qquad Further, say that $\kappa$ is \textbf{coherent ${<}\gamma$-Ramsey} if it is strategic $\alpha$-Ramsey for every $\alpha<\gamma$ and that there exists a choice of winning strategies $\tau_\alpha$ in $\G_\alpha(\kappa)$ for player II satisfying that $\tau_\alpha\subset\tau_\beta$ whenever $\alpha<\beta$. In other words, there is a single strategy $\tau$ for player II in $\G_\gamma(\kappa)$ such that $\tau$ is a winning strategy for player II in $\G_\alpha(\kappa)$ for every $\alpha<\gamma$, but we do not require $\tau$ to be winning in $\G_\gamma(\kappa)$\footnote{Note that, with this terminology, ``coherent'' is a stronger notion than ``strategic''. We could have called the cardinals \textit{coherent strategic ${<}\gamma$-Ramseys}, but we opted for brevity instead.}.
}

This is not the original definition of (strategic) $\gamma$-Ramsey cardinals however, as this involved elementary embeddings between weak $\kappa$-models -- but as the following theorem of \cite{HolySchlicht} shows, the two definitions coincide whenever $\gamma$ is a regular cardinal.

\qtheo[Holy-Schlicht]{
	\label{theo.Ramseydef}
	For regular cardinals $\lambda$, a cardinal $\kappa$ is $\lambda$-Ramsey iff for arbitrarily large $\theta>\kappa$ and every subset $A\subset\kappa$ there is a weak $\kappa$-model $\M\prec H_\theta$ with $\M^{<\lambda}\subset\M$ and $A\in\M$ with an $\M$-normal 1-good $\M$-measure $\mu$ on $\kappa$.
}


\section{The finite case}

In this section we are going to consider properties of the $n$-Ramsey cardinals for finite $n$. Note in particular that the $\G_n^\theta(\kappa)$ games are determined, making the ``strategic'' adjective superfluous in this case. We further note that the $\theta$'s are also dispensible in this finite case:

\prop[N.]{
	Let $\kappa<\theta$ be regular cardinals and $n<\omega$. Then player II has a winning strategy in $\G_n^\theta(\kappa)$ iff they have a winning strategy in the game $\G_n(\kappa)$, which is defined as $\G_n^\theta(\kappa)$ except that we do not require that $\M_n\prec H_\theta$.
}
\proof{
	$\Leftarrow$ is clear, so assume that II has a winning strategy $\tau$ in $\G_n^\theta(\kappa)$. Whenever player I plays $\M_k$ in $\G_n(\kappa)$ for $k\leq n$ then define $\M_k^*:=\hull^{H_\theta}(\P)$ where $\P\cong\M_k$ is the transitive collapse of $\M_k$, and play $\M_k^*$ in $\G_n^\theta(\kappa)$. Let $\mu_k$ be the $\tau$-responses to the $\M_k^*$'s and let player II play the $\mu_k$'s in $\G_n(\kappa)$ as well.

	\qquad Assume that this new strategy is not winning for player II in $\G_n(\kappa)$, so that $\ult(\M_n,\mu_n)$ is illfounded. This is witnessed by some $\omega$-sequence $\vec f:=\bra{f_k\mid k<\omega}$ of $f_k\in{^\kappa o(\M_n)}\cap\M_n$ with $X_k:=\{\alpha<\kappa\mid f_{k+1}(\alpha)<f_k(\alpha)\}\in\mu_n$ for all $k<\omega$. Let $\nu\gg\kappa$, $\h:=\chull^{H_\nu}(\M_n\cup\{\vec f,\M_n,\mu_n\})$ be the transitive collapse of the Skolem hull $\hull^{H_\nu}(\M_n\cup\{\vec f,\M_n,\mu_n\})$, and $\pi:\h\to H_\nu$ be the uncollapse; write $\bar x:=\pi^{-1}(x)$ for all $x\in\ran\pi$.

	\qquad Now $\bar A=A$ for every $A\in\p(\kappa)\cap\M_n$ and thus also $\bar\mu_n=\mu_n$. But now the $\bar f_k$'s witness that $\ult(\bar\M_n,\mu_n)$ is illfounded and thus also that $\ult(\M_n^*,\mu_n)$ is illfounded since $\M_n^*=\hull^{H_\theta}(\bar\M_n)$, contradicting that $\tau$ is winning.
}

For this reason we will work with the $\G_n(\kappa)$ games throughout this section. Since we do not have to deal with the $\theta$'s anymore we note that $n$-Ramseyness can now be described using a $\Pi^1_{2n+2}$-formula and normal $n$-Ramseyness using a $\Pi^1_{2n+3}$-formula.

\qquad We have the following characterisations, as proven in \cite{Abramson}.

\theo[Abramson et al.]{
	Let $\kappa=\kappa^{<\kappa}$ be a cardinal. Then
	\begin{enumerate}
		\item $\kappa$ is weakly compact if and only if it is $0$-Ramsey;
		\item $\kappa$ is weakly ineffable if and only if it is genuine $0$-Ramsey;
		\item $\kappa$ is ineffable if and only if it is normal $0$-Ramsey.
	\end{enumerate}
}
\proof{
	This is mostly just changing the terminology in \cite{Abramson} to the current game-theoretic one, so we only show $(i)$.
  
  \qquad Theorem 1.1.3 in \cite{Abramson} shows that $\kappa$ is weakly compact if and only if every $\kappa$-sized collection of subsets of $\kappa$ is measured by a ${<}\kappa$-complete measure, in the sense that every ${<}\kappa$-sequence (in $V$) of measure one sets has non-empty intersection.
	
	\qquad For the $\Rightarrow$ direction we can let player II respond to any $\M_0$ by first getting the ${<}\kappa$-complete $\M_0$-measure $\nu_0$ on $\kappa$ from the above-mentioned result, forming the (well-founded) ultrapower $\pi:\M_0\to\ult(\M_0,\nu)$ and then playing the derived measure of $\pi$, which is $\M_0$-normal and good. For $\Leftarrow$, if $X\subset\p(\kappa)$ has size $\kappa$ then, using that $\kappa=\kappa^{<\kappa}$, we can find a $\kappa$-model $\M_0\prec H_\theta$ with $X\subset\M_0$. Letting player I play $\M_0$ in $\G_0(\kappa)$ we get some $\M_0$-normal good $\M_0$-measure $\mu_0$ on $\kappa$. Since $\M_0$ is closed under ${<}\kappa$-sequences we get that $\mu_0$ is ${<}\kappa$-complete.
}


\subsection{Indescribability}

In this section we aim to prove that $n$-Ramseys are $\Pi^1_{2n+1}$-indescribable and that normal $n$-Ramseys are $\Pi^1_{2n+2}$-indescribable, which will also establish that the hierarchy of alternating $n$-Ramseys and normal $n$-Ramseys forms a strict hierarchy. Recall the following definition.

\defi{
	A cardinal $\kappa$ is \textbf{$\Pi^1_n$-indescribable} if whenever $\varphi(v)$ is a $\Pi_n$ formula, $X\subset V_\kappa$ and $V_{\kappa+1}\models\varphi[X]$, then there is an $\alpha<\kappa$ such that $V_{\alpha+1}\models\varphi[X\cap V_\alpha]$.
}

Our first indescribability result is then the following, where the $n=0$ case is inspired by the proof of weakly compact cardinals being $\Pi^1_1$-indescribable --- see \cite{Abramson}.

\theo[N.]{
  \label{theo.ind}
	Every $n$-Ramsey $\kappa$ is $\Pi^1_{2n+1}$-indescribable for $n<\omega$.
}
\proof{
  Let $\kappa$ be $n$-Ramsey and assume that it is not $\Pi^1_{2n+1}$-indescribable, witnessed by a $\Pi_{2n+1}$-formula $\varphi(v)$ and a subset $X\subset V_\kappa$, meaning that $V_{\kappa+1}\models\varphi[X]$ and, for every $\alpha<\kappa$, $V_{\alpha+1}\models\lnot\varphi[X\cap V_\alpha]$. We will deal with the $(2n+1)$-many quantifiers occuring in $\varphi$ in $(n+1)$-many steps. We will here describe the first two steps with the remaining steps following the same pattern.

\qquad \framebox{\textbf{First step.}} Write $\varphi(v)\equiv\forall v_1\psi(v,v_1)$ for a $\Sigma_{2n}$-formula $\psi(v,v_1)$. As we are assuming that $V_{\alpha+1}\models\lnot\varphi[X\cap V_\alpha]$ holds for every $\alpha<\kappa$, we can pick witnesses $A^{(0)}_\alpha\subset V_\alpha$ to the outermost existential quantifier in $\lnot\varphi[X\cap V_\alpha]$.

\qquad Let $\M_0$ be a weak $\kappa$-model such that $V_\kappa\subset\M_0$ and $\vec A^{(0)},X\in\M_0$. Fix a good $\M_0$-normal $\M_0$-measure $\mu_0$ on $\kappa$, using the $0$-Ramseyness of $\kappa$. Form $\mathcal A^{(0)}:=[\vec A^{(0)}]_{\mu_0}\in\ult(\M_0,\mu_0)$, where we without loss of generality may assume that the ultrapower is transitive. $\M_0$-normality of $\mu_0$ implies that $\mathcal A^{(0)}\subset V_\kappa$, so that we have that $V_{\kappa+1}\models\psi[X,\mathcal A^{(0)}]$. Now \L o\' s' Lemma, $\M_0$-normality of $\mu_0$ and $V_\kappa\subset\M_0$ also ensures that
\eq{
  \ult(\M_0,\mu_0)\models\godel{V_{\kappa+1}\models\lnot\psi[X,\mathcal A^{(0)}]}.\tag*{$(1)$}
}

This finishes the first step. Note that if $n=0$ then $\lnot\psi$ would be a $\Delta_0$-formula, so that $(1)$ would be absolute to the true $V_{\kappa+1}$, yielding a contradiction. If $n>0$ we cannot yet conclude this however, but that is what we are aiming for in the remaining steps.\\

\qquad \framebox{\textbf{Second step.}} Write $\psi(v,v_1)\equiv\exists v_2\forall v_3\chi(v,v_1,v_2,v_3)$ for a $\Sigma_{2(n-1)}$-formula $\chi(v,v_1,v_2,v_3)$. Since we have established that $V_{\kappa+1}\models\psi[X,\mathcal A^{(0)}]$ we can pick some $B^{(0)}\subset V_\kappa$ such that
\eq{
  V_{\kappa+1}\models\forall v_3\chi[X,\mathcal A^{(0)},B^{(0)},v_3]\tag*{$(2)$}
}

which then also means that, for every $\alpha<\kappa$,
\eq{
  V_{\alpha+1}\models\exists v_3\lnot\chi[X\cap V_\alpha,A^{(0)}_\alpha,B^{(0)}\cap V_\alpha,v_3].\tag*{$(3)$}
}

Fix witnesses $A^{(1)}_\alpha\subset V_\alpha$ to the existential quantifier in $(3)$ and define the sets
\eq{
  S_\alpha^{(0)}:=\{\xi<\kappa\mid A_\xi^{(0)}\cap V_\alpha=\mathcal A^{(0)}\cap V_\alpha\}
}

for every $\alpha<\kappa$ and note that $S_\alpha^{(0)}\in\mu_0$ for every $\alpha<\kappa$, since $V_\kappa\subset\M_0$ ensures that $\mathcal A^{(0)}\cap V_\alpha\in\M_0$ and $\M_0$-normality of $\mu_0$ then implies that $S_\alpha^{(0)}\in\mu_0$ is equivalent to
\eq{
  \ult(\M_0,\mu_0)\models\mathcal A^{(0)}\cap V_\alpha=\mathcal A^{(0)}\cap V_\alpha,
}

which is clearly the case. Now let $\M_1\supset\M_0$ be a weak $\kappa$-model such that $\mathcal A^{(0)},\vec A^{(1)},\vec S^{(0)},B^{(0)}\in\M_1$. Let $\mu_1\supset\mu_0$ be an $\M_1$-normal $\M_1$-measure on $\kappa$, using the $1$-Ramseyness of $\kappa$, so that $\M_1$-normality of $\mu_1$ yields that $\triangle\vec S^{(0)}\in\mu_1$. Observe that $\xi\in\triangle\vec S^{(0)}$ if and only if $A^{(0)}_\xi\cap V_\alpha=\mathcal A^{(0)}\cap V_\alpha$ for every $\alpha<\xi$, so if $\xi$ is a limit ordinal then it holds that $A^{(0)}_\xi=\mathcal A^{(0)}\cap V_\xi$. Now, as before, form $\mathcal A^{(1)}:=[\vec A^{(1)}]_{\mu_1}\in\ult(\M_1,\mu_1)$, so that $(2)$ implies that
\eq{
  V_{\kappa+1}\models\chi[X,\mathcal A^{(0)},B^{(0)},\mathcal A^{(1)}]
}

and the definition of the $A_\alpha^{(1)}$'s along with $(3)$ gives that, for every $\alpha<\kappa$,
\eq{
  V_{\alpha+1}\models\lnot\chi[X\cap V_\alpha,A_\alpha^{(0)},B^{(0)}\cap V_\alpha,A_\alpha^{(1)}].
}

Now this, paired with the above observation regarding $\triangle\vec S^{(0)}$, means that for every $\alpha\in\triangle\vec S^{(0)}\cap\text{Lim}$ we have that
\eq{
  V_{\alpha+1}\models\lnot\chi[X\cap V_\alpha,\mathcal A^{(0)}\cap V_\alpha,B^{(0)}\cap V_\alpha,A_\alpha^{(1)}],
}

so that $\M_1$-normality of $\mu_1$ and \L o\' s' lemma implies that
\eq{
  \ult(\M_1,\mu_1)\models\godel{V_{\kappa+1}\models\lnot\chi[X,\mathcal A^{(0)},B^{(0)},\mathcal A^{(1)}]}.
}

This finishes the second step. Continue in this way for a total of $(n+1)$-many steps, ending with a $\Delta_0$-formula $\phi(v,v_1,\hdots,v_{2n+1})$ such that
\eq{
	V_{\kappa+1}\models\phi[X,\mathcal A^{(0)},B^{(0)},\hdots,\mathcal A^{(n-1)},B^{(n-1)},\mathcal A^{(n)}]\tag*{$(4)$}
}

and that $\ult(\M_n,\mu_n)\models\godel{V_{\kappa+1}\models\lnot\phi[X,\mathcal A^{(0)},B^{(0)},\hdots,\mathcal A^{(n)}]}$. But now absoluteness of $\lnot\phi$ means that $V_{\kappa+1}\models\lnot\phi[X,\mathcal A^{(0)},B^{(0)},\hdots,\mathcal A^{(n)}]$, contradicting $(4)$.
}

Note that this is optimal, as $n$-Ramseyness can be described by a $\Pi^1_{2n+2}$-formula.\footnote{Define the formula} As a corollary we then immediately get the following.

\qcoro[N.]{
	\label{coro.ind}
	Every ${<}\omega$-Ramsey cardinal is $\Delta^2_0$-indescribable.
}

The second indescribability result concerns the normal $n$-Ramseys, where the $n=0$ case here is inspired by the proof of ineffable cardinals being $\Pi^1_2$-indescribable --- see \cite{Abramson}.

\theo[N.]{
	\label{theo.normind}
	Every normal $n$-Ramsey $\kappa$ is $\Pi^1_{2n+2}$-indescribable for $n<\omega$.
}

Before we commence with the proof, note that we cannot simply do the same thing as we did in the proof of Theorem \ref{theo.ind}, as we would end up with a $\Pi^1_1$ statement in an ultrapower, and as $\Pi^1_1$ statements are not upwards absolute in general we would not be able to get our contradiction.\\

\proof{
	Let $\kappa$ be normal $n$-Ramsey and assume that it is not $\Pi^1_{2n+2}$-indescribable, witnessed by a $\Pi_{2n+2}$-formula $\varphi(v)$ and a subset $X\subset V_\kappa$. Use that $\kappa$ is $n$-Ramsey to perform the same $n+1$ steps as in the proof of Theorem \ref{theo.ind}. This gives us a $\Sigma_1$-formula $\phi(v,v_1,\hdots,v_{2n+1})$ along with sequences $\bra{\mathcal A^{(0)},\cdots,\mathcal A^{(n)}}$, $\bra{B^{(0)},\hdots,B^{(n-1)}}$ and a play $\bra{\M_k,\mu_k\mid k\leq n}$ of $\G_n(\kappa)$ in which player II wins and $\mu_n$ is normal, such that
\eq{
  V_{\kappa+1}\models\phi[X,\mathcal A^{(0)},B^{(0)},\hdots,\mathcal A^{(n-1)},B^{(n-1)},\mathcal A^{(n)}]\tag*{$(1)$}
}

and, for $\mu_n$-many $\alpha<\kappa$,
\eq{
  V_{\alpha+1}\models\lnot\phi[X\cap V_\alpha,\mathcal A^{(0)}\cap V_\alpha,B^{(0)}\cap V_\alpha,\hdots,\mathcal A^{(n-1)}\cap V_\alpha,B^{(n-1)}\cap V_\alpha,A^{(n)}_\alpha].
}

Now form $S^{(n)}_\alpha\in\mu_n$ as in the proof of Theorem \ref{theo.ind}. The main difference now is that we do not know if $\vec S^{(n)}\in\M_n$ (in the proof of Theorem \ref{theo.ind} we only ensured that $\vec S^{(k)}\in\M_{k+1}$ for every $k<n$ and we only defined $\vec S^{(k)}$ for $k<n$), but we can now use normality\footnote{Recall that this is stronger than just requiring it to be $\M_n$-normal --- we do not require $\vec S^{(n)}\in\M_n$.} of $\mu_n$ to ensure that we \textit{do} have that $\triangle\vec S^{(n)}$ is stationary in $\kappa$. This means that we get a stationary set $S\subset\kappa$ such that for every $\alpha\in S$ it holds that
\eq{
  V_{\alpha+1}\models\lnot\phi[X\cap V_\alpha,\mathcal A^{(0)}\cap V_\alpha,B^{(0)}\cap V_\alpha,\hdots, B^{(n-1)}\cap V_\alpha,\mathcal A^{(n)}\cap V_\alpha].\tag*{$(2)$}
}

Now note that since $\kappa$ is inaccessible it is $\Sigma^1_1$-indescribable, meaning that we can reflect $(1)$. Furthermore, Lemma 3.4.3 of \cite{Abramson} shows that the set of reflection points of $\Sigma^1_1$-formulas is in fact club, so intersecting this club with $S$ we get a $\zeta\in S$ satisfying that
\eq{
  V_{\zeta+1}\models\phi[X\cap V_\zeta,\mathcal A^{(0)}\cap V_\zeta,B^{(0)}\cap V_\zeta,\hdots, B^{(n-1)}\cap V_\zeta,\mathcal A^{(n)}\cap V_\zeta],
}

contradicting $(2)$.
}

Note that this is optimal as well, since normal $n$-Ramseyness can be described by a $\Pi^1_{2n+3}$-formula. In particular this then means that every $(n{+}1)$-Ramsey is a normal $n$-Ramsey stationary limit of normal $n$-Ramseys, and every normal $n$-Ramsey is an $n$-Ramsey stationary limit of $n$-Ramseys, making the hierarchy of alternating $n$-Ramseys and normal $n$-Ramseys a strict hierarchy.


\subsection{Downwards absoluteness to $L$}

Our absoluteness result below, Theorem \ref{theo.genuine-normal-absoluteness}, is inspired by arguments in \cite{Abramson}, and uses the following lemma from that paper.

\qlemm[Abramson et al][lemm.abramson-inacc]{
  There is a $\Pi^1_1$ formula $\varphi(A)$ such that, for any ordinal $\alpha$, $(V_\alpha, V_{\alpha+1})\models\varphi[A]$ iff $\alpha$ is a regular cardinal and $A$ is a non-constructible subset of $\alpha$.\footnote{This appears as Lemma 4.1.2 in \cite{Abramson}.}
}

\theo[N.][theo.genuine-normal-absoluteness]{
	Genuine- and normal $n$-Ramseys are downwards absolute to $L$, for every $n<\omega$.
}
\proof{
	Assume first that $n=0$ and that $\kappa$ is a genuine $0$-Ramsey cardinal. Let $\M\in L$ be a weak $\kappa$-model --- we want to find a genuine $\M$-measure inside $L$. By assumption we \textit{can} find such a measure $\mu$ in $V$; we will show that in fact $\mu\in L$. Fix any enumeration $\bra{A_\xi\mid\xi<\kappa}\in L$ of $\p(\kappa)\cap\M$. It then clearly suffices to show that $T\in L$, where $T:=\{\alpha<\kappa\mid A_\xi\in\mu\}$.

  \clai{
		\label{clai.pospart}
		$T\cap\alpha\in L$ for any $\alpha<\kappa$.
	}

	\cproof{
		Let $\vec B$ be the \textbf{$\mu$-positive part} of $\vec A$, meaning that $B_\xi:=A_\xi$ if $A_\xi\in\mu$ and $B_\xi:=\lnot A_\xi$ if $A_\xi\notin\mu$. As $\mu$ is genuine we get that $\triangle\vec B$ has size $\kappa$, so we can pick $\delta\in\triangle\vec B$ with $\delta>\alpha$. Then $T\cap\alpha=\{\xi<\alpha\mid\delta\in A_\xi\}$, which can be constructed within $L$.
	}

	Now let $\varphi$ be the $\Pi^1_1$ formula given by Lemma \ref{lemm.abramson-inacc}. If we therefore assume that $T\notin L$ then $(V_\kappa, V_{\kappa+1})\models\varphi[T]$, which by $\Pi^1_1$-indescribability of $\kappa$ means that there exists some $\alpha<\kappa$ such that $(V_\alpha, V_{\alpha+1})\models\varphi[T\cap V_\alpha]$, i.e. that $T\cap\alpha\notin L$, contradicting the claim. Therefore $\mu\in L$. It is still genuine in $L$ as $(\triangle\mu)^L=\triangle\mu$, and if $\mu$ was normal then that is still true in $L$ as clubs in $L$ are still clubs in $V$. The cases where $\kappa$ is a genuine- or normal $n$-Ramsey cardinal is analogous.
}

Since $(n{+}1)$-Ramseys are normal $n$-Ramseys we then immediately get the following.

\qcoro[N.]{
	Every $(n{+}1)$-Ramsey is normal $n$-Ramsey in $L$, for every $n<\omega$. In particular, ${<}\omega$-Ramseys are downwards absolute to $L$.
}


\subsection{Complete ineffability}

In this subsection we provide a characterisation of the \textit{completely ineffable} cardinals\footnote{See Appendix \ref{prelims.large-cardinals} for a definition of the completely ineffable cardinals.} in terms of the $\alpha$-Ramseys. To arrive at such a characterisation, we need a slight strengthening of the ${<}\omega$-Ramsey cardinals, namely the \textit{coherent ${<}\omega$-Ramseys} as defined in \ref{defi.cohramsey}. Note that a coherent ${<}\omega$-Ramsey is precisely a cardinal satisfying the $\omega$-filter property, as defined in \cite{HolySchlicht}.

\qquad The following theorem shows that assuming coherency does yield a strictly stronger large cardinal notion. The idea of its proof is closely related to the proof of Theorem \ref{theo.normind} (the indescribability of normal $n$-Ramseys), but the main difference is that we want everything to occur locally inside our weak $\kappa$-models. We will need another lemma from \cite{Abramson}.

\qlemm[Abramson et al][lemm.abramson-club]{
  Let $\kappa$ be inaccessible, $X\subset\kappa$ and $\varphi$ a $\Sigma^1_1$-formula such that $(V_\kappa, \in, X)\models\varphi[X]$. Then
  \eq{
    \{\alpha<\kappa\mid(V_\alpha,\in,X\cap V_\alpha)\models\varphi[X\cap V_\alpha]\}
  }

  is a club.
}

\theo[N.]{
	Every coherent ${<}\omega$-Ramsey is a stationary limit of ${<}\omega$-Ramseys.
}
\proof{
	Let $\kappa$ be coherent ${<}\omega$-Ramsey. Let $\theta\gg\kappa$ be regular and let $\M_0\prec H_\theta$ be a weak $\kappa$-model with $V_\kappa\subset\M_0$. Let then player I play arbitrarily while player II plays according to her coherent winning strategies in $\G_n(\kappa)$, yielding a weak $\kappa$-model $\M\prec H_\theta$ with an $\M$-normal $\M$-measure $\mu:=\bigcup_{n<\omega}\mu_n$ on $\kappa$.
		
	\qquad Assume towards a contradiction that $X:=\{\xi<\kappa\mid\xi\text{ is ${<}\omega$-Ramsey}\}\notin\mu$. Since $X=\bigcap\vec X$ and $\vec X\in\M$, where $X_n:=\{\xi<\kappa\mid\xi\text{ is $n$-Ramsey}\}$, we must have by $\M$-normality of $\mu$ that $\lnot X_k\in\mu$ for some $k<\omega$. Note that $\lnot X_k\in\M_0$ by elementarity, so that $\lnot X_k\in\mu_0$ as well. Perform the $k+1$ steps as in the proof of Theorem \ref{theo.normind} with $\varphi(\xi)$ being $\godel{\xi\text{ is $k$-Ramsey}}$, so that we get a weak $\kappa$-model $\M_{k+1}\prec H_\theta$, an $\M_{k+1}$-normal $\M_{k+1}$-measure $\tilde\mu_{k+1}$ on $\kappa$, a $\Sigma_1$-formula $\varphi(v,v_1,v_2,\hdots,v_{2k+1})$ and sequences $\bra{\mathcal A^{(0)},\hdots,\mathcal A^{(k)}}$ and $\bra{B^{(0)},\hdots,B^{(k-1)}}$ such that
	\eq{
		V_{\kappa+1}\models\varphi[\kappa,\mathcal A^{(0)},B^{(0)},\mathcal A^{(1)},B^{(1)},\hdots,\mathcal A^{(k-1)},B^{(k-1)},\mathcal A^{(k)}]\tag*{$(2)$}
	}

	and there is a $Y\in\tilde\mu_{k+1}$ with $Y\subset\lnot X_k$ such that given any $\xi\in Y$,
	\eq{
		V_{\xi+1}\models\lnot\varphi[\xi,A_\xi^{(0)},B^{(0)}\cap V_\xi,A^{(1)}_\xi,B^{(1)}\cap V_\xi,\hdots,A^{(k-1)}_\xi,B^{(k-1)}\cap V_\xi,A^{(k)}_\xi]\tag*{$(3)$},
	}

	where $\mathcal A^{(i)}=[\vec A^{(i)}]_{\mu_i}\in\ult(\M_i,\mu_i)$ as in the proof of Theorem \ref{theo.ind}.

	\qquad Since $\kappa$ in particular is $\Sigma^1_1$-indescribable, Lemma \ref{lemm.abramson-club} implies that we get a club $C\subset\kappa$ of reflection points of $(2)$. Let $\M_{k+2}\supset\M_{k+1}$ be a weak $\kappa$-model with $\mathcal A^{(k)}\in\M_{k+2}$, where the above $(n+1)$-steps ensured that the $B^{(i)}$'s and the remaining $\mathcal A^{(i)}$'s are all elements of $\M_{k+1}$. In particular, as $C$ is a definable subset in the $\mathcal A^{(i)}$'s and $B^{(i)}$'s we also get that $C\in\M_{k+2}$. Letting $\tilde\mu_{k+2}$ be the associated measure on $\kappa$, $\M_{k+2}$-normality of $\tilde\mu_{k+2}$ ensures that $C\in\tilde\mu_{k+2}$. Now define, for every $\alpha<\kappa$,
	\eq{
		S_\alpha:=\{\xi\in Y\mid\forall i\leq k:\mathcal A^{(i)}\cap V_\alpha=A^{(i)}_\xi\cap V_\alpha\}
	}

	and note that $S_\alpha\in\tilde\mu_{k+2}$ for every $\alpha<\kappa$. Write $\vec S:=\bra{S_\alpha\mid\alpha<\kappa}$ and note that since $\vec S$ is definable it is an element of $\M_{k+2}$ as well. Then $\M_{k+2}$-normality of $\tilde\mu_{k+2}$ ensures that $\triangle\vec S\in\tilde\mu_{k+2}$, so that $C\cap\triangle\vec S\in\tilde\mu_{k+2}$ as well. But letting $\zeta\in C\cap\triangle\vec S$ we see, as in the proof of Theorem \ref{theo.ind}, that
	\eq{
		V_{\zeta+1}\models\varphi[\zeta,A^{(0)}_\zeta,B^{(0)}\cap V_\zeta,A^{(1)}_\zeta,B^{(1)}\cap V_\zeta,\hdots,A^{(k)}_\zeta]
	}

	since $\triangle\vec S\subset Y$, contradicting $(3)$. Hence $X\in\mu$, and since $\M\prec H_\theta$ we have that $\M$ is correct about stationary subsets of $\kappa$, meaning that $\kappa$ is a stationary limit of ${<}\omega$-Ramseys.
}

Now, having established the strength of this large cardinal notion, we move towards complete ineffability. We recall the following definitions.

\defi{
	A collection $R\subset\p(\kappa)$ is a \textbf{stationary class} if
	\begin{enumerate}
		\item $R\neq\emptyset$;
		\item every $A\in R$ is stationary in $\kappa$;
		\item if $A\in R$ and $B\supset A$ then $B\in R$.
	\end{enumerate}
}

\defi{
	A cardinal $\kappa$ is \textbf{completely ineffable} if there is a stationary class $R$ such that for every $A\in R$ and $f:[A]^2\to 2$ there is an $H\in R$ homogeneous for $f$.
}

We then arrive at the following characterisation, influenced by the proof of Theorem 1.3.4 in \cite{Abramson}.

\theo[N.][theo.ineff]{
	A cardinal $\kappa$ is completely ineffable if and only if it is coherent ${<}\omega$-Ramsey.
}
\proof{
	$(\Leftarrow)$: Assume $\kappa$ is coherent ${<}\omega$-Ramsey, witnessed by strategies $\bra{\tau_n\mid n<\omega}$. Let $f:[\kappa]^2\to 2$ be arbitrary and form the sequence $\bra{A_\alpha^f\mid\alpha<\kappa}$ as
	\eq{
		A_\alpha^f:=\{\beta>\alpha\mid f(\{\alpha,\beta\})=0\}.
	}

	Let $\M_f$ be a transitive weak $\kappa$-model with $\vec A^f\in\M_f$, and let $\mu_f$ be the associated $\M_f$-measure on $\kappa$ given by $\tau_0$.\footnote{Technically we would have to require that $\M_f\prec H_\theta$ for some regular $\theta>\kappa$ to be able to use $\tau_0$, but note that we could simply get a measure on $\hull^{H_\theta}(\M_f)$ and restrict it to $\M_f$. We will use this throughout the proof.} $1$-Ramseyness of $\kappa$ ensures that $\mu_f$ is normal, meaning $\triangle\mu_f$ is stationary in $\kappa$. Define a new sequence $\vec B^f$ as the $\mu_f$-positive part of $\vec A^f$.\footnote{The \textit{$\mu$-positive part} was defined in Claim \ref{clai.pospart}.} Then $B_\alpha^f\in\mu_f$ for all $\alpha<\kappa$, so that normality of $\mu_f$ implies that $\triangle\vec B^f$ is stationary.
	
	\qquad Let now $\M_f'$ be a new transitive weak $\kappa$-model with $\M_f\subset\M_f'$ and $\mu_f\in\M_f'$, and use $\tau_1$ to get an $\M_f'$-measure $\mu_f'\supset\mu_f$ on $\kappa$. Then $\triangle\vec B^f\cap\{\xi<\kappa\mid A_\xi^f\in\mu_f\}$ and $\triangle\vec B^f\cap\{\xi<\kappa\mid A_\xi^f\notin\mu_f\}$ are both elements of $\M_f'$, so one of them is in $\mu_f'$; set $H_f$ to be that one. Note that $H_f$ is now both stationary in $\kappa$ and homogeneous for $f$.

	\qquad Now let $g:[H_f]^2\to 2$ be arbitrary and again form
	\eq{
		A_\alpha^g:=\{\beta\in H_f\mid\beta>\alpha\land g(\{\alpha,\beta\})=0\}
	}
	
	for $\alpha\in H_f$. Let $\M_{f,g}\supset\M_f'$ be a transitive weak $\kappa$-model with $\vec A^g\in\M_{f,g}$ and use $\tau_2$ to get an $\M_{f,g}$-measure $\mu_{f,g}\supset\mu_f'$ on $\kappa$. As before we then get a stationary $H_{f,g}\in\mu_{f,g}'$ which is homogeneous for $g$. We can continue in this fashion since $\tau_n\subset\tau_{n+1}$ for all $n<\omega$. Define then
	\eq{
		R:=\{A\subset\kappa\mid\exists\vec f:H_{\vec f}\subset A\},
	}

	where the $\vec f$'s range over finite sequences of functions as above; i.e. $f_0:[\kappa]^2\to 2$ and $f_{k+1}:[H_{f_k}]\to 2$ for $k<\omega$.	This is clearly a stationary class which satisfies that whenever $A\in R$ and $g:[A]^2\to 2$, we can find $H\in R$ which is homogeneous for $f$. Indeed, if we let $\vec f$ be such that $H_{\vec f}\subset A$, which exists as $A\in R$, then we can simply let $H:=H_{\vec f,g}$. This shows that $\kappa$ is completely ineffable.

	\qquad $(\Rightarrow)$: Now assume that $\kappa$ is completely ineffable and let $R$ be the corresponding stationary class. We show that $\kappa$ is $n$-Ramsey for all $n<\omega$ by induction, where we inductively make sure that the resulting strategies are coherent as well. Let player I in $\G_0(\kappa)$ play $\M_0$ and enumerate $\p(\kappa)\cap\M_0$ as $\vec A^0\bra{A^0_\alpha\mid\alpha<\kappa}$ such that $A^0_\xi\subset A^0_\zeta$ implies $\xi\leq\zeta$. For $\alpha<\kappa$ define sequences $r_\alpha:\alpha\to 2$ as $r_\alpha(\xi)=1$ iff $\alpha\in A^0_\xi$. Let $<_{\text{lex}}^\alpha$ be the lexicographical ordering on $^\alpha 2$. Define now a colouring $f:[\kappa]^2\to 2$ as
	\eq{
		f(\{\alpha,\beta\}):=\left\{\begin{array}{ll}0 & \text{if }r_{\min(\alpha,\beta)}<_{\text{lex}}^{\min(\alpha,\beta)}r_{\max(\alpha,\beta)}\restr\min(\alpha,\beta)\\ 1 & \text{otherwise}\end{array}\right.
	}

	Let $H_0\in R$ be homogeneous for $f$, using that $\kappa$ is completely ineffable. For $\alpha<\kappa$ consider now the sequence $\bra{r_\xi\restr\alpha\mid\xi\in H_0\land\xi>\alpha}$, which is of length $\kappa$ so there is an $\eta\in[\alpha,\kappa)$ satisfying that $r_\beta\restr\alpha=r_\gamma\restr\alpha$ for every $\beta,\gamma\in H_0$ with $\eta\leq\beta<\gamma$. Define $g:\kappa\to\kappa$ as $g(\alpha)$ being the least such $\eta$, which is then a continuous non-decreasing cofinal function, making the set of fixed points of $g$ club in $\kappa$ -- call this club $C$.

	\qquad Since $H_0$ is stationary we can pick some $\zeta\in C\cap H_0$. As $\zeta\in C$ we get $g(\zeta)=\zeta$, meaning that $r_\beta\restr\zeta=r_\gamma\restr\zeta$ holds for every $\beta,\gamma\in H_0$ with $\zeta\leq\beta<\gamma$. As $\zeta$ is also a member of $H_0$ we can let $\beta:=\zeta$, so that $r_\zeta=r_\gamma\restr\zeta$ holds for every $\gamma\in H_0$, $\gamma>\zeta$.	Now, by definition of $r_\alpha$ we get that for every $\alpha,\gamma\in H_0\cap C$ with $\alpha\leq\gamma$ and $\xi<\alpha$, $\alpha\in A^0_\xi$ iff $\gamma\in A^0_\xi$. Define thus the $\M_0$-measure $\mu_0$ on $\kappa$ as
	\eq{
		\mu_0(A^0_\xi)=1\quad&\text{iff}\quad(\forall\beta\in H_0\cap C)(\beta>\xi\to\beta\in A^0_\xi)\\
		&\text{iff}\quad(\exists\beta\in H_0\cap C)(\beta>\xi\land\beta\in A^0_\xi),
	}

	where the last equivalence is due to the above-mentioned property of $H_0\cap C$. Note that the choice of enumeration implies that $\mu_0$ is indeed a filter. Letting $\vec B=\bra{B_\alpha\mid\alpha<\kappa}$ be the $\mu_0$-positive part of $\vec A^0$, it is also simple to check that $H_0\cap C\subset\triangle\vec B$, making $\mu_0$ normal and hence also both $\M_0$-normal and good, showing that $\kappa$ is $0$-Ramsey.

	\qquad Assume now that $\kappa$ is $n$-Ramsey and let $\bra{\M_0,\mu_0,\hdots,\M_n,\mu_n,\M_{n+1}}$ be a partial play of $\G_{n+1}(\kappa)$. Again enumerate $\p(\kappa)\cap\M_{n+1}$ as $\vec A^{n+1}=\bra{A^{n+1}_\xi\mid\xi<\kappa}$, again satisfying that $\xi\leq\zeta$ whenever $A^{n+1}_\xi\subset A^{n+1}_\zeta$, but also such that given any $\xi<\kappa$ there are $\zeta,\zeta'\in(\xi,\kappa)$ satisfying that $A^{n+1}_\zeta\in\p(\kappa)\cap\M_n$ and $A^{n+1}_{\zeta'}\in(\p(\kappa)\cap\M_{n+1})-\M_n$. The plan now is to do the same thing as before, but we also have to check that the resulting measure extends the previous ones.

	\qquad Let $H_n\in R$ and $C$ be club in $\kappa$ such that $H_n\cap C\subset\triangle\mu_n$, which exist by our inductive assumption. For $\alpha<\kappa$ define $r_\alpha:\alpha\to 2$ as $r_\alpha(\xi)=1$ iff $\alpha\in A^{n+1}_\xi$, and define a colouring $f:[H_n]^2\to 2$ as
	\eq{
		f(\{\alpha,\beta\}):=\left\{\begin{array}{ll}0 & \text{if }r_{\min(\alpha,\beta)}<_{\text{lex}}^{\min(\alpha,\beta)}r_{\max(\alpha,\beta)}\restr\min(\alpha,\beta)\\ 1 & \text{otherwise}\end{array}\right.
	}

	As $H_n\in R$ there is an $H_{n+1}\in R$ homogeneous for $f$. Just as before, define $g:\kappa\to\kappa$ as $g(\alpha)$ being the least $\eta\in[\alpha,\kappa)$ such that $r_\beta\restr\alpha=r_\gamma\restr\alpha$ for every $\beta,\gamma\in H_{n+1}$ with $\eta\leq\beta<\gamma$, and let $D$ be the club of fixed points of $g$. As above we get that given any $\alpha,\gamma\in H_{n+1}\cap D$ with $\alpha\leq\gamma$ and $\xi<\alpha$, $\alpha\in A^{n+1}_\xi$ iff $\gamma\in A^{n+1}_\xi$. Define then the $\M_{n+1}$-measure $\mu_{n+1}$ on $\kappa$ as
	\eq{
		\mu_{n+1}(A^{n+1}_\xi)=1\quad&\text{iff}\quad(\forall\beta\in H_{n+1}\cap D\cap C)(\beta>\xi\to\beta\in A^{n+1}_\xi)\\
		&\text{iff}\quad(\exists\beta\in H_{n+1}\cap D\cap C)(\beta>\xi\land\beta\in A^{n+1}_\xi).
	}

	Then $H_{n+1}\cap D\cap C\subset\triangle\mu_{n+1}$, making $\mu_{n+1}$ normal, $\M_{n+1}$-normal and good, just as before. It remains to show that $\mu_n\subset\mu_{n+1}$. Let thus $A\in\mu_n$ be given, and say $A=A^{n+1}_\xi=A^n_\eta$, where $\vec A^n$ was the enumeration of $\p(\kappa)\cap\M_n$ used at the $n$'th stage. Then by definition of $\mu_n$ we get that for every $\beta\in H_n\cap C$ with $\beta>\eta$, $\beta\in A^n_\eta$. We need to show that
	\eq{
		(\exists\beta\in H_{n+1}\cap D\cap C)(\beta>\xi\land\beta\in A^{n+1}_\xi)
	}

	holds. But here we can simply pick a $\beta>\max(\xi,\eta)$ with $\beta\in H_{n+1}\cap D\cap C\subset H_n\cap C$. This shows that $\mu_n\subset\mu_{n+1}$, making $\kappa$ $(n{+}1)$-Ramsey and thus inductively also coherent ${<}\omega$-Ramsey.
}



\section{The countable case}

This section covers the (strategic) $\gamma$-Ramsey cardinals whenever $\gamma$ has countable cofinality. This case is special because, as we cannot ensure that the final measure in $\G_\gamma^\theta(\kappa)$ is countably complete and so the existence of winning strategies \textit{might} depend on $\theta$, in contrast with the uncountable cofinality case.

\subsection{[Strategic] $\omega$-Ramsey cardinals}

We now move to the strategic $\omega$-Ramsey cardinals and their relationship to the (non-strategic) $\omega$-Ramseys. 

\theo[Schindler-N.][theo.gengame]{
  Let $\kappa<\theta$ be regular cardinals. Then $\kappa$ is faintly $\theta$-measurable iff player II has a winning strategy in $\C^\theta_\omega(\kappa)$.
}
\proof{
  $(\Leftarrow):$ Fix a winning strategy $\sigma$ for player II in $\C^\theta_\omega(\kappa)$. Let $g\subset\col(\omega,H^V_\theta)$ be $V$-generic and in $V[g]$ fix an elementary chain $\bra{\M_n\mid n<\omega}$ of weak $\kappa$-models $\M_n\prec H^V_\theta$ such that $H^V_\theta\subset\bigcup_{n<\omega}\M_n$, using that $\theta$ is regular and has countable cofinality in $V[g]$. Player II follows $\sigma$, resulting in a $H^V_\theta$-normal $H^V_\theta$-measure $\mu$ on $\kappa$.

  \qquad We claim that $\ult(H^V_\theta,\mu)$ is well-founded, so assume not, witnessed by a sequence $\bra{g_n\mid n<\omega}$ of functions $g_n\colon\kappa\to\theta$ such that $g_n\in H^V_\theta$ and
  \eq{
    \{\alpha<\kappa\mid g_{n+1}(\alpha)<g_n(\alpha)\}\in\mu.
  }

  Now, in $V$, define a tree $\T$ of triples $(f,M_f,\mu_f)$ such that $f\colon\kappa\to\theta$, $M_f$ is a weak $\kappa$-model, $\mu_f$ is an $M_f$-measure on $\kappa$ and letting $f_0<_{\T}\cdots<_{\T}f_n=f$ be the $\T$-predecessors of $f$,
  \begin{itemize}
    \item $\bra{M_{f_0},\mu_{f_0},\hdots,M_{f_n},\mu_{f_n}}$ is a partial play of $\C^\theta_\omega(\kappa)$ in which player II follows $\sigma$; and
    \item $\{\alpha<\kappa\mid f_{k+1}(\alpha)<f_k(\alpha)\}\in\mu_{k+1}$ for every $k<n$.\\
  \end{itemize}

  Now the $g_n$'s induce a cofinal branch through $\T$ in $V[g]$, so by absoluteness of well-foundedness there is a cofinal branch $b$ through $\T$ in $V$ as well. But $b$ now gives us a play of $C^\theta_\omega(\kappa)$ where player II is following $\sigma$ but player I wins, a contradiction. Thus $\ult(H^V_\theta,\mu)$ is well-founded, so that the ultrapower embedding $\pi\colon H_\theta^V\to\ult(H_\theta^V,\mu)$ witnesses that $\kappa$ is faintly $\theta$-measurable.

  \qquad $(\Rightarrow):$ Assume that $\kappa$ is faintly $\theta$-measurable. Let $\mathbb P$ be a forcing $\dot\mu$ a $\mathbb P$-name for an $H_\theta^V$-normal $H_\theta^V$-measure on $\kappa$ and $\dot\pi$ a $\mathbb P$-name for the associated ultrapower embedding. Define a strategy for player II in $\C_\omega^\theta(\kappa)$ as follows: Whenever player I plays $\M_n$ then fix some $\mathbb P$-condition $p_n$ such that, letting $\bra{f_i^n\mid i<k}$ enumerate all functions in $\M_n$ with domain $\kappa$,
  \eq{
    p_n\forces\godel{\check\mu\cap\M_n=\check\mu_n\cap\forall i<\check k\colon\dot\pi(\check f_i^n)(\check\kappa)=\check\alpha_i^n},
  }

  with $\mu_n,\alpha_i^n\in V$. Note here that we can ensure $\mu_n\in V$ because it is finite. Also, ensure that the $p_n$'s are $\leq$-decreasing. Assume now that $\ult(\M_\omega,\mu_\omega)$ is illfounded, witnessed by functions $g_n\in{^\kappa\M_\omega}\cap\M_\omega$ for $n<\omega$. Then $g_n=f_{i_n}^{k_n}$ for some $k_n,i_n<\omega$, and hence $p_{k_{n+1}}\forces\godel{\check\alpha_{i_{n+1}}^{k_{n+1}}<\check\alpha_{i_n}^{k_n}}$ for every $n<\omega$, so in $V$ we get an $\omega$-sequence of strictly decreasing ordinals, $\contr$.
}

We note that the above Theorem along with our results from Chapter \ref{chapter.virtual-large-cardinals} shows that winning the Cohen games does not guarantee weak compactness.

\coro[N.]{
	Let $\kappa$ be inaccessible.
	\begin{enumerate}
		\item If player II wins $\C_\omega^\theta(\kappa)$ for all regular $\theta>\kappa$ then $\kappa$ is not necessarily weakly compact;
		\item If player II wins $\C_\kappa(\kappa)$ then $\kappa$ \textit{is} weakly compact.
	\end{enumerate}
}
\proof{
	The first claim is directly by Proposition \ref{prop.faintly-not-weakly-compact} and Theorem \ref{theo.gengame}, and the second claim is because the hypothesis implies that player II wins $\G_0(\kappa)$ so that inaccesibility of $\kappa$ makes $\kappa$ weakly compact --- see e.g. \cite{Ramsey1} for this characterisation of weak compactness.
}

Here is a near-analogous result of Theorem \ref{theo.gengame} for the $\G^\theta_\omega(\kappa)$ game.

\theo[Schindler-N.][theo.virtstrat]{
  Let $\kappa<\theta$ be regular cardinals. If $\kappa$ is virtually $\theta$-prestrong then player II has a winning strategy in $\G^\theta_\omega(\kappa)$, and if player II has a winning strategy in $\G_\omega^\theta(\kappa)$ then $\kappa$ is faintly $\theta$-power-measurable. In particular, $\G^\theta_\omega(\kappa)^L\sim\C^\theta_\omega(\kappa)^L$.
}

\proof{
  The second statement is exactly like the $(\Leftarrow)$ direction in the previous theorem, so we show the first statement. Assume $\kappa$ is virtually $\theta$-prestrong and fix a regular $\theta>\kappa$, a transitive $\M\in V$, a poset $\mathbb P$ and, in $V^{\mathbb P}$, an elementary embedding $\pi\colon H_\theta^V\to\M$ with $\crit\pi=\kappa$. Fix a name $\dot\mu$ and a $\mathbb P$-condition $p$ such that
  \eq{
    p\forces\godel{\text{$\dot\mu$ is a weakly amenable $\check H_\theta$-normal $\check H_\theta$-measure with a well-founded ultrapower}}.
  }

  We now define a strategy $\sigma$ for player II in $\G^\theta_\omega(\kappa)$ as follows. Whenever player I plays a weak $\kappa$-model $\M_n\prec H_\theta^V$, player II fixes $p_n\in\mathbb P$, an $\M_n$-measure $\mu_n$ and a function $\pi_n\colon\M_n\to\pi(\M_n)$ such that $p_0\leq p$, $p_n\leq p_k$ for every $k\leq n$ and that
  \eq{
    p_n\forces\godel{\dot\mu\cap\check\M_n=\check\mu_n\cap\check\mu_n=\dot\mu\restr\check\M_n}.\tag*{$(1)$}
  }

  Note that by the Ancient Kunen Lemma \ref{lemm.kunen} we get that $\pi\restr\M_n\in\M\subset V$, so such $\pi_n$ always exist in $V$. The $\mu_n$'s also always exist in $V$, by weak amenability of $\mu$. Player II responds to $\M_n$ with $\mu_n$. It is clear that the $\mu_n$'s are legal moves for player II, so it remains to show that $\mu_\omega:=\bigcup_{n<\omega}\mu_n$ has a well-founded ultrapower. Assume it has not, so that we have a sequence $\bra{g_n\mid n<\omega}$ of functions $g_n\colon\kappa\to\M_\omega:=\bigcup_{n<\omega}\M_n$ such that $g_n\in\M_\omega$ and
  \eq{
    X_{n+1}:=\{\alpha<\kappa\mid g_{n+1}(\alpha)<g_n(\alpha)\}\in\mu_\omega\tag*{$(2)$}
  }

  for every $n<\omega$. Without loss of generality we can assume that $g_n,X_n\in\M_n$. Then $(2)$ implies that $p_{n+1}\forces\godel{\dot\pi(\check g_{n+1})(\check\kappa)<\dot\pi(\check g_n)(\check\kappa)}$, but by $(1)$ this also means that
  \eq{
    p_{n+1}\forces\godel{\check\pi_{n+1}(\check g_{n+1})(\check\kappa)<\check\pi_n(\check g_n)(\check\kappa)},
  }

  so defining, in $V$, the ordinals $\alpha_n:=\pi_n(g_n)(\kappa)$, $(3)$ implies that $\alpha_{n+1}<\alpha_n$ for all $n<\omega$, $\contr$. So $\mu_\omega$ has a well-founded ultrapower, making $\sigma$ a winning strategy.
}

We get the following immediate corollary.

\qcoro[N.-Schindler]{
	\label{coro.strategic}
  Strategic $\omega$-Ramseys are downwards absolute to $L$, and the existence of a strategic $\omega$-Ramsey cardinal is equiconsistent with the existence of a virtually measurable cardinal. Further, in $L$ the two notions are equivalent.
}

Note also that the proof of Theorem \ref{theo.virtstrat} shows that whenever $\kappa$ is strategic $\omega$-Ramsey then for every regular $\nu>\kappa$ there is a generic extension in which there exists a weakly amenable $H_\nu^V$-normal $H_\nu$-measure on $\kappa$.

\qquad We end this section with a result showing precisely where in the large cardinal hierarchy the strategic $\omega$-Ramsey cardinals and $\omega$-Ramsey cardinals lie, namely that strategic $\omega$-Ramseys are equiconsistent with \textit{remarkables} and $\omega$-Ramseys are strictly below. Theorem 4.8 of \cite{Ramsey2} showed that 2-iterables are limits of remarkables, and our Propositions \ref{prop.tildegame} and \ref{prop.ramseyit} shows that $\omega$-Ramseys are limits of 1-iterables, so that the strategic $\omega$-Ramseys and the $\omega$-Ramseys both lie strictly between the 2-iterables and 1-iterables. It was shown in \cite{HolySchlicht} that $\omega$-Ramseys are consistent with $V=L$.

\qquad Remarkable cardinals were introduced by \cite{remarkable}, and \cite{GitmanSchindler} showed that they are equivalent to virtually supercompact cardinals, and thus by Theorem~\ref{theo.rem} also equivalent to virtually strong cardinals. To maintain consistent terminology we will therefore denote remarkables by simply virtually strongs/supercompacts.

\qquad Combining Corollaries~\ref{coro.strategic} and \ref{coro.strong-meas-equiconsistent} we get the following immediate corollary.

\qcoro[N.-Schindler]{
  Strategic $\omega$-Ramsey cardinals are equiconsistent with virtually strong cardinals.
}

Now, using these results we show that the strategic $\omega$-Ramseys have strictly stronger consistency strength than the $\omega$-Ramseys.

\theo[N.][theo.remlimram]{
	Virtually supercompact cardinals are strategic $\omega$-Ramsey limits of $\omega$-Ramsey cardinals.
}
\proof{
	Let $\kappa$ be virtually supercompact. We can then find a transitive $M$ closed under $2^\kappa$-sequences and a generic elementary embedding $\pi:H_\nu^V\to M$ for some $\nu>2^\kappa$. We will show that $\kappa$ is $\omega$-Ramsey in $M$. Note that remarkables are clearly virtually measurable, and thus by Theorem \ref{theo.virtstrat} also strategic $\omega$-Ramsey; let $\tau_\theta$ be the winning strategy for player II in $\G_\omega^\theta(\kappa)$ for all regular $\theta>\kappa$.

	\qquad In $M$ we fix some regular $\theta>\kappa$ and let $\sigma$ be some strategy for player I in $\G_\omega^\theta(\kappa)^M$. Since $M$ is closed under $2^\kappa$-sequences it means that $\p(\p(\kappa))\subset M$ and thus that $M$ contains all possible filters on $\kappa$. We let player II follow $\tau$, which produces a play $\sigma*\tau$ in which player II wins. But all player II's moves are in $\p(\p(\kappa))$ and hence in $M$, and as $M$ is furthermore closed under $\omega$-sequences, $\sigma*\tau\in M$. This means that $M$ sees that $\sigma$ is not winning, so $\kappa$ is $\omega$-Ramsey in $M$.
	
	\qquad This also implies that $\kappa$ is a limit of $\omega$-Ramseys in $H_\nu$. But as $\kappa$ is virtually supercompact it holds that $H_\kappa\prec_2 V$, in analogy with the same property for strongs and supercompacts, and as being $\omega$-Ramsey is a $\Pi_2$-notion this means that $\kappa$ \textit{is} a limit of $\omega$-Ramseys.
}

This immediately yields the following corollary.

\coro[N.-Schindler]{
	If $\kappa$ is a strategic $\omega$-Ramsey cardinal then
	\eq{
		L_\kappa\models\godel{\text{there is a proper class of $\omega$-Ramseys}}.\tag*{$\dashv$}
	}
}

\subsection{$(\omega,\alpha)$-Ramsey cardinals}

A natural generalisation of the $\gamma$-Ramsey definition is to require more iterability of the last measure. Of course, by Proposition \ref{prop.tildegame} we have that $\G_\gamma(\kappa,\zeta)$ is equivalent to $\G_\gamma(\kappa)$ when $\cof\gamma>\omega$ so the next definition is only interesting whenever $\cof\gamma=\omega$.

\defi[N.]{
	Let $\alpha,\beta$ be ordinals. Then a cardinal $\kappa$ is \textbf{$(\alpha,\beta)$-Ramsey} if player I does not have a winning strategy in $\G_\alpha^\theta(\kappa,\beta)$ for all regular $\theta>\kappa$.\footnote{Note that an $\alpha$-Ramsey cardinal is the same as an $(\alpha,0)$-Ramsey cardinal.}
}

\defi[Gitman]{
	A cardinal $\kappa$ is \textbf{$\alpha$-iterable} if for every $A\subset\kappa$ there exists a \textit{transitive} weak $\kappa$-model $\M$ with $A\in\M$ and an $\alpha$-good $\M$-measure $\mu$ on $\M$.
}

\prop[][prop.ramseyit]{
	If $\beta>0$ then every $(\alpha,\beta)$-Ramsey is a $\beta$-iterable stationary limit of $\beta$-iterables.
}
\proof{
  Fix $\beta>0$, an ordinal $\alpha$ and an $(\alpha,\beta)$-Ramsey cardinal $\kappa$. Fix a subset $A\subset\kappa$ and let $\M_0 := \chull^{H_{\kappa^+}}(\{A,\kappa\}\cup\kappa)$. Now let player I play $\M_0$ and play arbitrary \textit{transitive} legal moves for the rest of the $\G_\alpha^\theta(\kappa,\beta)$ game. Since $\kappa$ is $(\alpha,\beta)$-Ramsey this strategy is not winning for player I, meaning that there exists a play $\vec\mu$ by player II such that player II wins against the strategy.
  
  \qquad Let $\M$ be the final model of this game, and $\mu$ the final measure. By definition $\mu$ is now a $\beta$-good $\M$-measure on $\kappa$. Since $\M$ is transitive by choice of our strategy for player I, this shows that $\kappa$ is $\beta$-iterable.
	
	\qquad That $\kappa$ is $\beta$-iterable is reflected to some $H_\theta$, so let now $(\N,\in,\nu)$ be a result of a play of $\G_\alpha^\theta(\kappa,\beta)$ in which player II won. Then $\N\prec H_\theta$, so that $\kappa$ is also $\beta$-iterable in $\N$. Since being $\beta$-iterable is witnessed by a subset of $\kappa$ and $\beta>0$ implies\footnote{Recall that $\beta$-good for $\beta>0$ in particular implies weak amenability.} that we get a $\kappa$-powerset preserving $j:\N\to\P$, $\P$ also thinks that $\kappa$ is $\beta$-iterable, making $\kappa$ a stationary limit of $\beta$-iterables by elementarity.
}

We now move towards Theorem \ref{theo.upperlimit} which gives an upper consistency bound for the $(\omega,\alpha)$-Ramseys. We first recall a few definitions and a folklore lemma.

\defi{
	For an infinite ordinal $\alpha$, a cardinal $\kappa$ is \textbf{$\alpha$-Erd\H os} for $\alpha\leq\kappa$ if given any club $C\subset\kappa$ and regressive $c:[C]^{<\omega}\to\kappa$ there is a set $H\in[C]^\alpha$ homogeneous for $c$; i.e. that $\abs{c``[H]^n}\leq 1$ holds for every $n<\omega$.
}

\defi{
	\label{defi.remind}
	A set of indiscernibles $I$ for a structure $\M=(M,\in,A)$ is \textbf{remarkable} if $I-\iota$ is a set of indiscernibles for $(M,\in,A,\bra{\xi\mid\xi<\iota})$ for every $\iota\in I$.\footnote{Note that this terminology is not at all related to remarkable \textit{cardinals}.}
}

\lemm[Folklore]{
	\label{lemma}
	Let $\kappa$ be $\alpha$-Erd\H os where $\alpha\in[\omega,\kappa]$ and let $C\subset\kappa$ be club. Then any structure $\M$ in a countable language $\mathcal L$ with $\kappa+1\subset\M$ has a remarkable set of indiscernibles $I\in[C]^\alpha$.
}
\proof{
	Let $\bra{\varphi_n\mid n<\omega}$ enumerate all $\mathcal L$-formulas and define $c:[C]^{<\omega}\to\kappa$ as follows. For an increasing sequence $\alpha_1<\cdots<\alpha_{2n}\in C$ let
	\eq{
		c(\{\alpha_1,\hdots,\alpha_{2n}\}):=&\text{ the least }\lambda<\alpha_1\text{ such that }\\
    &\exists\delta_1<\cdots\delta_k\exists m<\omega:\lambda=\bra{m,\delta_1,\hdots,\delta_k}\land\\
		&\M\not\models\varphi_m[\vec\delta,\alpha_1,\hdots,\alpha_n]\leftrightarrow\varphi_m[\vec\delta,\alpha_{n+1},\hdots,\alpha_{2n}]
	}

	if such a $\lambda$ exists, and $c(s)=0$ otherwise. Clearly $c$ is regressive, so since $\kappa$ is $\alpha$-Erd\H os we get a homogeneous $I\in[C]^\alpha$ for $c$; i.e. that $\abs{c``[I]^n}\leq 1$ for every $n<\omega$. Then $c(\{\alpha_1,\hdots,\alpha_{2n}\})=0$ for every $\alpha_1,\hdots,\alpha_{2n}\in I$, as otherwise there exists an $m<\omega$ and $\delta_1<\cdots\delta_k$ such that for any $\alpha_1<\hdots<\alpha_{2n}\in I$,
	\eq{
		\M\not\models\varphi_m[\vec\delta,\alpha_1,\hdots,\alpha_n]\leftrightarrow\varphi_m[\vec\delta,\alpha_{n+1},\hdots,\alpha_{2n}].\tag*{$(\dagger)$}
	}
	
	But then simply pick $\alpha_1<\hdots\alpha_{2n}<\alpha_1'<\cdots<\alpha_{2n}'$ so that both $\{\alpha_1,\hdots,\alpha_{2n}\}$ and $\{\alpha_1',\hdots,\alpha_{2n}'\}$ witnesses $(\dagger)$; then either $\{\alpha_1,\hdots,\alpha_n,\alpha_1',\alpha_n'\}$ or \\$\{\alpha_1,\hdots,\alpha_n,\alpha_{n+1}',\hdots,\alpha_{2n}'\}$ also witnesses that $(\dagger)$ fails, $\contr$.
}

\theo[N.][theo.upperlimit]{
	Let $\alpha\in[\omega,\omega_1]$ be additively closed. Then any $\alpha$-Erd\H os cardinal is a limit of $(\omega,\alpha)$-Ramsey cardinals.
}
\proof{
	Let $\kappa$ be $\alpha$-Erd\H os, $\theta>\kappa$ a regular cardinal and $\beta<\kappa$ any ordinal. Use the above Lemma \ref{lemma} to get a set of remarkable indiscernibles $I\in[\kappa]^\alpha$ for the structure $(H_\theta,\in,\bra{\xi\mid\xi<\beta})$, and let $\iota\in I$ be the least indiscernible in $I$. We will show that player I has no winning strategy in $\G_\omega^\theta(\iota,\alpha)$, so by the proof of Theorem 5.5(d) in \cite{HolySchlicht} it suffices to find a weak $\iota$-model $\M\prec H_\theta$ and an $\alpha$-good $\M$-measure on $\iota$. Define
	\eq{
		\M:=\hull^{H_\theta}(\iota\cup I)\prec H_\theta
	}
	
	and let $\pi:I\to I$ be the right-shift map. Since $I$ is remarkable, $I$ ($=I-\iota$) is a set of indiscernibles for the structure $(H_\theta,\in,\bra{\xi\mid\xi<\iota})$, so that $\pi$ induces an elementary embedding $j:\M\to\M$ with $\crit j=\iota$, given as
	\eq{
		j(\tau^{\M}[\vec\xi,\iota_{i_0},\hdots,\iota_{i_k}]):=\tau^{\M}[\vec\xi,\iota_{i_0+1},\hdots,\iota_{i_k+1}],
	}

	with $\vec\xi\subset\iota$. Since $j$ is trivially $\iota$-powerset preserving we get that $\M\prec H_\theta$ is a weak $\iota$-model satisfying $\zfc^-$ with a 1-good $\M$-measure $\mu_j$ on $\iota$. Furthermore, as we can linearly iterate $\M$ simply by applying $j$ we get an $\alpha$-iteration of $\M$ since there are $\alpha$-many indiscernibles. Note that at limit stages $\gamma<\alpha$ our iteration sends $\tau^{\M}[\vec\xi,\iota_{i_0},\hdots,\iota_{i_k}]$ to $\tau^{\M}[\vec\xi,\iota_{i_0+\gamma},\hdots,\iota_{i_k+\gamma}]$ so here we are using that $\alpha$ is additively closed.
	
	\qquad This shows that player I has no winning strategy in $\G_\omega^\theta(\iota,\alpha)$. Since $\iota>\beta$ and $\beta<\kappa$ was arbitrary, $\kappa$ is a limit of $\eta$ such that player I has no winning strategy in $\G_\omega^\theta(\eta,\alpha)$. If we repeat this procedure for all regular $\theta>\kappa$ we get by the pidgeon hole principle that $\kappa$ is a limit of $(\omega,\alpha)$-Ramsey cardinals.
}

As Theorem 4.5 in \cite{GitmanSchindler} shows that $(\alpha{+}1)$-iterable cardinals have $\alpha$-Erd\H os cardinals below them for $\alpha\geq\omega$ additively closed, this shows that the $(\omega,\alpha)$-Ramseys form a strict hierarchy. Further, as $\alpha$-Erd\H os cardinals are consistent with $V=L$ when $\alpha<\omega_1^L$ and $\omega_1$-iterable cardinals are not consistent with $V=L$, we also get that $(\omega,\alpha)$-Ramsey cardinals are consistent with $V=L$ if $\alpha<\omega_1^L$ and that they are not if $\alpha=\omega_1$.


\subsection{[Strategic] $(\omega{+}1)$-Ramsey cardinals}

The next step is then to consider $(\omega{+}1)$-Ramseys, which turn out to cause a considerable jump in consistency strength. We first need the following result which is implicit in \cite{Mitchell} and in the proof of Lemma 1.3 in \cite{Kapplications} --- see also \cite{Dodd} and \cite{Ramsey1}.

\qtheo[Jensen, Mitchell]{
	\label{theo.ramseycondition}
	A cardinal $\kappa$ is Ramsey if and only if every $A\subset\kappa$ is an element of a weak $\kappa$-model $\M$ such that there exists a weakly amenable countably complete $\M$-measure on $\kappa$.
}

The following theorem then supplies us with a lower bound for the strength of the $(\omega{+}1)$-Ramsey cardinals. It should be noted that a better lower bound will be shown in Theorem \ref{theo.plusone}, but we include this Ramsey lower bound as well for completeness.

\theo[N.][theo.ramlimram]{
	Every $(\omega{+}1)$-Ramsey cardinal is a Ramsey limit of Ramseys.
}
\proof{
	Let $\kappa$ be $(\omega{+}1)$-Ramsey and $A\subset\kappa$. Let $\sigma$ be a strategy for player I in $\G_{\omega+1}^{\kappa^+}(\kappa)$ satisfying that whenever $\vec\M_\alpha*\vec\mu_\alpha$ is consistent with $\sigma$ it holds that $A\in\M_0$ and $\mu_\alpha\in\M_{\alpha+1}$ for all $\alpha\leq\omega$. Then $\sigma$ is not winning as $\kappa$ is $(\omega{+}1)$-Ramsey, so we may fix a play $\sigma*\vec\mu_\alpha$ of $\G_{\omega+1}^{\kappa^+}(\kappa)$ in which player II wins. Then by the choice of $\sigma$ we get that $\mu_\omega$ is a weakly amenable $\M_\omega$-measure on $\kappa$, and by the rules of $\G_{\omega+1}^{\kappa^+}(\kappa)$ it is also countably complete (it is even normal), which makes $\kappa$ Ramsey by the above Theorem \ref{theo.ramseycondition}.

	\qquad Since $\kappa$ is Ramsey, $\M_\omega\models\godel{\kappa\text{ is Ramsey}}$ as well. Letting $j:\M_\omega\to\N$ be the $\kappa$-powerset preserving embedding induced by $\mu_\omega$, we also get that $\N\models\godel{\kappa\text{ is Ramsey}}$ by $\kappa$-powerset preservation. This then implies that $\kappa$ \textit{is} a stationary limit of Ramsey cardinals inside $\M_\omega$, and thus also in $V$ by elementarity.
}

As for the \textit{consistency} strength of the strategic $(\omega{+}1)$-Ramsey cardinals, we get the following result that they reach a measurable cardinal. The proof of the following is closely related to the proof due to Silver and Solovay that player II having a winning strategy in the \textit{cut and choose game} is equiconsistent with a measurable cardinal --- see e.g. p. 249 in \cite{c&c}.

\theo[N.]{
	\label{theo.omegaplusone}
	If $\kappa$ is a strategic $(\omega{+}1)$-Ramsey cardinal then, in $V^{\col(\omega,2^\kappa)}$, there is a transitive class $N$ and an elementary embedding $j:V\to N$ with $\crit j=\kappa$. In particular, the existence of a strategic $(\omega{+}1)$-Ramsey cardinal is equiconsistent with the existence of a measurable cardinal.
}
\proof{
	Set $\mathbb P:=\col(\omega,2^\kappa)$ and let $\sigma$ be player II's winning strategy in $\G_{\omega+1}^{\kappa^+}(\kappa)$. Let $\dot\M$ be a $\mathbb P$-name of an $\omega$-sequence $\bra{\M_n\mid n<\omega}$ of weak $\kappa$-models $\M_n\in V$ such that $\M_n\prec H_{\kappa^+}^V$ and $\p(\kappa)^V\subset\bigcup_{n<\omega}\M_n$, and let $\dot\mu$ be a $\mathbb P$-name for the $\omega$-sequence of $\sigma$-responses to the $\M_n$'s in $\G_{\omega+1}^{\kappa^+}(\kappa)^V$.

	\qquad Assume that there is a $\mathbb P$-condition $p$ which forces the generic ultrapower $\ult(V,\bigcup_n\dot\mu_n)$ to be illfounded, meaning that we can fix a $\mathbb P$-name $\dot f$ for an $\omega$-sequence $\bra{f_n\mid n<\omega}$ such that
	\eq{
		p\forces\dot X_n:=\{\alpha<\kappa\mid\dot f_{n+1}(\alpha)<\dot f_n(\alpha)\}\in\bigcup_{n<\omega}\dot\mu_n.
	}

	Now, in $V$, we fix some large regular $\theta\gg\kappa$ and a countable $\N\prec H_\theta$ such that $\dot\M,\dot\mu,\dot f,H_{\kappa^+}^V,\sigma,p\in\N$. We can find an $\N$-generic $g\subset\mathbb P^{\N}$ in $V$ with $p\in g$ since $\N$ is countable, so that $\N[g]\in V$. But the play $\dot\M^g_n*\dot\mu^g_n$ is a play of $\G_\omega^{\kappa^+}(\kappa)^V$ which is according to $\sigma$, meaning that $\bigcup_{n<\omega}\dot\mu^g_n$ is normal and in particular countably complete (in $V$). Then $\bigcap_{n<\omega}\dot X_n^g\neq\emptyset$, but if $\alpha\in\bigcap_{n<\omega}\dot X_n^g$ then $\bra{\dot f^g_n(\alpha)\mid n<\omega}$ is a strictly decreasing $\omega$-sequence of ordinals, $\contr$. This means that $\ult(V,\bigcup_n\mu_n)$ is indeed well-founded.

	\qquad This conclusion is well-known to imply that $\kappa$ is a measurable in an inner model; see e.g. Lemma 4.2 in \cite{KellnerShelah}.
}

The above Theorem \ref{theo.omegaplusone} then answers Question 9.2 in \cite{HolySchlicht} in the negative, asking if $\lambda$-Ramseys are strategic $\lambda$-Ramseys for uncountable cardinals $\lambda$, as well as answering Question 9.7 from the same paper in the positive, asking whether strategic fully Ramseys are equiconsistent with a measurable.


\section{The general case}

\subsection{Gitman's cardinals}

In this subsection we define the strongly- and super Ramsey cardinals from \cite{Ramsey1} and investigate further connections between these and the $\alpha$-Ramsey cardinals. First, a definition.

\defi[Gitman]{
	A cardinal $\kappa$ is \textbf{strongly Ramsey} if every $A\subset\kappa$ is an element of a transitive $\kappa$-model $\M$ with a weakly amenable $\M$-normal $\M$-measure $\mu$ on $\kappa$. If furthermore $\M\prec H_{\kappa^+}$ then we say that $\kappa$ is \textbf{super Ramsey}.
}

Note that since the model $\M$ in question is a \textit{$\kappa$-model} it is closed under countable sequences, so that the measure $\mu$ is automatically countably complete. The definition of the strongly Ramseys is thus exactly the same as the characterisation of Ramsey cardinals, with the added condition that the model is closed under ${<}\kappa$-sequences. \cite{Ramsey1} shows that every super Ramsey cardinal is a strongly Ramsey limit of strongly Ramsey cardinals, and that $\kappa$ is strongly Ramsey iff every $A\subset\kappa$ is an element of a transitive $\kappa$-model $\M\models\zfc$ with a weakly amenable $\M$-normal $\M$-measure $\mu$ on $\kappa$.

\qquad Now, a first connection between the $\alpha$-Ramseys and the strongly- and super Ramseys is the result in \cite{HolySchlicht} that fully Ramsey cardinals are super Ramsey limits of super Ramseys. The following result then shows that the strongly- and super Ramseys are sandwiched between the almost fully Ramseys and the fully Ramseys.

\theo[N.-Welch]{
  Every strongly Ramsey cardinal is a stationary limit of almost fully Ramseys.
}
\proof{
  Let $\kappa$ be strongly Ramsey and let $\M\models\zfc$ be a transitive $\kappa$-model with $V_\kappa\in\M$ and $\mu$ a weakly amenable $\M$-normal $\M$-measure. Let $\gamma<\kappa$ have uncountable cofinality and $\sigma\in\M$ a strategy for player I in $\G_\gamma(\kappa)^{\M}$. Now, whenever player I plays $\M_\alpha\in\M$ let player II play $\mu\cap\M_\alpha$, which is an element of $\M$ by weak amenability of $\mu$. As $\M^{<\kappa}\subset\M$ the resulting play is inside $\M$, so $\M$ sees that $\sigma$ is not winning.

	\qquad Now, letting $j_\mu:\M\to\N$ be the induced embedding, $\kappa$-powerset preservation of $j_\mu$ implies that $\mu$ is also a weakly amenable $\N$-normal $\N$-measure on $\kappa$. This means that we can copy the above argument to ensure that $\kappa$ is also almost fully Ramsey in $\N$, entailing that it is a stationary limit of almost fully Ramseys in $\M$. But note now that $\lambda$ is almost fully Ramsey iff it is almost fully Ramsey in a transitive $\zfc$-model containing $H_{(2^\lambda)^+}$ as an element by Theorem 5.5(e) in \cite{HolySchlicht}, so that $\kappa$ being inaccessible, $V_\kappa\in\M$ and $\M$ being transitive implies that $\kappa$ really \textit{is} a stationary limit of almost fully Ramseys.
}


\subsection{Downwards absoluteness to $K$}

Lastly, we consider the question of whether the $\alpha$-Ramseys are downwards absolute to $K$, which turns out to at least be true in many cases. The below Theorem \ref{theo.downK} then also answers Question 9.4 from \cite{HolySchlicht} in the positive, asking whether $\alpha$-Ramseys are downwards absolute to the Dodd-Jensen core model for $\alpha\in[\omega,\kappa]$ a cardinal. We first recall the definition of $0^\pistol$.

\defi{
	$0^\pistol$ is ``the sharp for a strong cardinal'', meaning the minimal sound active mouse $\M$ with $\M\l\crit(\dot F^{\M})\models\godel{\text{There exists a strong cardinal}}$, with $\dot F^{\M}$ being the top extender of $\M$.
}

\theo[N.-Welch]{
	\label{theo.downK}
	Assume $0^\pistol$ does not exist. Let $\lambda$ be a limit ordinal with uncountable cofinality and let $\kappa$ be $\lambda$-Ramsey. Then $K\models\godel{\kappa\text{ is a $\lambda$-Ramsey cardinal}}$.
}
\proof{
	Note first that $\kappa^{+K}=\kappa^+$ by \cite{SchindlerCovering}, since $\kappa$ in particular is weakly compact.	Let $\sigma\in K$ be a strategy for player I in $\G_\lambda^{\kappa^+}(\kappa)^K$, so that a play following $\sigma$ will produce weak $\kappa$-models $\M\prec K\l\kappa^+$. We can then define a strategy $\tilde\sigma$ for player I in $\G_\lambda^{\kappa^+}(\kappa)$ as follows. Firstly let $\tilde\sigma(\emptyset):=\hull^{H_{\kappa^+}}(K\l\kappa\cup\sigma(\emptyset))$. Assuming now that $\bra{\tilde\M_\alpha,\tilde\mu_\alpha\mid\alpha<\gamma}$ is a partial play of $\G_\lambda^{\kappa^+}(\kappa)$ which is consistent with $\tilde\sigma$, we have two cases. If $\tilde\mu_\alpha\in K$ for every $\alpha<\gamma$ then let $\bra{\M_\alpha\mid\alpha<\gamma}$ be the corresponding models played in $\G_\lambda^{\kappa^+}(\kappa)^K$ from which the $\tilde\M_\alpha$'s are derived and let
	\eq{
		\tilde\sigma(\bra{\tilde\M_\alpha,\tilde\mu_\alpha\mid\alpha<\gamma}):=\hull^{H_{\kappa^+}}(K\l\kappa\cup\sigma(\bra{\M_\alpha,\tilde\mu_\alpha\mid\alpha<\gamma})),
	}

	and otherwise let $\tilde\sigma$ play arbitrarily. As $\kappa$ is $\lambda$-Ramsey (in $V$) there exists a play $\bra{\tilde\M_\alpha,\tilde\mu_\alpha\mid\alpha\leq\lambda}$ of $\G_\lambda^{\kappa^+}(\kappa)$ which is consistent with $\tilde\sigma$ in which player II won. Note that $\tilde\M_\lambda\cap K\l\kappa^+\prec K\l\kappa^+$ so let $\N$ be the transitive collapse of $\tilde\M_\lambda\cap K\l\kappa^+$. But if $j:\N\to K\l\kappa^+$ is the uncollapse then $\crit j$ is both an $\N$-cardinal and also $>\kappa$ because we ensured that $K\l\kappa\subset\N$. This means that $j=\id$ because $\kappa$ is the largest $\N$-cardinal by elementarity in $K\l\kappa^+$, so that $\tilde\M_\lambda\cap K\l\kappa^+=\N$ is a transitive elementary substructure of $K\l\kappa^+$, making it an initial segment of $K$.
	
	\qquad Now, since $\mu:=\tilde\mu_\lambda$ is a countably complete weakly amenable $K\l o(\N)$-measure\footnote{Here we use that $\N\pinit K$.}, the ``beaver argument''\footnote{See Appendix \ref{prelims.core-model-theory} for details regarding the beaver argument.} shows that $\mu\in K$, so that we can then define a strategy $\tau$ for player II in $\G_\lambda^{\kappa^+}(\kappa)^K$ as simply playing $\mu\cap\N\in K$ whenever player I plays $\N$. Since $\mu=\tilde\mu_\lambda$ we also have that $\mu\cap\M_\alpha=\tilde\mu_\alpha\cap\M_\alpha$, so that $\sigma$ will eventually play $\N$, making $\tau$ win against $\sigma$.\footnote{Note that $\tau$ is not necessarily a winning strategy --- all we know is that it is winning against this particular strategy $\sigma$.}
}

Note that the only thing we used $\cof\lambda>\omega$ for in the above proof was to ensure that $\mu$ was countably complete. If now $\kappa$ instead was either genuine- or normal $\alpha$-Ramsey for any limit ordinal $\alpha$ then $\mu_\alpha$ would also be countably complete and weakly amenable, so the same proof shows the following.

\qcoro[N.-Welch]{
	\label{coro.downK}
	Assume $0^\pistol$ does not exist and let $\alpha$ be any limit ordinal. Then every genuine- and every normal $\alpha$-Ramsey cardinal is downwards absolute to $K$. In particular, if $\alpha$ is a limit of limit ordinals then every ${<}\alpha$-Ramsey cardinal is downwards absolute to $K$ as well.
}


\subsection{Indiscernible games}

We now move to the strategic versions of the $\alpha$-Ramsey hierarchy. The first thing we want to do is define \textit{$\alpha$-very Ramsey cardinals}, introduced in \cite{SharpeWelch}, and show the tight connection between these and the strategic $\alpha$-Ramseys. We need a few more definitions. Recall the definition of a remarkable set of indiscernibles from Definition \ref{defi.remind}.

\defi{
	A \textbf{good set of indiscernibles} for a structure $\M$ is a set $I\subset\M$ of remarkable indiscernibles for $\M$ such that $\M\l\iota\prec\M$ for any $\iota\in I$.
}

A key result of \cite{SharpeWelch} involves a translation procedure between indiscernibles and measures. The following result is translating from a measure to indiscernibles and is a special case of \cite[Lemma 2.9]{SharpeWelch}.

\qtheo[Sharpe-Welch][theo.sharpe-welch-m2i]{
  Let $\kappa$ be an uncountable cardinal, $\alpha>\kappa$, and fix $A\subset\alpha$ and $a\subset\kappa$. Let $\M := \bra{J_\alpha[A],\in,A}$ and $m := \bra{J_\kappa[a],\in,a}$. Assume the following:
  \begin{enumerate}
    \item $m\in\M$;
    \item $\M$ and $m$ are both amenable structures;
    \item $\M$ is a weak $\kappa$-model with $\kappa$ being its largest cardinal;
    \item There exists a countably complete weakly amenable $\M$-measure $\mu$ on $\kappa$.
  \end{enumerate}

  Then there exists a good set of indiscernibles $I\in\mu$ for $m$.
}

The converse translation procedure, going from a good set of indiscernibles to a measure, is then the following, which is part of \cite[Lemma 2.18]{SharpeWelch}.

\qtheo[Sharpe-Welch][theo.sharpe-welch-i2m]{
  Let $\kappa$ be an uncountable cardinal, $A\subset\kappa$ and define $m:=\bra{J_\kappa[a],\in,a}$. Assume that
  \begin{enumerate}
    \item $m$ is amenable;
    \item $m\models\zfc$;
    \item There exists a good set of indiscernibles $I\in[\kappa]^\kappa$ for $m$.
  \end{enumerate}

  Then there exists an ordinal $\alpha>\kappa$ and $A\subset\alpha$ such that $\M := \bra{J_\alpha[A],\in,\A}$ is an amenable weak $\kappa$-model with $\kappa$ being its largest cardinal, $m\in\M$ and there exists a countably complete weakly amenable $\M$-measure $\mu$ on $\kappa$.
}

\defi[Sharpe-Welch]{
  Define the \textbf{indiscernible game} $G^I_\gamma(\kappa)$ in $\gamma$ many rounds as follows\footnote{This game is denoted by $G_r(\kappa,\gamma)$ in \cite{SharpeWelch}.}
	\game{\M_0}{I_0}{\M_1}{I_1}{\M_2}{I_2}{\cdots}{\cdots}

	Here $\M_\alpha$ is an amenable structure of the form $(J_\kappa[A],\in,A)$ for some $A\subset\kappa$, $I_\alpha\in[\kappa]^\kappa$ is a good set of indiscernibles for $\M_\alpha$ and the $I_\alpha$'s are $\subset$-decreasing. Player II wins iff they can continue playing through all the rounds.
}

\defi[Sharpe-Welch]{
	A cardinal $\kappa$ is \textbf{$\gamma$-very Ramsey} if player II has a winning strategy in the game $G^I_\gamma(\kappa)$.
}

The next couple of results concerns the connection between the strategic $\alpha$-Ramseys and the $\alpha$-very Ramseys. We start with the following.

\theo[N.][theo.plusone]{
	Every $(\omega{+}1)$-Ramsey is an $\omega$-very Ramsey stationary limit of $\omega$-very Ramseys.
}
\proof{
	Let $\kappa$ be $(\omega{+}1)$-Ramsey. We will describe a winning strategy for player II in the indiscernible game $G_\omega^I(\kappa)$. If player I plays $\M_0=(J_\kappa[A_0],\in,A_0)$ in $G_\omega^I(\kappa)$ then let player I in $\G_{\omega+1}^{\kappa^+}(\kappa)$ play
	\eq{
		\h_0:=\hull^{H_{\kappa^+}}(J_\kappa[A_0]\cup\{\M_0,\kappa,A_0\})\prec H_{\kappa^+}.
	}

	Let player I now follow a strategy in $\G_{\omega+1}^{\kappa^+}(\kappa)$ which starts off with $\h_0$ and ensures that, whenever $\vec\M_\alpha*\vec\mu_\alpha$ is consistent with player I's strategy, then $\mu_\alpha\in\M_{\alpha+1}$ for all $\alpha\leq\omega$. Since player II is not losing in $\G_{\omega+1}^{\kappa^+}(\kappa)$ there is a play $\vec\M_\alpha*\vec\mu_\alpha$ in which player I follows this strategy just described and where player II wins -- write $\h_0^{(\alpha)}:=\M_\alpha$ and $\mu_0^{(\alpha)}:=\mu_\alpha$ for the models and measures in this play.
	\game{\h_0^{(0)}}{\mu_0^{(0)}}{\cdots}{\cdots}{\h_0^{(\omega)}}{\mu_0^{(\omega)}}{\h_0^{(\omega+1)}}{\mu_0^{(\omega+1)}}

  By the choice of player I's strategy we get that $\mu_0^{(\omega)}$ is both weakly amenable, and it is also countably complete by the rules of $\G_{\omega+1}^{\kappa^+}(\kappa)$ (it is even normal). Now Theorem \ref{theo.sharpe-welch-m2i} gives us a set of good indiscernibles $I_0\in\mu_0^{(\omega)}$ for $\M_0$, as $\M_0\in\h_0^{(\omega)}$ and $\mu_0^{(\omega)}$ is a countably complete weakly amenable $\h_0^{(\omega)}$-normal $\h_0^{(\omega)}$-measure on $\kappa$. Let player II play $I_0$ in $G_\omega^I(\kappa)$. Let now $\M_1=(J_\kappa[A_1],\in,A_1)$ be the next play by player I in $G_\omega^I(\kappa)$.
	\game{\M_0}{I_0}{\M_1}{}{}{}{}{}

	Since $\mu_0^{(\omega)}=\bigcup_n\mu_0^{(n)}$ we must have that $I_0\in\mu_0^{(n_0)}$ for some $n_0<\omega$. In the $(n_0{+}1)$'st round of $\G_{\omega+1}^{\kappa^+}(\kappa)$ we change player I's strategy and let player I play
	\eq{
		\h_1:=\hull^{H_{\kappa^+}}(J_\kappa[A_0]\cup\{\M_0,\M_1,\kappa,A_0,A_1,\bra{\h_0^{(k)},\mu_0^{(k)}\mid k\leq n_0}\})\prec H_{\kappa^+}
	}

	and otherwise continues following some strategy, as long as the measures played by player II keep being elements of the following models.	Our play of the game $\G_{\omega+1}^{\kappa^+}(\kappa)$ thus looks like the following so far.

	\game{\h_0^{(0)}}{\mu_0^{(0)}}{\cdots}{\cdots}{\h_0^{(n_0)}}{\mu_0^{(n_0)}}{\h_1}{}

	Now player II in $\G_{\omega+1}^{\kappa^+}(\kappa)$ is not losing at round $n_0$, so there is a play extending the above in which player I follows their revised strategy and in which player II wins. As before we get a set $I_1'\in\mu_1^{(n_1)}$ of good indiscernibles for $\M_1$, where $n_1<\omega$. Since $I_0\in\mu_0^{(n_0)}\subset\mu_1^{(n_1)}$ we can let player II in $G_\omega^I(\kappa)$ play $I_1:=I_0\cap I_1'\in\mu_1^{(n_1)}$. Continuing like this, player II can keep playing throughout all $\omega$ rounds of $G_\omega^I(\kappa)$, making $\kappa$ $\omega$-very Ramsey.

	\qquad As for showing that $\kappa$ is a stationary limit of $\omega$-very Ramseys, let $\M\prec H_{\kappa^+}$ be a weak $\kappa$-model with a weakly amenable countably complete $\M$-normal $\M$-measure $\mu$ on $\kappa$, which exists by Theorem \ref{theo.ramlimram} as $\kappa$ is $(\omega{+}1)$-Ramsey. Then by elementarity $\M\models\godel{\kappa\text{ is $\omega$-very Ramsey}}$ and since $\kappa$ being $\omega$-very Ramsey is absolute between structures having the same subsets of $\kappa$ it also holds in the $\mu$-ultrapower, meaning that $\kappa$ is a stationary limit of $\omega$-very Ramseys by elementarity. 
}

The above proof technique can be generalised to the following.

\theo[N.]{
	\label{theo.stratvery}
	For limit ordinals $\alpha$, every coherent ${<}\omega\alpha$-Ramsey is $\omega\alpha$-very Ramsey.
}
\proof{
	This is basically the same proof as the proof of Theorem \ref{theo.plusone}. We do the ``going-back'' trick in $\omega$-chunks, and at limit stages we continue our non-losing strategy in $\G_{\omega\alpha}^{\kappa^+}(\kappa)$ by using our winning strategy, which we have available as we are assuming coherent ${<}\omega\alpha$-Ramseyness. We need $\alpha$ to be a limit ordinal for this to work, as otherwise we would be in trouble in the last $\omega$-chunk, as we cannot just extend the play to get a countably complete measure, which we need to use the proof of Theorem \ref{theo.plusone}.
}

As for going from the $\alpha$-very Ramseys to the strategic $\alpha$-Ramseys we got the following.

\theo[N.]{
	\label{theo.succverystrat}
	For $\gamma$ any ordinal, every coherent ${<}\gamma$-very Ramsey\footnote{Here the coherency again just means that the winning strategies $\sigma_\alpha$ for player II in $G_\alpha^I(\kappa)$ are $\subset$-increasing.} is coherent ${<}\gamma$-Ramsey.\footnote{Here a ``coherent ${<}\gamma$-very Ramsey cardinal'' is defined from $\gamma$-very Ramseys in the same way as coherent ${<}\gamma$-Ramsey cardinals is defined from $\gamma$-Ramseys. When $\gamma$ is a limit ordinal then coherent ${<}\gamma$-very Ramseys are precisely the same as $\gamma$-very Ramseys, so this is solely to ``subtract one'' when $\gamma$ is a successor ordinal --- i.e. a coherent ${<}(\gamma+1)$-very Ramsey cardinal is the same thing as a $\gamma$-very Ramsey cardinal.}
}
\proof{
	The reason why we work with ${<}\gamma$-Ramseys here is to ensure that player II only has to satisfy a closed game condition (i.e. to continue playing throughout all the rounds). If $\gamma=\beta+1$ then set $\zeta:=\beta$ and otherwise let $\zeta:=\gamma$. Let $\kappa$ be $\zeta$-very Ramsey and let $\tau$ be a winning strategy for player II in $G_\zeta^I(\kappa)$. Let $\M_\alpha\prec H_\theta$ be any move by player I in the $\alpha$'th round of $\G_\zeta(\kappa)$. Let $A_\alpha\subset\kappa$ encode all subsets of $\kappa$ in $\M_\alpha$ and form now
	\eq{
		\N_\alpha:=(J_\kappa[A_\alpha],\in,A_\alpha),
	}

  which is a legal move for player I in $G_\zeta^I(\kappa)$, yielding a good set of indiscernibles $I_\alpha\in[\kappa]^\kappa$ for $\N_\alpha$ such that $I_\alpha\subset I_\beta$ for every $\beta<\alpha$. Now by Theorem \ref{theo.sharpe-welch-i2m} we get a structure $\P_\alpha$ with $\N_\alpha\in\P_\alpha$ and a $\P_\alpha$-measure $\tilde\mu_\alpha$ on $\kappa$, generated by $I_\alpha$.\footnote{By \textit{generated} here we mean that $X\in\tilde\mu_\alpha$ iff $X$ contains a tail of indiscernibles from $I_\alpha$.} Set $\mu_\alpha:=\tilde\mu_\alpha\cap\M_\alpha$ and let player II play $\mu_\alpha$ in $\G_\zeta(\kappa)$.

	\qquad As the $\mu_\alpha$'s are generated by the $I_\alpha$'s, the $\mu_\alpha$'s are $\subset$-increasing. We have thus created a strategy for player II in $\G_\zeta(\kappa)$ which does not lose at any round $\alpha<\gamma$, making $\kappa$ coherent ${<}\gamma$-Ramsey.
}

The following result is then a direct corollary of Theorems \ref{theo.stratvery} and \ref{theo.succverystrat}.

\qcoro[N.]{
	\label{coro.verystrat}
	For limit ordinals $\alpha$, $\kappa$ is $\omega\alpha$-very Ramsey iff it is coherent ${<}\omega\alpha$-Ramsey. In particular, $\kappa$ is $\lambda$-very Ramsey iff it is strategic $\lambda$-Ramsey for any $\lambda$ with uncountable cofinality.
}

We can now use this equivalence to transfer results from the $\alpha$-very Ramseys over to the strategic versions. The \textit{completely Ramsey cardinals} are the cardinals topping the hierarchy defined in \cite{Feng}. A completely Ramsey cardinal implies the consistency of a Ramsey cardinal, see e.g. Theorem 3.51 in \cite{SharpeWelch}. We are going to use the following characterisation of the completely Ramsey cardinals, which is Lemma 3.49 in \cite{SharpeWelch}.

\qtheo[Sharpe-Welch]{
	\label{theo.comvery}
	A cardinal is completely Ramsey if and only if it is $\omega$-very Ramsey.
}

This, together with Theorem \ref{theo.plusone}, immediately yields the following strengthening of Theorem \ref{theo.ramlimram}.

\qcoro[N.]{
	Every $(\omega{+}1)$-Ramsey cardinal is a completely Ramsey stationary limit of completely Ramsey cardinals.
}

The above Theorem \ref{theo.succverystrat} also yields the following consequence.

\coro[N.]{
	Every completely Ramsey cardinal is completely ineffable.
}
\proof{
	From Theorem \ref{theo.comvery} we have that being completely Ramsey is equivalent to being $\omega$-very Ramsey, so the above Theorem \ref{theo.succverystrat} then yields that a completely Ramsey cardinal is coherent ${<}\omega$-Ramsey, which we saw in Theorem \ref{theo.ineff} is equivalent to being completely ineffable.
}

Now, moving to the uncountable case, Corollary \ref{coro.verystrat} yields that strategic $\omega_1$-Ramsey cardinals are $\omega_1$-very Ramsey, and Theorem 3.50 in \cite{SharpeWelch} states that $\omega_1$-very Ramseys are measurable in the core model $K$, assuming $0^\pistol$ does not exist, which then shows the following theorem. We also include the original direct proof of that theorem, due to Welch.

\theo[Welch]{
	\label{theo.stratramsey}
	Assuming $0^\pistol$ does not exist, every strategic $\omega_1$-Ramsey cardinal is measurable in $K$.
}
\proof{
	Let $\kappa$ be strategic $\omega_1$-Ramsey, say $\tau$ is the winning strategy for player II in $\G_{\omega_1}(\kappa)$. Jump to $V[g]$, where $g\subset\col(\omega_1,\kappa^+)$ is $V$-generic. Since $\col(\omega_1,\kappa^+)$ is $\omega$-closed, $V$ and $V[g]$ have the same countable sequences of $V$, so $\tau$ is still a strategy for player II in $\G_{\omega_1}(\kappa)^{V[g]}$, as long as player I only plays elements of $V$.
	
	\qquad Now let $\bra{\kappa_\alpha\mid\alpha<\omega_1}$ be an increasing sequence of regular $K$-cardinals cofinal in $\kappa^+$, let player I in $\G_{\omega_1}(\kappa)$ play $\M_\alpha:=\hull^{H_\theta}(K\l\kappa_\alpha)\prec H_\theta$ and player II follow $\tau$. This results in a countably complete weakly amenable $K$-measure $\mu_{\omega_1}$, which the ``beaver argument''\footnote{See Appendix \ref{prelims.core-model-theory} for details regarding the beaver argument.} then shows is actually an element of $K$, making $\kappa$ measurable in $K$.
}

A natural question is whether this behaviour persists when going to larger core models. It turns out that the answer is affirmative: every strategic $\omega_1$-Ramsey cardinal is also measurable in Steel's core model below a Woodin\footnote{See Appendix \ref{prelims.core-model-theory}.}, a result due to Schindler which we include with his permission here. We will need the following special case of Corollary 3.1 from \cite{SchindlerIterates}.\footnote{That paper assumes the existence of a measurable as well, but by \cite{JensenSteel} we can omit that here.}

\qtheo[Schindler]{
  \label{theo.Schindler}
  Assume that there exists no inner model with a Woodin cardinal, let $\mu$ be a measure on a cardinal $\kappa$, and let $\pi:V\to\ult(V,\mu)\cong N$ be the ultrapower embedding. Assume that $N$ is closed under countable sequences. Write $K^N$ for the core model constructed inside $N$. Then $K^N$ is a normal iterate of $K$, i.e. there is a normal iteration tree $\T$ on $K$ of successor length such that $\M_{\infty}^{\T}=K^N$. Moreover, we have that $\pi_{0\infty}^{\T}=\pi\restr K$.
}

\theo[Schindler]{
	\label{theo.Schindlerstrat}
  Assuming there exists no inner model with a Woodin cardinal, every strategic $\omega_1$-Ramsey cardinal is measurable in $K$.
}
\proof{
	Fix a large regular $\theta\gg 2^\kappa$. Let $\kappa$ be strategic $\omega_1$-Ramsey and fix a winning strategy $\sigma$ for player II in $\G_{\omega_1}(\kappa)$. Let $g\subset\col(\omega_1,2^\kappa)$ be $V$-generic and in $V[g]$ fix an elementary chain $\bra{M_\alpha\mid\alpha<\omega_1}$ of weak $\kappa$-models $M_\alpha\prec H_\theta^V$ such that $M_\alpha\in V$, $^\omega M_\alpha\subset M_{\alpha+1}$ and $H_{\kappa^+}^V\subset M_{\omega_1}:=\bigcup_{\alpha<\omega_1}M_\alpha$.

	\qquad Note that $V$ and $V[g]$ have the same countable sequences since $\col(\omega_1,2^\kappa)$ is ${<}\omega_1$-closed, so we can apply $\sigma$ to the $M_\alpha$'s, resulting in an $M_{\omega_1}$-measure $\mu$ on $\kappa$. Let $j:M_{\omega_1}\to\ult(M_{\omega_1},\mu)$ be the ultrapower embedding. Since we required that $^\omega M_\alpha\subset M_{\alpha+1}$ we get that $\M_{\omega_1}$ is closed under $\omega$-sequences in $V[g]$, making $\mu$ countably complete in $V[g]$. As we also ensured that $H_{\kappa^+}^V\subset\M_{\omega_1}$ we can lift $j$ to an ultrapower embedding $\pi:V\to\ult(V,\mu)\cong N$ with $N$ transitive.
	
	\qquad Since $V$ is closed under $\omega$-sequences in $V[g]$ we get by standard arguments that $N$ is as well, which means that Theorem \ref{theo.Schindler} applies, meaning that $\pi\restr K:K\to K^N$ is an iteration map with critical point $\kappa$, making $\kappa$ measurable in $K$.
}


\end{document}
