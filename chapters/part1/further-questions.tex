\documentclass[../../main]{subfiles}
\pagestyle{fancy}

\begin{document}

\chapter{Further questions}
\thispagestyle{fancy}


\section{Virtually strongs \& supercompacts}

\ques[][ques.remarkableequiv]{
  Are virtually $\theta$-strong cardinals, virtually $\theta$-supercompacts and virtually $\theta$-supercompacts ala Magidor all equivalent, for any uncountable regular cardinal $\theta$?
}



\section{Behaviour in core models}

\ques{
  What happens in larger core models? It seems that in both $L[\mu]$ and $K$ below $0^\pistol$ we get that generically $\theta$-measurables are equivalent to virtually $\theta$-measurables, but the measurable in $L[\mu]$ is virtually measurable and not virtually $\kappa^{++}$-strong. What happens to winning strategies in $\G^\theta_\omega(\kappa)$ then?
}


\section{Separation results}

\ques[][ques.prestrong-equivalence-lmu]{
  Assume $V=L[\mu]$. Is every virtually $\theta$-measurable cardinal also virtually $\theta$-prestrong?
}

\ques{
	Can we find a virtually $\infty$-measurable which isn't measurable?
}


\section{Berkeleys}

Question 1.7 in \cite{RemarkableWilson} asks whether the existence of a non-$\Sigma_2$-reflecting \textit{weakly remarkable} cardinal always implies the existence of an $\omega$-Erd\H os cardinal. Here a weakly remarkable cardinal is a rewording of a virtually prestrong cardinal, and Lemmata 2.5 and 2.8 in the same paper also shows that being $\omega$-Erd\H os is equivalent to being virtually club berkeley and that the least such is also the least virtually berkeley.\footnote{Note that this also shows that virtually club berkeley cardinals and virtually berkeley cardinals are equiconsistent, which is an open question in the non-virtual context.}

\qquad Furthermore, they also showed that a non-$\Sigma_2$-reflecting virtually prestrong cardinal is equivalent to a virtually prestrong cardinal which isn't virtually strong. We can therefore reformulate their question to the following equivalent question.

\ques[Wilson]{
  If there exists a virtually prestrong cardinal which is not virtually strong, is there then a virtually berkeley cardinal?
}

\cite{RemarkableWilson} showed that their question has a positive answer in $L$, which in particular shows that they are equiconsistent. Applying our Theorem \ref{theo.virtchar} we can ask the following related question, where a positive answer to that question would imply a positive answer to Wilson's question.

\ques{
If there exists a cardinal $\kappa$ which is virtually $(\theta,\omega)$-superstrong for arbitrarily large cardinals $\theta>\kappa$, is there then a virtually berkeley cardinal?
}

Our results above at least gives a partially positive result:

\coro[N.]{
  If there exists a virtually $A$-prestrong cardinal for every class $A$ and there are no virtually strong cardinals, then there exists a virtually berkeley cardinal.
}
\proof{
  The assumption implies by definition that $\on$ is virtually prewoodin but not virtually woodin, so Theorem \ref{theo.berkeleyequiv} supplies us with the desired.
}

The assumption that there is a virtually $A$-prestrong cardinal for every class $A$ in the above corollary may seem a bit strong, but Theorem \ref{theo.berkeleyequiv} shows that this is necessary, which might lead one to think that the question could have a negative answer. 



\section{Games}

\ques{
  If $\kappa$ is generically $\theta$-power-measurable, does player II then have a winning strategy in $\G^\theta_\omega(\kappa)$?
}


\section{Ideals}

\ques{
	Is ``$\omega$-distributive $(\kappa,\kappa)$-distributive'' ideal-absolute? Does it correspond to generically power-measurables?
}




\end{document}
