\documentclass[../../main]{subfiles}
\pagestyle{fancy}

\begin{document}

\chapter{Further questions}
\thispagestyle{fancy}

Here we record many open questions related to the content of the preceeding chapters, broadly separated by topic.

\section{Berkeleys}

Question 1.7 in \cite{RemarkableWilson} asks whether the existence of a non-$\Sigma_2$-reflecting \textit{weakly remarkable} cardinal always implies the existence of an $\omega$-Erd\H os cardinal. Here a weakly remarkable cardinal is a rewording of a virtually prestrong cardinal, and Lemmata 2.5 and 2.8 in the same paper also shows that being $\omega$-Erd\H os is equivalent to being virtually club berkeley and that the least such is also the least virtually berkeley.\footnote{Note that this also shows that virtually club berkeley cardinals and virtually berkeley cardinals are equiconsistent, which is an open question in the non-virtual context.}

\qquad Furthermore, they also showed that a non-$\Sigma_2$-reflecting virtually prestrong cardinal is equivalent to a virtually prestrong cardinal which isn't virtually strong. We can therefore reformulate their question to the following equivalent question.

\ques[Wilson]{
  If there exists a virtually prestrong cardinal which is not virtually strong, is there then a virtually berkeley cardinal?
}

\cite{RemarkableWilson} showed that their question has a positive answer in $L$, which in particular shows that they are equiconsistent. Applying our Theorem \ref{theo.virtchar} we can ask the following related question, where a positive answer to that question would imply a positive answer to Wilson's question.

\ques{
If there exists a cardinal $\kappa$ which is virtually $(\theta,\omega)$-superstrong for arbitrarily large cardinals $\theta>\kappa$, is there then a virtually berkeley cardinal?
}

Theorem \ref{theo.berkeleyequiv} from Chapter \ref{chapter.virtual-large-cardinals} at least gives a partially positive result, noting that the assumption by definition implies that $\on$ is virtually prewoodin but not virtually woodin.

\qcoro[N.]{
  If there exists a virtually $A$-prestrong cardinal for every class $A$ and there are no virtually strong cardinals, then there exists a virtually berkeley cardinal.
}

The assumption that there is a virtually $A$-prestrong cardinal for every class $A$ in the above corollary may seem a bit strong, but Theorem \ref{theo.berkeleyequiv} shows that this is necessary, which might lead one to think that the question could have a negative answer. 



\section{Relations between virtuals}

The analysis in Chapter \ref{chapter.virtual-large-cardinals} showed several implication and separation results between the virtual large cardinals. A few of these relations remain open, however.

\ques[][ques.remarkableequiv]{
  Are virtually $\theta$-strong cardinals, virtually $\theta$-supercompacts and virtually $\theta$-Magidor-supercompacts all equivalent, for any uncountable regular cardinal $\theta$?
}

\ques[][ques.prestrong-equivalence]{
  Let $\theta$ be an uncountable cardinal.
  \begin{enumerate}
    \item Is every virtually $\theta$-measurable cardinal also virtually $\theta$-prestrong? What if we assume $V=L[\mu]$ or $V=K$, with $K$ being the core model below a woodin cardinal?
    \item Is every virtually $\theta$-strong cardinal virtually $\theta$-supercompact? Are they equiconsistent?
  \end{enumerate}
}


\section{Indestructibility}

Our original goal concerning indestructibility was to see what indestructibility properties the faintly supercompacts have, whether any analogy with the supercompact cardinals holds. This still remains open.

\ques{
  Do faintly supercompact cardinals have indestructibility properties? For instance, if $\kappa$ is faintly supercompact, does it remain supercompact after forcing with $\add(\kappa, 1)$?
}

We proved several indestructibility properties of the ostensibly stronger notion of \textit{generically setwise supercompacts}, and several questions then arise concerning the nature of these cardinals.

\ques{
  What's the consistency strength of the generically setwise supercompact cardinals? The best upper bound is a virtually extendible, as given by Usuba's Theorem \ref{theo.usuba}, and a lower bound is the trivial faintly supercompact one. What if we require the cardinal to be inaccessible?
}

\ques{
  Is it consistent to have a faintly supercompact cardinal which isn't generically setwise supercompact?
}

\ques{
  Assume there exists no inner model with a woodin cardinal. Can there then exist generically setwise supercompact cardinals in $K$?
}



\section{Games and small embeddings}

Our results in Chapter \ref{chapter.filters-and-games} provide answers to the following questions, which were posed in \cite{HolySchlicht}.

\begin{enumerate}
	\item If $\gamma$ is an uncountable cardinal and the challenger does not have a winning strategy in the game $\G_\gamma^\theta(\kappa)$, does it follow that the judge has one?
	\item If $\omega\leq\alpha\leq\kappa$, are $\alpha$-Ramsey cardinals downwards absolute to the Dodd-Jensen core model?
	\item Does $2$-iterability imply $\omega$-Ramseyness, or conversely?
	\item Does $\kappa$ having the strategic $\kappa$-filter property have the consistency strength of a measurable cardinal?\\
\end{enumerate}

Here the ``challenger'' is player I and the ``judge'' is player II, so this is asking if every $\gamma$-Ramsey is strategic $\gamma$-Ramsey, when $\gamma$ is an uncountable cardinal. Theorem \ref{theo.stratramsey} therefore gives a negative answer to (i) for all uncountable ordinals $\gamma$. Theorem \ref{theo.downK} and Corollary \ref{coro.downK} answer (ii) positively, for $\alpha$-Ramseys with $\alpha$ having uncountable cofinality, and for ${<}\alpha$-Ramseys when $\alpha$ is a limit of limit ordinals. Note that (ii) in the $\alpha=\omega$ case was answered positively in \cite{HolySchlicht}.

\qquad As for (iii), it's mentioned in \cite{HolySchlicht} that Gitman has showed that $\omega$-Ramseys are not in general $2$-iterable by showing that $2$-iterables have strictly stronger consistency strength than the $\omega$-Ramseys, which also follows from Theorem \ref{theo.remlimram} and Theorem 4.8 in \cite{Ramsey2}. Corollary \ref{coro.ind} shows that $\omega$-Ramsey cardinals are $\Delta^2_0$-indescribable, and as $2$-iterables are (at least) $\Pi^1_3$-definable it holds that any $2$-iterable $\omega$-Ramsey cardinal is a limit of $2$-iterables, so that in general $2$-iterables can't be $\omega$-Ramsey either, answering (iii) in the negative. Lastly, Theorem \ref{theo.omegaplusone} gives a positive answer to (iv).

\qquad We conjecture the following two questions to be true. The first is a direct analogue to Theorem \ref{theo.gengame}, and the latter is a suspected analogy between the genuine $n$-Ramsey cardinals and the weakly ineffable cardinals.

\ques{
  If $\kappa$ is faintly $\theta$-power-measurable, does player II then have a winning strategy in $\G^\theta_\omega(\kappa)$?
}

\ques{
	Are genuine $n$-Ramsey cardinals limits of $n$-Ramsey cardinals? We conjecture this to be true, in analogy with the weakly ineffables being limits of weakly compacts. Since ``weakly ineffable = $\Pi^1_1$-indescribability + subtlety'', this might involve some notion of ``$n$-iterated subtlety''. The difference here is that $n$-Ramseys cannot be \textit{equivalent} to $\Pi^1_{2n+1}$-indescribables for consistency reasons, so there is some work to be done.
}

We showed in Theorem \ref{theo.ineff}, see also Corollary \ref{coro.ineff} that completely ineffable cardinals could be characterised in terms of player II having a winning strategy in $\G_\omega^-(\kappa)$. This lends itself to the following question.

\ques{
  Are there higher analogues of ineffability which are equivalent to player II having a winning strategy in $\G_\alpha^-(\kappa)$ for $\alpha>\omega$?
}


\section{Ideal absoluteness}

One can ask of any poset property whether it is ideal-absolute, but we choose to only highlight one particular property here. We saw in Corollary \ref{coro.powerclosed} that ${<}\lambda$-closed faintly power-measurables ``corresponds to'' $(\kappa,\kappa)$-distributive ${<}\lambda$-closed forcings, and in Corollary \ref{coro.ineff} that completely ineffable cardinals ``corresponds to'' $(\kappa,\kappa)$-distributive forcings. In an attempt to find the forcing that corresponds to the faintly power-measurables, we arrive at the following question.

\ques{
	For $\kappa$ a regular cardinal, are the following equivalent?
  \begin{enumerate}
    \item $\kappa$ is faintly power-measurable;
    \item $\kappa$ is ideally power-measurable;
    \item $\kappa$ is $(\kappa,\kappa)$-distributive $\omega$-distributive faintly measurable;
    \item $\kappa$ is $(\kappa,\kappa)$-distributive $\omega$-distributive ideally measurable;
    \item Player II has a winning strategy in $\G_\omega(\kappa)$.
  \end{enumerate}
}




\end{document}
